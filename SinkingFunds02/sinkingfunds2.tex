
\documentclass[
letterpaper,
11pt, % Cambiar a 10 si es que no cabe
oneside,
onecolumn, %twocolumn para dos columnas
article
]{memoir}

\usepackage[spanish,es-nodecimaldot]{babel}
\usepackage[utf8]{inputenc}
\usepackage[T1]{fontenc}
\usepackage{tgtermes} % La fuente a usar, si no compila quitar esta línea
\usepackage[svgnames]{xcolor} % Required for colour specification
\usepackage{blindtext} % Controls the indentation and the space between paragraphs
\usepackage{tikzsymbols} % Emojis
\usepackage{tikz} %Grphics 
\usepackage{empheq} % Hace la hoja tamaño carta
\usetikzlibrary{snakes,positioning, decorations.pathreplacing,decorations.markings,babel} % Diagrams
\usepackage{rotating} % Diagrams
\usepackage{pifont} % Figuras para referenciar
\usepackage{cancel} % To draw diagonal lines through expressions
\usepackage{tabularx} % Tables
\usepackage{multicol} % Multiple columns
\usepackage{enumitem} % Enumerate with diferent bullets
\usepackage{ulem} % Underline fixing code errors of normal \underline{•}
\usepackage{color,soul} % Underline with colors
\medievalpage

% Paquetes para matemáticas
\usepackage{amscd}
\usepackage{amsfonts}
\usepackage{amssymb}
\usepackage{amsmath}
\usepackage{amsthm}
\usepackage{latexsym}
\usepackage{mathrsfs}
\usepackage{bm}
\usepackage{bbm}
\usepackage{mathtools}
\usepackage{listings}
\usepackage[spanish,onelanguage,ruled,linesnumbered]{algorithm2e}
\usepackage{stackengine}
\usepackage[mathscr]{euscript}
\usepackage[scr]{rsfso}
\usepackage{empheq}
\usepackage[final]{microtype}
\usepackage{graphicx} % Para incluir figuras
\usepackage{lipsum}
\usepackage{actuarialsymbol} %Actuarial notation
\usepackage{hyperref}

% Command "alignedbox{}{}" for a box within an align environment
% Source: http://www.latex-community.org/forum/viewtopic.php?f=46&t=8144
\newlength\dlf  % Define a new measure, dlf
\newcommand\alignedbox[2]{
% Argument #1 = before & if there were no box (lhs)
% Argument #2 = after & if there were no box (rhs)
&  % Alignment sign of the line
{
\settowidth\dlf{$\displaystyle #1$}  
    % The width of \dlf is the width of the lhs, with a displaystyle font
\addtolength\dlf{\fboxsep+\fboxrule}  
    % Add to it the distance to the box, and the width of the line of the box
\hspace{-\dlf}  
    % Move everything dlf units to the left, so that & #1 #2 is aligned under #1 & #2
\boxed{#1 #2}
    % Put a box around lhs and rhs
}
}

\setlrmarginsandblock{0.15\paperwidth}{*}{1} % Para onecolumn
\setulmarginsandblock{0.5in}{1.5in}{1}  % Márgenes superior e inferior
\checkandfixthelayout

\parindent=0pt % disables indentation
\parskip=12pt % adds vertical space between paragraphs

\addto{\captionsspanish}{%
  \renewcommand{\bibname}{\Large Referencias}
}

\counterwithout{section}{chapter}
\counterwithout{figure}{chapter}

\makepagestyle{plain}
\makeevenfoot{plain}{\thepage}{}{}
\makeoddfoot{plain}{}{}{\thepage}
\makeevenhead{plain}{}{}{}
\makeoddhead{plain}{}{}{}

\makeatletter %
\makechapterstyle{standard}{
  \setlength{\beforechapskip}{2\baselineskip}
  \setlength{\midchapskip}{0\baselineskip}
  \setlength{\afterchapskip}{2\baselineskip}
  \renewcommand{\chapterheadstart}{\vspace*{\beforechapskip}}
  \renewcommand{\chapnamefont}{\normalfont\Large}
  \renewcommand{\printchaptername}{}
  \renewcommand{\chapternamenum}{\space}
  \renewcommand{\chapnumfont}{\normalfont\Large}
  %\renewcommand{\printchapternum}{\chapnumfont \thechapter.}
  %\renewcommand{\afterchapternum}{\par\nobreak\vskip \midchapskip}
  \renewcommand{\afterchapternum}{ }
  \renewcommand{\printchapternonum}{\vspace*{\midchapskip}\vspace*{5mm}}
  \renewcommand{\chaptitlefont}{\bfseries\LARGE}
  \renewcommand{\printchaptertitle}[1]{\chaptitlefont ##1}
  \renewcommand{\afterchaptertitle}{\par\nobreak\vskip \afterchapskip}
}
\makeatother

\chapterstyle{standard}

\makeatletter %
\makechapterstyle{appendix}{
  \setlength{\beforechapskip}{2\baselineskip}
  \setlength{\midchapskip}{0\baselineskip}
  \setlength{\afterchapskip}{2\baselineskip}
  \renewcommand{\chapterheadstart}{\vspace*{\beforechapskip}}
  \renewcommand{\chapnamefont}{\normalfont\Large}
  \renewcommand{\printchaptername}{\chapnamefont \@chapapp}
  \renewcommand{\chapternamenum}{\space}
  \renewcommand{\chapnumfont}{\normalfont\Large}
  \renewcommand{\printchapternum}{\chapnumfont \thechapter.}
  %\renewcommand{\afterchapternum}{\par\nobreak\vskip \midchapskip}
  \renewcommand{\afterchapternum}{ }
  \renewcommand{\printchapternonum}{\vspace*{\midchapskip}\vspace*{5mm}}
  \renewcommand{\chaptitlefont}{\bfseries\LARGE}
  \renewcommand{\printchaptertitle}[1]{\chaptitlefont ##1}
  \renewcommand{\afterchaptertitle}{\par\nobreak\vskip \afterchapskip}
}
\makeatother

\setlength{\columnseprule}{1pt} %Line between paragraphs

\tikzset{
  % style to apply some styles to each segment of a path
  on each segment/.style={
    decorate,
    decoration={
      show path construction,
      moveto code={},
      lineto code={
        \path [#1]
        (\tikzinputsegmentfirst) -- (\tikzinputsegmentlast);
      },
      curveto code={
        \path [#1] (\tikzinputsegmentfirst)
        .. controls
        (\tikzinputsegmentsupporta) and (\tikzinputsegmentsupportb)
        ..
        (\tikzinputsegmentlast);
      },
      closepath code={
        \path [#1]
        (\tikzinputsegmentfirst) -- (\tikzinputsegmentlast);
      },
    },
  },
  % style to add an arrow in the middle of a path
  end arrow/.style={postaction={decorate,decoration={
        markings,
        mark=at position 0.999 with {\arrow[#1]{stealth}}
      }}},
} % Curved lines

% Declaración de comandos y operadores
\newcommand\RR{\mathbb R}
\newcommand\NN{\mathbb N}
\newcommand\PP{\mathbb P}
\newcommand\dpartial[1]{\frac{\partial}{\partial #1}}
\newcommand\deriv[1]{\frac{d}{d #1}}
\newcommand\integral[4]{\int_{#1}^{#2} #3 \, d#4}
\newcommand*\circled[1]{\tikz[baseline=(char.base)]{
            \node[shape=circle,draw,inner sep=2pt] (char) {#1};}}
\DeclareMathOperator\Ber{Bernoulli}

% Se definen los comandos para escribir teoremas, definiciones y demás.
\theoremstyle{plain}
\newtheorem*{theorem}{Teorema}
\newtheorem*{corollary}{Corolario}
\newtheorem*{lemma}{Lema}
\newtheorem*{proposition}{Proposici\'on}
\theoremstyle{definition}
\newtheorem*{definition}{Definici\'on}
\theoremstyle{remark}
\newtheorem*{remark}{Observaci\'on}

\begin{document}

%%%%%%%%%%%%%%%%%%%%%%%%%
% Aquí va la portada
%%%%%%%%%%%%%%%%%%%%%%%%%

\begin{titlingpage} % Portada

    \raggedleft % Alineada a la derecha
    %\raggedright % Alineada a la izquierda
	
	\vspace*{\baselineskip} % Whitespace at the top of the page
	
	\vspace*{0.25\textheight} % Whitespace before the title
	
	%------------------------------------------------
	%	Cosas del título
	%------------------------------------------------
    
    \vspace*{0.1\textheight}

    {\Huge{\textbf{Sinking Funds}}}\\[\baselineskip] % Aquí va el título
    \vspace*{0.1\textheight}

    %------------------------------------------------
	%	Aquí van los nombres
	%------------------------------------------------
    
    {\Large Eduardo Selim Matínez Mayorga}\\[\baselineskip]
	
	\vfill

\end{titlingpage}

\thispagestyle{empty}

\chapter*{Sinking Funds 2}
\textcolor{magenta}{Recordatorio:}
\begin{multicols}{2}
En amortización con pagos nivelados
\begin{align*}
    L &= R\cdot \ax{\angln j}\\
    &i.e. \quad R= \frac{L}{\ax{\angln j}}
\end{align*}

$$OB_k^{\textit{Amort}} = R\cdot \ax{\angl{n-k}j}$$

$$I_k^{\textit{Amort}} = OB_{k-1}^{\textit{Amort}}\cdot j$$

\columnbreak

En \textit{sinking funds} con depósitos nivelados
\begin{align*}
    L &= D S\angln i\\
    &i.e.\quad D = \frac{L}{S\angln i}
\end{align*}
Servicio de la deuda: $L_J$\\
Desembolso global periódico
\begin{align*}
    D+L_j &= \frac{L}{S\angln i} + L_j\\
    &= L\bigg(\frac{1}{S\angln i} + j\bigg)
\end{align*}
$$OB_k^{\textit{SF}} = L- DS\angl{k}i$$
$$I_k^{\textit{SF}} = OB_{k-1}^{\textit{SF}}\cdot i$$
\end{multicols}

¿Qué pasa si $i=j$?\\
En \textit{sinking fund}, el desembolso global periódico
\begin{align*}
    D+ L_j &= \frac{L}{S\angln i} + L_j = L\bigg(\frac{1}{S\angln i} +j \bigg)\\
    &= L\bigg(\frac{1}{S\angln \textcolor{red}{j}} +j \bigg)\\
    &= L\bigg(\frac{1 + j\cdot S\angln j}{S\angln j} \bigg)\\
    &= L\bigg[\frac{1+ (1+j)^n - 1}{S\angln j}\bigg]\\
    &= L\bigg[\frac{(1+j)^n}{S\angln j}\bigg] = L \bigg[\frac{1}{(1+j)^n S\angln j}\bigg]\\
    &= L\bigg[\frac{1}{\ax{\angln j}}\bigg] \\
    &= \frac{L}{\ax{\angln j}} = \textcolor{red}{R \longrightarrow \text{¡El $R$ de amortización}!}
\end{align*}

Es decir, en el caso de que $i=j$, el desembolso periódico que hace el prestatario (el que pide prestado) es el mismo bajo amortización que bajo \textit{sinking fund}
\begin{align*}
    OB_k^{SF} &= L - DS\angl{k}i = L - \frac{L}{S\angln i}S\angl{k}i\\
    &= L - \frac{L}{S\angln i}S\angl{k}i \frac{(1+i)^{-n}}{(1+i)^{-n}}\\
    &= L - \frac{L S\angl{k}i (1+i)^{-n}}{\ax{\angln i}}\\
    &= \frac{L}{\ax{\angln i}}\bigg[\ax{\angln i} - S\angl{k}i (1+i)^{-n}\bigg]\\
    &= \frac{L}{\ax{\angln i}}\bigg[\ax{\angln i} - \big(1+(1+i)+\dotsc+(1+i)^{k-1}\big)(1+i)^{-n}\bigg]\\
    &= \frac{L}{\ax{\angln i}}\bigg[\ax{\angln i} - (1+i)^{-n} - (1+i)^{-n+1}-\dotsc-(1+i)^{-n+k-1}\bigg]\\
    &= \frac{L}{\ax{\angln i}}\bigg[\ax{\angln i} - (1+i)^{-(n-k+1)}-\dotsc- (1+i)^{-(n-1)}-(1+i)^{-n}\bigg]\\
    &= \frac{L}{\ax{\angln i}}\bigg[\ax{\angl{n-k}i}\bigg]\\
    &= \frac{L}{\ax{\angln \textcolor{red}{j}}}\cdot\ax{\angl{n-k}\textcolor{red}{j}} = \textcolor{red}{R}\cdot \ax{\angl{n-k}j}
\end{align*}
Es decir, si $j=i$
$$OB_k^{SF} = L -DS\angl{k}i = R\cdot\ax{\angl{n-k}j} = OB_k^{Amort}$$
$$\therefore OB_k^{SF} = OB_k^{Amort}$$
$i.e.$ si $i=j$, entonces el saldo restante al final de cada periodo es el mismo bajo ambos métodos.\\
También 
$$I_k^{Amort} = OB_{k-1}^{Amort}\cdot j = OB_{k-1}^{SF}\cdot i = I_k^{SF}$$
Es decir, si $i=j$, el interés que gana el \textit{SF} es el mismo que el que se paga de interés sobre la deuda.


\end{document}
