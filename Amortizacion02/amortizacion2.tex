
\documentclass[
letterpaper,
11pt, % Cambiar a 10 si es que no cabe
oneside,
onecolumn, %twocolumn para dos columnas
article
]{memoir}

\usepackage[spanish,es-nodecimaldot]{babel}
\usepackage[utf8]{inputenc}
\usepackage[T1]{fontenc}
\usepackage{tgtermes} % La fuente a usar, si no compila quitar esta línea
\usepackage[svgnames]{xcolor} % Required for colour specification
\usepackage{blindtext} % Controls the indentation and the space between paragraphs
\usepackage{tikzsymbols} % Emojis
\usepackage{tikz} %Grphics 
\usepackage{empheq} % Hace la hoja tamaño carta
\usetikzlibrary{snakes,positioning, decorations.pathreplacing,decorations.markings,babel} % Diagrams
\usepackage{rotating} % Diagrams
\usepackage{pifont} % Figuras para referenciar
\usepackage{cancel} % To draw diagonal lines through expressions
\usepackage{tabularx} % Tables
\usepackage{multicol} % Multiple columns
\usepackage{enumitem} % Enumerate with diferent bullets
\usepackage{ulem} % Underline fixing code errors of normal \underline{•}
\usepackage{color,soul} % Underline with colors
\medievalpage

% Paquetes para matemáticas
\usepackage{amscd}
\usepackage{amsfonts}
\usepackage{amssymb}
\usepackage{amsmath}
\usepackage{amsthm}
\usepackage{latexsym}
\usepackage{mathrsfs}
\usepackage{bm}
\usepackage{bbm}
\usepackage{mathtools}
\usepackage{listings}
\usepackage[spanish,onelanguage,ruled,linesnumbered]{algorithm2e}
\usepackage{stackengine}
\usepackage[mathscr]{euscript}
\usepackage[scr]{rsfso}
\usepackage{empheq}
\usepackage[final]{microtype}
\usepackage{graphicx} % Para incluir figuras
\usepackage{lipsum}
\usepackage{actuarialsymbol} %Actuarial notation
\usepackage{hyperref}

% Command "alignedbox{}{}" for a box within an align environment
% Source: http://www.latex-community.org/forum/viewtopic.php?f=46&t=8144
\newlength\dlf  % Define a new measure, dlf
\newcommand\alignedbox[2]{
% Argument #1 = before & if there were no box (lhs)
% Argument #2 = after & if there were no box (rhs)
&  % Alignment sign of the line
{
\settowidth\dlf{$\displaystyle #1$}  
    % The width of \dlf is the width of the lhs, with a displaystyle font
\addtolength\dlf{\fboxsep+\fboxrule}  
    % Add to it the distance to the box, and the width of the line of the box
\hspace{-\dlf}  
    % Move everything dlf units to the left, so that & #1 #2 is aligned under #1 & #2
\boxed{#1 #2}
    % Put a box around lhs and rhs
}
}

\setlrmarginsandblock{0.15\paperwidth}{*}{1} % Para onecolumn
\setulmarginsandblock{0.5in}{1.5in}{1}  % Márgenes superior e inferior
\checkandfixthelayout

\parindent=0pt % disables indentation
\parskip=12pt % adds vertical space between paragraphs

\addto{\captionsspanish}{%
  \renewcommand{\bibname}{\Large Referencias}
}

\counterwithout{section}{chapter}
\counterwithout{figure}{chapter}

\makepagestyle{plain}
\makeevenfoot{plain}{\thepage}{}{}
\makeoddfoot{plain}{}{}{\thepage}
\makeevenhead{plain}{}{}{}
\makeoddhead{plain}{}{}{}

\makeatletter %
\makechapterstyle{standard}{
  \setlength{\beforechapskip}{2\baselineskip}
  \setlength{\midchapskip}{0\baselineskip}
  \setlength{\afterchapskip}{2\baselineskip}
  \renewcommand{\chapterheadstart}{\vspace*{\beforechapskip}}
  \renewcommand{\chapnamefont}{\normalfont\Large}
  \renewcommand{\printchaptername}{}
  \renewcommand{\chapternamenum}{\space}
  \renewcommand{\chapnumfont}{\normalfont\Large}
  %\renewcommand{\printchapternum}{\chapnumfont \thechapter.}
  %\renewcommand{\afterchapternum}{\par\nobreak\vskip \midchapskip}
  \renewcommand{\afterchapternum}{ }
  \renewcommand{\printchapternonum}{\vspace*{\midchapskip}\vspace*{5mm}}
  \renewcommand{\chaptitlefont}{\bfseries\LARGE}
  \renewcommand{\printchaptertitle}[1]{\chaptitlefont ##1}
  \renewcommand{\afterchaptertitle}{\par\nobreak\vskip \afterchapskip}
}
\makeatother

\chapterstyle{standard}

\makeatletter %
\makechapterstyle{appendix}{
  \setlength{\beforechapskip}{2\baselineskip}
  \setlength{\midchapskip}{0\baselineskip}
  \setlength{\afterchapskip}{2\baselineskip}
  \renewcommand{\chapterheadstart}{\vspace*{\beforechapskip}}
  \renewcommand{\chapnamefont}{\normalfont\Large}
  \renewcommand{\printchaptername}{\chapnamefont \@chapapp}
  \renewcommand{\chapternamenum}{\space}
  \renewcommand{\chapnumfont}{\normalfont\Large}
  \renewcommand{\printchapternum}{\chapnumfont \thechapter.}
  %\renewcommand{\afterchapternum}{\par\nobreak\vskip \midchapskip}
  \renewcommand{\afterchapternum}{ }
  \renewcommand{\printchapternonum}{\vspace*{\midchapskip}\vspace*{5mm}}
  \renewcommand{\chaptitlefont}{\bfseries\LARGE}
  \renewcommand{\printchaptertitle}[1]{\chaptitlefont ##1}
  \renewcommand{\afterchaptertitle}{\par\nobreak\vskip \afterchapskip}
}
\makeatother

\setlength{\columnseprule}{1pt} %Line between paragraphs

\tikzset{
  % style to apply some styles to each segment of a path
  on each segment/.style={
    decorate,
    decoration={
      show path construction,
      moveto code={},
      lineto code={
        \path [#1]
        (\tikzinputsegmentfirst) -- (\tikzinputsegmentlast);
      },
      curveto code={
        \path [#1] (\tikzinputsegmentfirst)
        .. controls
        (\tikzinputsegmentsupporta) and (\tikzinputsegmentsupportb)
        ..
        (\tikzinputsegmentlast);
      },
      closepath code={
        \path [#1]
        (\tikzinputsegmentfirst) -- (\tikzinputsegmentlast);
      },
    },
  },
  % style to add an arrow in the middle of a path
  end arrow/.style={postaction={decorate,decoration={
        markings,
        mark=at position 0.999 with {\arrow[#1]{stealth}}
      }}},
} % Curved lines

% Declaración de comandos y operadores
\newcommand\RR{\mathbb R}
\newcommand\NN{\mathbb N}
\newcommand\PP{\mathbb P}
\newcommand\dpartial[1]{\frac{\partial}{\partial #1}}
\newcommand\deriv[1]{\frac{d}{d #1}}
\newcommand\integral[4]{\int_{#1}^{#2} #3 \, d#4}
\newcommand*\circled[1]{\tikz[baseline=(char.base)]{
            \node[shape=circle,draw,inner sep=2pt] (char) {#1};}}
\DeclareMathOperator\Ber{Bernoulli}

% Se definen los comandos para escribir teoremas, definiciones y demás.
\theoremstyle{plain}
\newtheorem*{theorem}{Teorema}
\newtheorem*{corollary}{Corolario}
\newtheorem*{lemma}{Lema}
\newtheorem*{proposition}{Proposici\'on}
\theoremstyle{definition}
\newtheorem*{definition}{Definici\'on}
\theoremstyle{remark}
\newtheorem*{remark}{Observaci\'on}

\begin{document}

%%%%%%%%%%%%%%%%%%%%%%%%%
% Aquí va la portada
%%%%%%%%%%%%%%%%%%%%%%%%%

\begin{titlingpage} % Portada

    \raggedleft % Alineada a la derecha
    %\raggedright % Alineada a la izquierda
	
	\vspace*{\baselineskip} % Whitespace at the top of the page
	
	\vspace*{0.25\textheight} % Whitespace before the title
	
	%------------------------------------------------
	%	Cosas del título
	%------------------------------------------------
    
    \vspace*{0.1\textheight}

    {\Huge{\textbf{Amortización}}}\\[\baselineskip] % Aquí va el título
    \vspace*{0.1\textheight}

    %------------------------------------------------
	%	Aquí van los nombres
	%------------------------------------------------
    
    {\Large Eduardo Selim Matínez Mayorga}\\[\baselineskip]
	
	\vfill

\end{titlingpage}

\thispagestyle{empty}


\chapter*{Amortización de un bono cuponado}
\textcolor{magenta}{Recordatorio:} Un bono es un instrumento financiero que, a cambio de un pago (el precio del bono), el comprador tiene derecho de recibir un flujo de efectivo futuro.\\
Los dos tipos de bonos que vimos son:
\begin{itemize}
    \item Bono cupón-cero
    \item Bono cuponado nivelado
\end{itemize}

En particular, en un bono cuponado nivelado.
\begin{center}
    

\tikzset{every picture/.style={line width=0.75pt}} %set default line width to 0.75pt        

\begin{tikzpicture}[x=0.75pt,y=0.75pt,yscale=-1,xscale=1]
%uncomment if require: \path (0,300); %set diagram left start at 0, and has height of 300

%Straight Lines [id:da34657222641698626] 
\draw    (141,191.9) -- (505,191.9) ;
%Straight Lines [id:da8320031647356826] 
\draw    (190.83,187.23) -- (190.83,197.4) ;
%Straight Lines [id:da908003181732433] 
\draw    (456.33,187.07) -- (456.33,197.23) ;
%Straight Lines [id:da7312982626415365] 
\draw    (149.83,187.23) -- (149.83,197.4) ;
%Straight Lines [id:da5763637162450717] 
\draw    (230.83,187.23) -- (230.83,197.4) ;
%Straight Lines [id:da30794614098512674] 
\draw    (309.83,187.23) -- (309.83,197.4) ;
%Straight Lines [id:da37656947767695326] 
\draw    (349.83,187.73) -- (349.83,197.9) ;
%Straight Lines [id:da6195061540260131] 
\draw    (390.33,187.23) -- (390.33,197.4) ;
%Straight Lines [id:da5214754470951718] 
\draw    (500.83,187.07) -- (500.83,197.23) ;

% Text Node
\draw (255.6,197.5) node [anchor=north west][inner sep=0.75pt]   [align=left] {. . .};
% Text Node
\draw (144.9,201.7) node [anchor=north west][inner sep=0.75pt]  [font=\small]  {$0$};
% Text Node
\draw (185.4,201.7) node [anchor=north west][inner sep=0.75pt]  [font=\small]  {$1$};
% Text Node
\draw (226.4,201.7) node [anchor=north west][inner sep=0.75pt]  [font=\small]  {$2$};
% Text Node
\draw (293.8,202.3) node [anchor=north west][inner sep=0.75pt]  [font=\small]  {$k-1$};
% Text Node
\draw (439.8,201.8) node [anchor=north west][inner sep=0.75pt]  [font=\small]  {$n-1$};
% Text Node
\draw (345.8,202.8) node [anchor=north west][inner sep=0.75pt]  [font=\small]  {$k$};
% Text Node
\draw (374.3,202.8) node [anchor=north west][inner sep=0.75pt]  [font=\small]  {$k+1$};
% Text Node
\draw (494.8,200.3) node [anchor=north west][inner sep=0.75pt]  [font=\small]  {$n$};
% Text Node
\draw (411.1,199) node [anchor=north west][inner sep=0.75pt]   [align=left] {. . .};
% Text Node
\draw (492.9,169.7) node [anchor=north west][inner sep=0.75pt]  [font=\small]  {$F_{r}$};
% Text Node
\draw (143.9,220.7) node [anchor=north west][inner sep=0.75pt]  [font=\small]  {$P$};
% Text Node
\draw (449.9,169.7) node [anchor=north west][inner sep=0.75pt]  [font=\small]  {$F_{r}$};
% Text Node
\draw (225.3,172.9) node [anchor=north west][inner sep=0.75pt]  [font=\small]  {$F_{r}$};
% Text Node
\draw (344.3,173.7) node [anchor=north west][inner sep=0.75pt]  [font=\small]  {$F_{r}$};
% Text Node
\draw (304.1,172.9) node [anchor=north west][inner sep=0.75pt]  [font=\small]  {$F_{r}$};
% Text Node
\draw (185.9,171.9) node [anchor=north west][inner sep=0.75pt]  [font=\small]  {$F_{r}$};
% Text Node
\draw (145.1,171.3) node [anchor=north west][inner sep=0.75pt]  [font=\small]  {$F_{r}$};
% Text Node
\draw (384.3,174.3) node [anchor=north west][inner sep=0.75pt]  [font=\small]  {$F_{r}$};


\end{tikzpicture}

\end{center}

\begin{itemize}
    \item[ ] $F$: Valor nominal (Face value o valor facial)
    \item[ ] $r$: Tasa cupón
    \item[ ] $P$: Precio del bono
    \item[ ] $j$: Tasa de rendimiento del bono
    \item[ ] $n$: Plazo del bono o número de cupones
    \item[ ] $C$: Valor de redención
    \item[ ] $F_r$: Cupón
\end{itemize}

Vimos al menos dos fórmulas para calcular el precio de un bono:\\
\textit{Fórmula fundamental}
$$P=F_r\ax{\angln j} + C(1+j)^{-n}$$
\textit{Fórmula premio/descuento}
$$P=(F_r-C_j) \ax{\angln j} + C$$

\textcolor{blue}{Recordatorio:}
\begin{definition}
\begin{itemize}
    \item Se dice que un bono se redime a la par si $C=F$
    \item Se dice que un bono se redime con premio si $C>F$
    \item Se dice que un bono se redime con descuento si $C<F$
    \item Se dice que un bono se redime a la par si $P=C$
    \item Se dice que un bono se redime con premio si $P>C$
    \item Se dice que un bono se redime con descuento si $P<C$
\end{itemize}
\end{definition}

Nótese que:\\
\textcolor{purple}{$\bullet$}\textit{ El bono que se compra a la par:}
\begin{align*}
    P=C &\iff C+(F_r-C_j)\ax{\angln} =C \iff (F_r-C_j)\ax{\angln} = \varnothing\\
    &\iff F_r-C_j=0\textit{ pues } \ax{\angln}>0\\
    &\iff F_r=C_j
\end{align*}
Es decir, el bono se compra a la par si $F_r=C_j$

\textcolor{purple}{$\bullet$}\begin{align*}
    P>C &\iff C+(F_r-C_j)\ax{\angln}=C \iff (F_r -C_j)\ax{\angln}>\varnothing\\
    &\iff F_r-C_j>0 \textit{ pues }\ax{\angln}>0\\
    &\iff F_r> C_j
\end{align*}
Es decir, el bono se compra con premio si $F_r>C_j$

\textcolor{purple}{$\bullet$} $P<C\iff F_r<C_j$
Es decir, el bono se compra con descuento si $F_r<C_j$\\
También se puede considerar a la compra/venta de un bono de la siguiente forma:
\begin{itemize}
    \item El emisor \textit{(quien vende el bono)} pide hoy P \textit{(al comprador)} y pagará $C$ al término de $n$ periodos
    \item En las fechas $1,2,...,n$ \textit{el emisor} paga \textit{al comprador} una cantidad $F_r$ al comprador del bono.
\end{itemize}

¿Se puede considerar al cupón $F_r$ como pago de interés sobre el préstamo de $P$?\\
\textcolor{blue}{Depende como sea $P$ con respecto a $C$. Vamos a analizarlo poco a poco.}\\
Para un préstamo de $L=P$, ¿cuánto interés se debe después del 1er periodo?
\begin{align*}
    I_1 :&=L_j =P_j = [C+(F_r-C_j)\ax{\angln}]\cdot j\\
    &= C_j + (F_r-C_j)\cdot j\cdot\ax{\angln j} = C_j+ (F_r-C_j)j\bigg(\frac{1-v^n}{j}\bigg)\\
    &= C_j + (F_r-C_j)- (F_r-C_j)v^n = F_r - (F_r-C_j)v^n
\end{align*}
¿Cuánto pagó el emisor al comprador después de un periodo?\\
\textcolor{blue}{$F_R$, el cupón}

¿Cuánto es el neto con ese 1er cupón?\\
\textcolor{blue}{$P_1:= F_r -I_1 = F_r-[F_r-(F_r-C_j)v^n] = (F_r-C_j)v^n$}

¿Cuánto debe de la deuda original después del primer pago de cupón?
\begin{align*}
    OB_1 &= OB_0 -P_1\\
    &= L -P_1 = P - P_1\\
    &= [C + (F_r -C_j)\ax{\angln}] - (F_r-C_j)v^n\\
    &= C+ (F_r-C_j)(\ax{\angln} - v^n)\\
    C + (F_r - C_j)\ax{\angl{n-1}j}
\end{align*}

¿Cuánto interés se debe después del 2o periodo?
\begin{align*}
    I_2 :&= OB_1\cdot j = [C + (F_r-C_j)\ax{\angl{n-1}}]\cdot j\\
    &= C_j + (F_r-C_j)j\cdot\ax{\angl{n-1}j}\\
    &= C_j + (F_r -C_j)j\Big(\frac{1-v^{n-1}}{j}\Big)\\
    &= C_j + (F_r-C_j)(1-v^{n-1})\\
    &= C_j+(F_r-C_j)-(F_r-C_j)v^{n-1}\\
    &= F_r - (F_r-C_j)v^{n-1}
\end{align*}

¿Cuánto pagó el emisor al comprador en el 2o periodo?\\
\textcolor{blue}{$F_r$, el cupón}\\

¿Cuánto es el neto con ese 2do cupón?
\begin{align*}
    P_2 :&= F_r - I_2 = F_r-(F_r-(F_r-C_j)v^{n-1})\\
    &= (F_r-C_j)v^{n-1}
\end{align*}

¿Cuánto se debe de la deuda original después del segundo pago de cupón?
\begin{align*}
    OB_2 :&= OB_1-P_2\\
    &= [C+(F_r-C_j)\ax{\angl{n-1}}] - (F_r-C_j)v^{n-1}\\
    &= C+(F_r-C_j)[\ax{\angl{n-1}}-v^{n-1}]\\
    &= C+(F_r-C_j)\ax{\angl{n-2}}
\end{align*}

¿Cuánto se debe después del tercer periodo?
\begin{align*}
    I_3 :&= OB_2\cdot j = [C + (F_r-C_j)\ax{\angl{n-2}}]\cdot j\\
    &= C_j + (F_r-C_j)(1-v^{n-2})\\
    &= F_r - (F_r-C_j)v^{n-2}
\end{align*}

¿Cuánto pagó el emisor al comprador en el 3er periodo?\\
\textcolor{blue}{$F_r$, el cupón}\\

¿Cuánto es el neto con ese 3er cupón?
\begin{align*}
    P_3 :&= F_r - I_3 = F_r-(F_r-(F_r-C_j)v^{n-2})\\
    &= (F_r-C_j)v^{n-2}
\end{align*}

¿Cuánto se debe de la deuda original después del tercer pago de cupón?
\begin{align*}
    OB_3 :&= OB_2-P_3\\
    &= [C+(F_r-C_j)\ax{\angl{n-2}}] - (F_r-C_j)v^{n-2}\\
    &= C+(F_r-C_j)[\ax{\angl{n-2}}-v^{n-2}]\\
    &= C+(F_r-C_j)\ax{\angl{n-3}}
\end{align*}

\textit{Hay un patrón...}

En general, después del \textit{k-ésimo} periodo, ¿cuánto debe de interés el emisor al comprador?
$$I_k:= j\cdot OB_{k-1} = F_r - (F_rC_j)v^{n-k+1}$$

¿Cuánto pagó el emisor al comprador en el \textit{k-ésimo} periodo?\\
\textcolor{blue}{$F_r$, el cupón}\\

¿Cuánto es el neto con ese \textit{k-ésimo} cupón?
\begin{align*}
    P_k :&= F_r-I_k\\
    &= (F_r-C_j)v^{n-k+1}
\end{align*}

¿Cuánto debe de la deuda original después del \textit{k-ésimo} pago de cupón?
\begin{align*}
    OB_k &= OB_{k-1} - P_k\\
    &= C+ (F_r-C_j)\ax{\angl{n-k}}
\end{align*}
\textcolor{blue}{¿Cuánto vale $OB_n$?} ¿Cuánto me debes después de pagarme todos los cupones? \\
Respuesta: $C$\\
La matemática es noble/consistente
$$OB_n = C + (F_r - C_j)\ax{\angl{n-n}} = C + (F_r-C_j)\cdot 0 = C$$
Hecho que hace sentido pues después de que se pagaron todos los cupones ya sólo queda por pagar $C$.

En resumen:
\begin{enumerate}
    \item $I_K$: Interés que el emisor debe en el periodo $k$ al comprador
    \item $F_r$: Pago cupón
    \item $P_k:=F_r-I_k$ y representa el capital principal que se pagó después del \textit{k-ésimo} cupón
    \item $OB_k=OB_{k-1}-P_k$ y representa el saldo insoluto después de que se pagó el \textit{k-ésimo} cupón
\end{enumerate}

\textit{Observación}
\begin{align*}
    OB_k &= C+ (F_r-C_j)\ax{\angl{n-k}j}\\
    &= C+F_r\cdot \ax{\angl{n-k}j} - C_j\ax{\angl{n-k}j}\\
    &= C + F_r\ax{\angl{n-k}j} -C_j\Big[\frac{1-v^{n-k}}{j}\Big]\\
    &= C + F_r\ax{\angl{n-k}j} - C+Cv^{n-k}\\
    &= F_r\ax{\angl{n-k}j} + Cv^{n-k}
\end{align*}
Lo anterior representa el valor presente del flujo restante.
\begin{center}
    

\tikzset{every picture/.style={line width=0.75pt}} %set default line width to 0.75pt        

\begin{tikzpicture}[x=0.75pt,y=0.75pt,yscale=-1,xscale=1]
%uncomment if require: \path (0,300); %set diagram left start at 0, and has height of 300

%Straight Lines [id:da20079718076773256] 
\draw    (123.4,148.7) -- (487.4,148.7) ;
%Straight Lines [id:da20140309027965386] 
\draw    (173.23,144.03) -- (173.23,154.2) ;
%Straight Lines [id:da5849034461841619] 
\draw    (438.73,143.87) -- (438.73,154.03) ;
%Straight Lines [id:da9377575438639768] 
\draw    (132.23,144.03) -- (132.23,154.2) ;
%Straight Lines [id:da639336669705882] 
\draw    (213.23,144.03) -- (213.23,154.2) ;
%Straight Lines [id:da003432744363729423] 
\draw    (292.23,144.03) -- (292.23,154.2) ;
%Straight Lines [id:da6792924625244605] 
\draw    (332.23,144.53) -- (332.23,154.7) ;
%Straight Lines [id:da8685383153988511] 
\draw    (372.73,144.03) -- (372.73,154.2) ;
%Straight Lines [id:da24814193718307598] 
\draw    (483.23,143.87) -- (483.23,154.03) ;
%Shape: Free Drawing [id:dp9065786106999266] 
\draw  [color={rgb, 255:red, 9; green, 0; blue, 255 }  ,draw opacity=1 ][line width=0.75] [line join = round][line cap = round] (382.27,131.67) .. controls (380.64,131.67) and (381.24,127.85) .. (378.27,127.67) .. controls (374.54,127.43) and (370.79,127.42) .. (367.07,127.67) .. controls (363.11,127.93) and (361.47,149.39) .. (371.07,150.87) .. controls (382.12,152.57) and (388.59,145.9) .. (381.47,131.67) ;
%Shape: Free Drawing [id:dp02217973805327067] 
\draw  [color={rgb, 255:red, 255; green, 0; blue, 0 }  ,draw opacity=1 ][line width=0.75] [line join = round][line cap = round] (182.27,129.27) .. controls (181.03,130.4) and (165.47,143.27) .. (165.47,146.07) ;
%Shape: Free Drawing [id:dp3184191538770488] 
\draw  [color={rgb, 255:red, 255; green, 0; blue, 0 }  ,draw opacity=1 ][line width=0.75] [line join = round][line cap = round] (165.47,129.27) .. controls (173.6,129.27) and (179.57,138.64) .. (183.07,144.47) ;
%Shape: Free Drawing [id:dp40282330529079835] 
\draw  [color={rgb, 255:red, 255; green, 0; blue, 0 }  ,draw opacity=1 ][line width=0.75] [line join = round][line cap = round] (207.07,131.67) .. controls (212.69,131.67) and (231.07,138.05) .. (231.07,146.07) ;
%Shape: Free Drawing [id:dp913307588863777] 
\draw  [color={rgb, 255:red, 255; green, 0; blue, 0 }  ,draw opacity=1 ][line width=0.75] [line join = round][line cap = round] (226.27,128.47) .. controls (218.06,128.47) and (208.67,136.4) .. (208.67,143.67) ;
%Shape: Free Drawing [id:dp2848047534212973] 
\draw  [color={rgb, 255:red, 255; green, 0; blue, 0 }  ,draw opacity=1 ][line width=0.75] [line join = round][line cap = round] (288.67,131.67) .. controls (292.32,133.49) and (305.01,143.67) .. (306.27,143.67) ;
%Shape: Free Drawing [id:dp42447362878922834] 
\draw  [color={rgb, 255:red, 255; green, 0; blue, 0 }  ,draw opacity=1 ][line width=0.75] [line join = round][line cap = round] (302.27,131.67) .. controls (293.24,131.67) and (289.95,138.97) .. (283.07,141.27) ;
%Shape: Free Drawing [id:dp7261324650621672] 
\draw  [color={rgb, 255:red, 255; green, 0; blue, 0 }  ,draw opacity=1 ][line width=0.75] [line join = round][line cap = round] (342.27,132.47) .. controls (333.74,132.47) and (328.35,142.87) .. (323.87,142.87) ;
%Shape: Free Drawing [id:dp4233023651973924] 
\draw  [color={rgb, 255:red, 255; green, 0; blue, 0 }  ,draw opacity=1 ][line width=0.75] [line join = round][line cap = round] (324.67,131.67) .. controls (331.29,135.45) and (340.26,138.46) .. (345.47,143.67) ;
%Curve Lines [id:da03184478790222589] 
\draw [color={rgb, 255:red, 0; green, 6; blue, 255 }  ,draw opacity=1 ]   (374.4,126.8) .. controls (358.31,118.69) and (349.57,123.34) .. (339.82,130) ;
\draw [shift={(338.27,131.07)}, rotate = 325.3] [color={rgb, 255:red, 0; green, 6; blue, 255 }  ,draw opacity=1 ][line width=0.75]    (10.93,-3.29) .. controls (6.95,-1.4) and (3.31,-0.3) .. (0,0) .. controls (3.31,0.3) and (6.95,1.4) .. (10.93,3.29)   ;
%Curve Lines [id:da3206870602695284] 
\draw [color={rgb, 255:red, 0; green, 6; blue, 255 }  ,draw opacity=1 ]   (439.07,125.87) .. controls (398.3,104.11) and (352.92,109.44) .. (333.8,129.24) ;
\draw [shift={(332.67,130.47)}, rotate = 311.5] [color={rgb, 255:red, 0; green, 6; blue, 255 }  ,draw opacity=1 ][line width=0.75]    (10.93,-3.29) .. controls (6.95,-1.4) and (3.31,-0.3) .. (0,0) .. controls (3.31,0.3) and (6.95,1.4) .. (10.93,3.29)   ;
%Curve Lines [id:da5158322028124752] 
\draw [color={rgb, 255:red, 0; green, 6; blue, 255 }  ,draw opacity=1 ]   (480.67,126.87) .. controls (439.11,95.51) and (353,101.22) .. (332.25,121.04) ;
\draw [shift={(331.07,122.27)}, rotate = 311.5] [color={rgb, 255:red, 0; green, 6; blue, 255 }  ,draw opacity=1 ][line width=0.75]    (10.93,-3.29) .. controls (6.95,-1.4) and (3.31,-0.3) .. (0,0) .. controls (3.31,0.3) and (6.95,1.4) .. (10.93,3.29)   ;

% Text Node
\draw (238,154.3) node [anchor=north west][inner sep=0.75pt]   [align=left] {. . .};
% Text Node
\draw (127.3,158.5) node [anchor=north west][inner sep=0.75pt]  [font=\small]  {$0$};
% Text Node
\draw (167.8,158.5) node [anchor=north west][inner sep=0.75pt]  [font=\small]  {$1$};
% Text Node
\draw (208.8,158.5) node [anchor=north west][inner sep=0.75pt]  [font=\small]  {$2$};
% Text Node
\draw (276.2,159.1) node [anchor=north west][inner sep=0.75pt]  [font=\small]  {$k-1$};
% Text Node
\draw (422.2,158.6) node [anchor=north west][inner sep=0.75pt]  [font=\small]  {$n-1$};
% Text Node
\draw (328.2,159.6) node [anchor=north west][inner sep=0.75pt]  [font=\small]  {$k$};
% Text Node
\draw (356.7,159.6) node [anchor=north west][inner sep=0.75pt]  [font=\small]  {$k+1$};
% Text Node
\draw (477.2,157.1) node [anchor=north west][inner sep=0.75pt]  [font=\small]  {$n$};
% Text Node
\draw (393.5,155.8) node [anchor=north west][inner sep=0.75pt]   [align=left] {. . .};
% Text Node
\draw (460.3,126.5) node [anchor=north west][inner sep=0.75pt]  [font=\small]  {$F_{r} +C$};
% Text Node
\draw (126.3,177.5) node [anchor=north west][inner sep=0.75pt]  [font=\small]  {$P$};
% Text Node
\draw (432.3,126.5) node [anchor=north west][inner sep=0.75pt]  [font=\small]  {$F_{r}$};
% Text Node
\draw (207.7,129.7) node [anchor=north west][inner sep=0.75pt]  [font=\small]  {$F_{r}$};
% Text Node
\draw (326.7,130.5) node [anchor=north west][inner sep=0.75pt]  [font=\small]  {$F_{r}$};
% Text Node
\draw (286.5,129.7) node [anchor=north west][inner sep=0.75pt]  [font=\small]  {$F_{r}$};
% Text Node
\draw (168.3,128.7) node [anchor=north west][inner sep=0.75pt]  [font=\small]  {$F_{r}$};
% Text Node
\draw (366.7,131.1) node [anchor=north west][inner sep=0.75pt]  [font=\small]  {$F_{r}$};


\end{tikzpicture}

\end{center}

$$\textcolor{blue}{\boxed{\therefore OB_k = C+(F_r-C_j)\ax{\angl{n-k}j} = F_r\ax{\angl{n-k}j} + C(1+j)^{-(n-k)}}}$$

\textcolor{purple}{$\bullet$} Es decir, $OB_k$ también se puede interpretar como el valor del bono después del \textit{k-ésimo} cupón, es por esto que en el contexto de bonos se denota a $OB_k$ como $B_k$ y se conoce como \textit{Book value} después del \textit{k-ésimo} cupón.

\textcolor{purple}{$\bullet$} Regresando a la pregunta original\\
¿Se puede considerar a los cupones como pago de interés de la deuda?\\
\textcolor{blue}{Depende. Hay que ser más preciso con la pregunta}\\
Es decir, ¿$F_r=I_k$?\\

Recordatorio
$$I_k := j\cdot OB_{k-1} = F_r - (F_r - C_j)v^{n-k+1}$$

\textcolor{purple}{$\bullet$} \begin{align*}
    F_r = I_k &\iff F_r =F_r - (F_r-C_j)v^{n-k+1}\\
    &\iff (F_r-C_j)v^{n-k+1} =0\\
    &\iff F_rC_j = 0 \textit{ pues } v^{n-k+1}>0\\
    &\iff F_r=C_j\\
    &\iff \textit{ El bono se compara a la par}
\end{align*}
Que $F_r=C_j$ es poco común, lo normal es que $F_r<C_j$ ó $F_r>C_j$

\textcolor{purple}{$\bullet$} \begin{align*}
    F_r>I_k &\iff F_r> F_r - (F_r-C_j)v^{n-k+1}\\
    &\iff (F_r-C_j)v^{n-k+1} >0\\
    &\iff F_r-C_j>0 \textit{ pues }v^{n-k+1}>0\\
    &\iff F_r>C_j\\
    &\iff \textit{ El bono se compra con premio}
\end{align*}

\textcolor{purple}{$\bullet$} \begin{align*}
    F_r<I_k &\iff F_r< F_r - (F_r-C_j)v^{n-k+1}\\
    &\iff (F_r-C_j)v^{n-k+1} <0\\
    &\iff F_r-C_j<0 \textit{ pues }v^{n-k+1}>0\\
    &\iff F_r<C_j\\
    &\iff \textit{ El bono se compra con descuento}
\end{align*}

\textit{Observaciones importantes}
\begin{enumerate}
    \item La aplicación $k\longrightarrow \ax{\angl{n-k}}$ es decreciente
    \item Recuérdese que $OB_0 = B_0 =P$ y $OB_n=B_n=C$
    \item \textit{Veamos cuánto vale}\begin{align*}
        OB_k& - OB_{k+1}\\
        &= [C+(F_r-C_j)\ax{\angl{n-k}}] - [C+(F_r-C_j)\ax{\angl{n-k+1}}]\\
        &= (F_r-C_j)[\ax{\angl{n-k}} - \ax{\angl{n-k+1}}]\\
        &= (F_r-C_j)[v+v^2+\dotsc+v^{n-k+1} + v^{n-k}\\
        &\quad -(v+v^2+\dotsc+v^{n-k+1}]\\
        &= (F_r-C_j)v^{n-k} = (F_r-C_j)(1+j)^{-(n-k)}
    \end{align*}
    \item \begin{align*}
        OB_k &> OB_{k+1} \iff OB_k-OB_{k+1} >0\\
        &\iff (F_r-C_j)v^{n-k} >0 \iff F_rC_j>0\\
        &\iff F_r>C_j \iff \textit{ El bono se compra \textcolor{red}{con premio}}
    \end{align*}\\
    Es decir, la aplicación $k\longrightarrow OB_k$ es decreciente $\iff$ El bono se compra con \textcolor{red}{con premio}\\
    \textcolor{purple}{$\bullet$} Es decir que si el bono se compra con premio, el valor en libros del bono va de $B_0=P$ ó $B_n=C$ decrecientemente ($i.e. P>C$)\\
    \textcolor{purple}{$\bullet$} Como $F_r>C_j$, entonces $F_r>I_k$; es decir que, a través del cupón el emisor del bono pago periódicamente más interés del que debería. Por lo tanto, al vencimiento, el emisor paga $C$ que es \textcolor{red}{menos} dinero que la deuda original $P$\\
    \textcolor{purple}{$\bullet$} Como $k\longrightarrow OB_k$ es decreciente, el \textit{book value} se va ajustando gradualmente hacia abajo. Este proceso se conoce como \textcolor{red}{\underline{amortización del premio}} ó \textcolor{red}{\underline{writing down}}\\
    \textcolor{purple}{$\bullet$} A $P_k$ se conoce como \textcolor{red}{\underline{cantidad de amortización del premio}}
    \item \begin{align*}
        OB_k &< OB_{k+1} \iff OB_k-OB_{k+1} <0\\
        &\iff (F_r-C_j)v^{n-k} <0 \iff F_rC_j<0\\
        &\iff F_r<C_j \iff \textit{ El bono se compra \textcolor{magenta}{con descuento}}
    \end{align*}\\
    Es decir, la aplicación $k\longrightarrow OB_k$ es creciente $\iff$ El bono se compra con \textcolor{magenta}{con descuento}\\
    \textcolor{purple}{$\bullet$} Es decir que si el bono se compra con descuento, el valor en libros del bono va de $B_0=P$ ó $B_n=C$ crecientemente ($i.e. P<C$)\\
    \textcolor{purple}{$\bullet$} Como $F_r<C_j$, entonces $F_r<I_k$; es decir que, a través del cupón el emisor del bono paga periódicamente menos interés del que debería. Por lo tanto, al vencimiento, el emisor paga $C$ que es \textcolor{magenta}{más} dinero que la deuda original $P$\\
    \textcolor{purple}{$\bullet$} Como $k\longrightarrow OB_k$ es creciente, el \textit{book value} se va ajustando gradualmente hacia arriba. Este proceso se conoce como \textcolor{magenta}{\underline{amortización del descuento}} ó \textcolor{magenta}{\underline{writing up}}\\
    \textcolor{purple}{$\bullet$} A $P_k$ se le conoce como \textcolor{magenta}{\underline{cantidad de amortización del descuento}}
\end{enumerate}
\textcolor{purple}{Moraleja: Es muy importante la reacción que hay entre $F_r$ y $C_j$}
$$\textcolor{purple}{F_r=C_j, F_r<C_j,F_r>C_j}$$

\end{document}
