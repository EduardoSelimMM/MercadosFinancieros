
\documentclass[
letterpaper,
11pt, % Cambiar a 10 si es que no cabe
oneside,
onecolumn, %twocolumn para dos columnas
article
]{memoir}

\usepackage[spanish,es-nodecimaldot]{babel}
\usepackage[utf8]{inputenc}
\usepackage[T1]{fontenc}
\usepackage{tgtermes} % La fuente a usar, si no compila quitar esta línea
\usepackage[svgnames]{xcolor} % Required for colour specification
\usepackage{blindtext} % Controls the indentation and the space between paragraphs
\usepackage{tikzsymbols} % Emojis
\usepackage{tikz} %Grphics 
\usepackage{empheq} % Hace la hoja tamaño carta
\usetikzlibrary{snakes,positioning, decorations.pathreplacing,decorations.markings,babel} % Diagrams
\usepackage{rotating} % Diagrams
\usepackage{pifont} % Figuras para referenciar
\usepackage{cancel} % To draw diagonal lines through expressions
\usepackage{tabularx} % Tables
\usepackage{multicol} % Multiple columns
\usepackage{enumitem} % Enumerate with diferent bullets
\usepackage{ulem} % Underline fixing code errors of normal \underline{•}
\usepackage{color,soul} % Underline with colors
\medievalpage

% Paquetes para matemáticas
\usepackage{amscd}
\usepackage{amsfonts}
\usepackage{amssymb}
\usepackage{amsmath}
\usepackage{amsthm}
\usepackage{latexsym}
\usepackage{mathrsfs}
\usepackage{bm}
\usepackage{bbm}
\usepackage{mathtools}
\usepackage{listings}
\usepackage[spanish,onelanguage,ruled,linesnumbered]{algorithm2e}
\usepackage{stackengine}
\usepackage[mathscr]{euscript}
\usepackage[scr]{rsfso}
\usepackage{empheq}
\usepackage[final]{microtype}
\usepackage{graphicx} % Para incluir figuras
\usepackage{lipsum}
\usepackage{actuarialsymbol} %Actuarial notation
\usepackage{hyperref}

% Command "alignedbox{}{}" for a box within an align environment
% Source: http://www.latex-community.org/forum/viewtopic.php?f=46&t=8144
\newlength\dlf  % Define a new measure, dlf
\newcommand\alignedbox[2]{
% Argument #1 = before & if there were no box (lhs)
% Argument #2 = after & if there were no box (rhs)
&  % Alignment sign of the line
{
\settowidth\dlf{$\displaystyle #1$}  
    % The width of \dlf is the width of the lhs, with a displaystyle font
\addtolength\dlf{\fboxsep+\fboxrule}  
    % Add to it the distance to the box, and the width of the line of the box
\hspace{-\dlf}  
    % Move everything dlf units to the left, so that & #1 #2 is aligned under #1 & #2
\boxed{#1 #2}
    % Put a box around lhs and rhs
}
}

\setlrmarginsandblock{0.15\paperwidth}{*}{1} % Para onecolumn
\setulmarginsandblock{0.5in}{1.5in}{1}  % Márgenes superior e inferior
\checkandfixthelayout

\parindent=0pt % disables indentation
\parskip=12pt % adds vertical space between paragraphs

\addto{\captionsspanish}{%
  \renewcommand{\bibname}{\Large Referencias}
}

\counterwithout{section}{chapter}
\counterwithout{figure}{chapter}

\makepagestyle{plain}
\makeevenfoot{plain}{\thepage}{}{}
\makeoddfoot{plain}{}{}{\thepage}
\makeevenhead{plain}{}{}{}
\makeoddhead{plain}{}{}{}

\makeatletter %
\makechapterstyle{standard}{
  \setlength{\beforechapskip}{2\baselineskip}
  \setlength{\midchapskip}{0\baselineskip}
  \setlength{\afterchapskip}{2\baselineskip}
  \renewcommand{\chapterheadstart}{\vspace*{\beforechapskip}}
  \renewcommand{\chapnamefont}{\normalfont\Large}
  \renewcommand{\printchaptername}{}
  \renewcommand{\chapternamenum}{\space}
  \renewcommand{\chapnumfont}{\normalfont\Large}
  %\renewcommand{\printchapternum}{\chapnumfont \thechapter.}
  %\renewcommand{\afterchapternum}{\par\nobreak\vskip \midchapskip}
  \renewcommand{\afterchapternum}{ }
  \renewcommand{\printchapternonum}{\vspace*{\midchapskip}\vspace*{5mm}}
  \renewcommand{\chaptitlefont}{\bfseries\LARGE}
  \renewcommand{\printchaptertitle}[1]{\chaptitlefont ##1}
  \renewcommand{\afterchaptertitle}{\par\nobreak\vskip \afterchapskip}
}
\makeatother

\chapterstyle{standard}

\makeatletter %
\makechapterstyle{appendix}{
  \setlength{\beforechapskip}{2\baselineskip}
  \setlength{\midchapskip}{0\baselineskip}
  \setlength{\afterchapskip}{2\baselineskip}
  \renewcommand{\chapterheadstart}{\vspace*{\beforechapskip}}
  \renewcommand{\chapnamefont}{\normalfont\Large}
  \renewcommand{\printchaptername}{\chapnamefont \@chapapp}
  \renewcommand{\chapternamenum}{\space}
  \renewcommand{\chapnumfont}{\normalfont\Large}
  \renewcommand{\printchapternum}{\chapnumfont \thechapter.}
  %\renewcommand{\afterchapternum}{\par\nobreak\vskip \midchapskip}
  \renewcommand{\afterchapternum}{ }
  \renewcommand{\printchapternonum}{\vspace*{\midchapskip}\vspace*{5mm}}
  \renewcommand{\chaptitlefont}{\bfseries\LARGE}
  \renewcommand{\printchaptertitle}[1]{\chaptitlefont ##1}
  \renewcommand{\afterchaptertitle}{\par\nobreak\vskip \afterchapskip}
}
\makeatother

\setlength{\columnseprule}{1pt} %Line between paragraphs

\tikzset{
  % style to apply some styles to each segment of a path
  on each segment/.style={
    decorate,
    decoration={
      show path construction,
      moveto code={},
      lineto code={
        \path [#1]
        (\tikzinputsegmentfirst) -- (\tikzinputsegmentlast);
      },
      curveto code={
        \path [#1] (\tikzinputsegmentfirst)
        .. controls
        (\tikzinputsegmentsupporta) and (\tikzinputsegmentsupportb)
        ..
        (\tikzinputsegmentlast);
      },
      closepath code={
        \path [#1]
        (\tikzinputsegmentfirst) -- (\tikzinputsegmentlast);
      },
    },
  },
  % style to add an arrow in the middle of a path
  end arrow/.style={postaction={decorate,decoration={
        markings,
        mark=at position 0.999 with {\arrow[#1]{stealth}}
      }}},
} % Curved lines

% Declaración de comandos y operadores
\newcommand\RR{\mathbb R}
\newcommand\NN{\mathbb N}
\newcommand\PP{\mathbb P}
\newcommand\dpartial[1]{\frac{\partial}{\partial #1}}
\newcommand\deriv[1]{\frac{d}{d #1}}
\newcommand\integral[4]{\int_{#1}^{#2} #3 \, d#4}
\newcommand*\circled[1]{\tikz[baseline=(char.base)]{
            \node[shape=circle,draw,inner sep=2pt] (char) {#1};}}
\DeclareMathOperator\Ber{Bernoulli}

% Se definen los comandos para escribir teoremas, definiciones y demás.
\theoremstyle{plain}
\newtheorem*{theorem}{Teorema}
\newtheorem*{corollary}{Corolario}
\newtheorem*{lemma}{Lema}
\newtheorem*{proposition}{Proposici\'on}
\theoremstyle{definition}
\newtheorem*{definition}{Definici\'on}
\theoremstyle{remark}
\newtheorem*{remark}{Observaci\'on}

\begin{document}

%%%%%%%%%%%%%%%%%%%%%%%%%
% Aquí va la portada
%%%%%%%%%%%%%%%%%%%%%%%%%

\begin{titlingpage} % Portada

    \raggedleft % Alineada a la derecha
    %\raggedright % Alineada a la izquierda
	
	\vspace*{\baselineskip} % Whitespace at the top of the page
	
	\vspace*{0.25\textheight} % Whitespace before the title
	
	%------------------------------------------------
	%	Cosas del título
	%------------------------------------------------
    
    \vspace*{0.1\textheight}

    {\Huge{\textbf{Bonos}}}\\[\baselineskip] % Aquí va el título
    \vspace*{0.1\textheight}

    %------------------------------------------------
	%	Aquí van los nombres
	%------------------------------------------------
    
    {\Large Eduardo Selim Matínez Mayorga}\\[\baselineskip]
	
	\vfill

\end{titlingpage}

\thispagestyle{empty}

\chapter*{Bonos}
Hasta el momento hemos supuesto que cuando compramos/vendemos un bono, éste está recién emitido \textit{i.e.} es un bono nuevo. $\longrightarrow$ Esto se conoce como transacción en el \underline{mercado primario}.\\
¿Qué pasa si yo ya tengo un bono y me quiero deshacer de él?\\
Voy al mercado secundario\

\textcolor{purple}{$\bullet$} En el mercado secundario se compra/venden instrumentos financieros que ya existen con autoridad.\\
$\longrightarrow$ La pregunta que nos hacemos ahora es: ¿En cuánto \$ compraríamos/venderíamos un bono que ya fue emitido con anterioridad?\\
\textcolor{magenta}{Depende}
\textcolor{magenta}{\begin{enumerate}
    \item Si compra/vende justo después de pago cupón
    $$Precio=VP(\textit{Flujo faltante})$$
    \item Si compra/vende entre fechas de pago de cupón (es un poquito más complicado)\\
    El "problema" es: ¿cómo divido el último cupón que ya ocurrió?
\end{enumerate}}
\textcolor{purple}{$\bullet$} $OB_k$: El valor del bono justo después del \textit{k-ésimo} cupón\
$$OB_k=C+(F_r - C_j)\ax{\angl{n-k}} + C(1+j)^{-(n-k)}$$
\textcolor{purple}{$\bullet$} Sin embargo, $OB_k$ sólo está definido para $k\in\{0,1,...,n\}$ donde $OB_0=P$ y $OB_n=C$\

\textcolor{purple}{$\bullet$} Nos gustaría poder dar el valor del bono en cualquier $t\in[0,n]$\

\textcolor{purple}{$\bullet$} Hay 3 propuestas para hacer esto:
\begin{enumerate}
    \item Método teórico
    \item Método práctico
    \item Método semi-teórico
\end{enumerate}

\begin{center}
    

\tikzset{every picture/.style={line width=0.75pt}} %set default line width to 0.75pt        

\begin{tikzpicture}[x=0.75pt,y=0.75pt,yscale=-1,xscale=1]
%uncomment if require: \path (0,300); %set diagram left start at 0, and has height of 300

%Straight Lines [id:da34657222641698626] 
\draw    (141,191.9) -- (505,191.9) ;
%Straight Lines [id:da8320031647356826] 
\draw    (190.83,187.23) -- (190.83,197.4) ;
%Straight Lines [id:da908003181732433] 
\draw    (456.33,187.07) -- (456.33,197.23) ;
%Straight Lines [id:da7312982626415365] 
\draw    (149.83,187.23) -- (149.83,197.4) ;
%Straight Lines [id:da5763637162450717] 
\draw    (230.83,187.23) -- (230.83,197.4) ;
%Straight Lines [id:da30794614098512674] 
\draw    (309.83,187.23) -- (309.83,197.4) ;
%Straight Lines [id:da37656947767695326] 
\draw    (349.83,187.73) -- (349.83,197.9) ;
%Straight Lines [id:da6195061540260131] 
\draw    (390.33,187.23) -- (390.33,197.4) ;
%Straight Lines [id:da5214754470951718] 
\draw    (500.83,187.07) -- (500.83,197.23) ;
%Straight Lines [id:da47740699399543474] 
\draw [color={rgb, 255:red, 189; green, 16; blue, 224 }  ,draw opacity=1 ]   (370,176.07) -- (370,228.07) ;
\draw [shift={(370,230.07)}, rotate = 270] [color={rgb, 255:red, 189; green, 16; blue, 224 }  ,draw opacity=1 ][line width=0.75]    (10.93,-3.29) .. controls (6.95,-1.4) and (3.31,-0.3) .. (0,0) .. controls (3.31,0.3) and (6.95,1.4) .. (10.93,3.29)   ;

% Text Node
\draw (255.6,197.5) node [anchor=north west][inner sep=0.75pt]   [align=left] {. . .};
% Text Node
\draw (144.9,201.7) node [anchor=north west][inner sep=0.75pt]  [font=\small]  {$0$};
% Text Node
\draw (185.4,201.7) node [anchor=north west][inner sep=0.75pt]  [font=\small]  {$1$};
% Text Node
\draw (226.4,201.7) node [anchor=north west][inner sep=0.75pt]  [font=\small]  {$2$};
% Text Node
\draw (293.8,202.3) node [anchor=north west][inner sep=0.75pt]  [font=\small]  {$k-1$};
% Text Node
\draw (439.8,201.8) node [anchor=north west][inner sep=0.75pt]  [font=\small]  {$n-1$};
% Text Node
\draw (345.8,202.8) node [anchor=north west][inner sep=0.75pt]  [font=\small]  {$k$};
% Text Node
\draw (374.3,202.8) node [anchor=north west][inner sep=0.75pt]  [font=\small]  {$k+1$};
% Text Node
\draw (494.8,200.3) node [anchor=north west][inner sep=0.75pt]  [font=\small]  {$n$};
% Text Node
\draw (411.1,199) node [anchor=north west][inner sep=0.75pt]   [align=left] {. . .};
% Text Node
\draw (479.9,172.7) node [anchor=north west][inner sep=0.75pt]  [font=\small,color={rgb, 255:red, 245; green, 166; blue, 35 }  ,opacity=1 ]  {$F_{r} +C$};
% Text Node
\draw (117.9,220.7) node [anchor=north west][inner sep=0.75pt]  [font=\small,color={rgb, 255:red, 189; green, 16; blue, 224 }  ,opacity=1 ]  {$P=OB_{0}$};
% Text Node
\draw (449.9,172.7) node [anchor=north west][inner sep=0.75pt]  [font=\small,color={rgb, 255:red, 245; green, 166; blue, 35 }  ,opacity=1 ]  {$F_{r}$};
% Text Node
\draw (225.3,172.9) node [anchor=north west][inner sep=0.75pt]  [font=\small,color={rgb, 255:red, 245; green, 166; blue, 35 }  ,opacity=1 ]  {$F_{r}$};
% Text Node
\draw (344.3,172.7) node [anchor=north west][inner sep=0.75pt]  [font=\small,color={rgb, 255:red, 245; green, 166; blue, 35 }  ,opacity=1 ]  {$F_{r}$};
% Text Node
\draw (304.1,172.9) node [anchor=north west][inner sep=0.75pt]  [font=\small,color={rgb, 255:red, 245; green, 166; blue, 35 }  ,opacity=1 ]  {$F_{r}$};
% Text Node
\draw (185.9,171.9) node [anchor=north west][inner sep=0.75pt]  [font=\small,color={rgb, 255:red, 245; green, 166; blue, 35 }  ,opacity=1 ]  {$F_{r}$};
% Text Node
\draw (145.1,171.3) node [anchor=north west][inner sep=0.75pt]  [font=\small,color={rgb, 255:red, 245; green, 166; blue, 35 }  ,opacity=1 ]  {$F_{r}$};
% Text Node
\draw (384.3,173.3) node [anchor=north west][inner sep=0.75pt]  [font=\small,color={rgb, 255:red, 245; green, 166; blue, 35 }  ,opacity=1 ]  {$F_{r}$};
% Text Node
\draw (180.1,220.3) node [anchor=north west][inner sep=0.75pt]  [font=\small,color={rgb, 255:red, 189; green, 16; blue, 224 }  ,opacity=1 ]  {$OB_{1}$};
% Text Node
\draw (219.1,221.3) node [anchor=north west][inner sep=0.75pt]  [font=\small,color={rgb, 255:red, 189; green, 16; blue, 224 }  ,opacity=1 ]  {$OB_{2}$};
% Text Node
\draw (367,228.47) node [anchor=north west][inner sep=0.75pt]  [color={rgb, 255:red, 189; green, 16; blue, 224 }  ,opacity=1 ]  {$t$};
% Text Node
\draw (283,243.07) node [anchor=north west][inner sep=0.75pt]  [font=\footnotesize,color={rgb, 255:red, 189; green, 16; blue, 224 }  ,opacity=1 ] [align=left] {¿cuánto vale el bono entre $\displaystyle k$ y $\displaystyle k+1$?};


\end{tikzpicture}

\end{center}

\begin{multicols}{2}
    \centering
    \textcolor{orange}{Método teórico}
    $$V_t := OB_{\llcorner t\lrcorner}(1+j)^{t-\llcorner t\lrcorner}$$
   
    \columnbreak
    
    \textcolor{orange}{Método práctico}
    $$\Tilde{V}_t := OB_{\llcorner t\lrcorner}\big[1+j(1+j(t-\llcorner t\lrcorner)\big]$$
\end{multicols}

\textcolor{blue}{¿De dónde surgen estas definiciones?}\\
Calculo $OB_{\llcorner t\lrcorner} = OB_k$ (ya lo sé hacer) y luego acumulo esta cantidad del tiempo $k=\llcorner t\lrcorner$ a $t$
\begin{center}
    

\tikzset{every picture/.style={line width=0.75pt}} %set default line width to 0.75pt        

\begin{tikzpicture}[x=0.75pt,y=0.75pt,yscale=-1,xscale=1]
%uncomment if require: \path (0,300); %set diagram left start at 0, and has height of 300

%Straight Lines [id:da9890289656490168] 
\draw    (209,150) -- (427,150) ;
%Straight Lines [id:da7215314632081371] 
\draw    (320,145.4) -- (320,155.47) ;
%Straight Lines [id:da16812785542723963] 
\draw    (250,145.4) -- (250,155.47) ;
%Straight Lines [id:da6089489098469324] 
\draw    (380,146) -- (380,156.07) ;
%Curve Lines [id:da1699325406353832] 
\draw    (250,140) .. controls (262.84,125.28) and (301.66,125.86) .. (319.35,139.34) ;
\draw [shift={(320.67,140.4)}, rotate = 220.46] [color={rgb, 255:red, 0; green, 0; blue, 0 }  ][line width=0.75]    (10.93,-3.29) .. controls (6.95,-1.4) and (3.31,-0.3) .. (0,0) .. controls (3.31,0.3) and (6.95,1.4) .. (10.93,3.29)   ;
%Straight Lines [id:da6984120425153036] 
\draw    (250.83,170.23) -- (234.08,186.99) ;
\draw [shift={(232.67,188.4)}, rotate = 315] [color={rgb, 255:red, 0; green, 0; blue, 0 }  ][line width=0.75]    (10.93,-3.29) .. controls (6.95,-1.4) and (3.31,-0.3) .. (0,0) .. controls (3.31,0.3) and (6.95,1.4) .. (10.93,3.29)   ;
%Straight Lines [id:da4239479739707316] 
\draw    (381.83,170.23) -- (398.75,187.15) ;
\draw [shift={(400.17,188.57)}, rotate = 225] [color={rgb, 255:red, 0; green, 0; blue, 0 }  ][line width=0.75]    (10.93,-3.29) .. controls (6.95,-1.4) and (3.31,-0.3) .. (0,0) .. controls (3.31,0.3) and (6.95,1.4) .. (10.93,3.29)   ;

% Text Node
\draw (234.83,157.4) node [anchor=north west][inner sep=0.75pt]  [font=\footnotesize]  {$k=\llcorner t\lrcorner $};
% Text Node
\draw (317.83,157.4) node [anchor=north west][inner sep=0.75pt]  [font=\footnotesize]  {$t$};
% Text Node
\draw (366.83,157.4) node [anchor=north west][inner sep=0.75pt]  [font=\footnotesize]  {$k+1$};
% Text Node
\draw (232.5,203.93) node  [font=\scriptsize] [align=left] {\begin{minipage}[lt]{37.63pt}\setlength\topsep{0pt}
\begin{center}
con interés\\compuesto
\end{center}

\end{minipage}};
% Text Node
\draw (426.17,203.93) node [anchor=east] [inner sep=0.75pt]  [font=\scriptsize] [align=left] {\begin{minipage}[lt]{37.63pt}\setlength\topsep{0pt}
\begin{center}
con interés\\simple
\end{center}

\end{minipage}};


\end{tikzpicture}

\end{center}
\textcolor{orange}{Lema:} Para $k\in\{0,1,...,n,n-1\}$
$$OB_k(1+j) = OB_{k+1} + F_r$$

\underline{\textit{Demostración}}
\begin{align*}
    OB_k(1+j) &= \big[F_r\cdot \ax{\angl{n-k}} + Cv^{n-k}\big](1+j) + C(1+j)^{-(n-k)}(1+j)\\
    &= F_r\big(v+v^2+\dotsc+v^{n-k-1}+v^{n-k}\big)(1+j) + C((1+j)^{-(n-k)}(1+j)\\
    &= F_r\big(\underbrace{1+v+\dotsc+v^{n-k-1}+v^{n-k}}_{n-k\textit{ sumandos}}\big)(1+j) + C(1+j)^{-(n-k-1)}\\
    &= F_r + F_r(v+\dotsc+v^{n-k}+v^{n-k-1}) + C(1+j)^{-(n-k-1)}\\
    &= F_r + F_r\cdot\ax{\angl{n-k-1}} + Cv^{n-k-1}\\
    &=F_r + OB_{k+1}
\end{align*}
\textcolor{orange}{Observaciones importantes}
\begin{multicols}{2}
    \textcolor{orange}{$\longrightarrow$ Método teórico}
    \begin{align*}
        \lim_{x\to1^-}&\Big(V_{\llcorner t\lrcorner+x} - V_{\llcorner t\lrcorner+1}\Big)\\
        &=\lim_{x\to1^-}\Big(V_{\llcorner t\lrcorner+x} - OB_{\llcorner t\lrcorner+1}\Big)\\
        &\textit{ pues }\llcorner t\lrcorner\textit{ es entero}\\
        &=\lim_{x\to1^-}\Big(OB_{\llcorner t\lrcorner}(1+j)^x - OB_{\llcorner t\lrcorner+1}\Big)\\
        &=\lim_{x\to1^-}\Big(OB_{\llcorner t\lrcorner}(1+j)^x \\
        &- \big(OB_{\llcorner t\lrcorner}(1+j)-F_r\big)\Big)\\
        &= OB_{\llcorner t\lrcorner}(1+j) - OB_{\llcorner t\lrcorner}(1+j) + F_r\\
        &= F_r\\
        \therefore& \lim_{x\to1^-} V_{\llcorner t\lrcorner+x} = V_{\llcorner t\lrcorner+1}+F_r
    \end{align*}
    
    \columnbreak
    
    \textcolor{orange}{$\longrightarrow$ Método práctico}
    \begin{align*}
        \lim_{x\to1^-}&\Big(\Tilde{V}_{\llcorner t\lrcorner+x} - \Tilde{V}_{\llcorner t\lrcorner+1}\Big)\\
        &=\lim_{x\to1^-}\Big(\Tilde{V}_{\llcorner t\lrcorner+x} - OB_{\llcorner t\lrcorner+1}\Big)\\
        &\textit{ pues }\llcorner t\lrcorner\textit{ es entero}\\
        &=\lim_{x\to1^-}\Big(OB_{\llcorner t\lrcorner}(1+jx) - OB_{\llcorner t\lrcorner+1}\Big)\\
        &=\lim_{x\to1^-}\Big(OB_{\llcorner t\lrcorner}(1+jx) \\
        &- OB_{\llcorner t\lrcorner}(1+j)+F_r\Big)\\
        &= OB_{\llcorner t\lrcorner}(1+j) - OB_{\llcorner t\lrcorner}(1+j) + F_r\\
        &= F_r\\
        \therefore& \lim_{x\to1^-} \Tilde{V}_{t+x} = \Tilde{V}_{\llcorner t\lrcorner+1}+F_r
    \end{align*}
\end{multicols}

Es decir, tanto función $t\to V_t$ como la función $t\to\Tilde{V}_j$ son discontinues en $1,2,...,n-1$.\

Para ''arreglar'' esta discontinuidad de $t\to\Tilde{V}_j$ se puede definir una función $\gamma$ que cumpla:
\begin{itemize}
    \item[(i)] $\gamma$ es continua
    \item[(ii)] \begin{align*}
        \gamma(0)&=OB_0\\
        \gamma(1)&=OB_1\\
        &\vdots\\
        \gamma(n)&=OB_n
    \end{align*}
    \item[(iii)] Sea un segmento de recta en los intervalos $(k,k+1)$.
\end{itemize}
¿Cómo obtenemos la regla de correspondencia de $\gamma$?
\begin{center}
    

\tikzset{every picture/.style={line width=0.75pt}} %set default line width to 0.75pt        

\begin{tikzpicture}[x=0.75pt,y=0.75pt,yscale=-1,xscale=1]
%uncomment if require: \path (0,300); %set diagram left start at 0, and has height of 300

%Straight Lines [id:da11483983820175225] 
\draw    (350.17,195.4) -- (350.17,205.4) ;
%Straight Lines [id:da30526072049388187] 
\draw    (330.17,195.4) -- (330.17,205.4) ;
%Straight Lines [id:da1304058758067589] 
\draw    (290.17,195.4) -- (290.17,205.4) ;
%Shape: Axis 2D [id:dp4802451250070878] 
\draw  (237.33,200.38) -- (382,200.38)(250.43,93.07) -- (250.43,212.07) (375,195.38) -- (382,200.38) -- (375,205.38) (245.43,100.07) -- (250.43,93.07) -- (255.43,100.07)  ;
%Straight Lines [id:da3823753294892204] 
\draw    (245.65,109.9) -- (255.65,109.9) ;
%Straight Lines [id:da764091905390827] 
\draw    (245.66,129.9) -- (255.66,129.9) ;
%Straight Lines [id:da14197494646053066] 
\draw    (245.69,169.9) -- (255.69,169.9) ;
%Straight Lines [id:da17668147611498453] 
\draw [color={rgb, 255:red, 189; green, 16; blue, 224 }  ,draw opacity=1 ] [dash pattern={on 4.5pt off 4.5pt}]  (250.17,109.73) -- (350.5,109.73) ;
%Straight Lines [id:da5388869901927572] 
\draw [color={rgb, 255:red, 189; green, 16; blue, 224 }  ,draw opacity=1 ] [dash pattern={on 4.5pt off 4.5pt}]  (250.17,170.23) -- (290,170.23) ;
%Straight Lines [id:da29184588010983425] 
\draw [color={rgb, 255:red, 189; green, 16; blue, 224 }  ,draw opacity=1 ] [dash pattern={on 4.5pt off 4.5pt}]  (350.5,109.9) -- (350.5,200.4) ;
%Straight Lines [id:da7569647002557991] 
\draw [color={rgb, 255:red, 245; green, 166; blue, 35 }  ,draw opacity=1 ] [dash pattern={on 4.5pt off 4.5pt}]  (250.17,130.23) -- (330.5,130.23) ;
%Straight Lines [id:da9414423247678357] 
\draw [color={rgb, 255:red, 245; green, 166; blue, 35 }  ,draw opacity=1 ] [dash pattern={on 4.5pt off 4.5pt}]  (330.5,129.9) -- (330.5,200.4) ;
%Straight Lines [id:da8872388789932882] 
\draw [color={rgb, 255:red, 251; green, 165; blue, 165 }  ,draw opacity=1 ][line width=2.25]    (290,170.23) -- (350.5,109.9) ;

% Text Node
\draw (157.17,106.63) node [anchor=north west][inner sep=0.75pt]  [font=\scriptsize]  {$\gamma ( k+1) =OB_{k+1}$};
% Text Node
\draw (218.17,126.63) node [anchor=north west][inner sep=0.75pt]  [font=\scriptsize]  {$\gamma ( t)$};
% Text Node
\draw (183.17,165.63) node [anchor=north west][inner sep=0.75pt]  [font=\scriptsize]  {$\gamma ( k) =OB_{k}$};
% Text Node
\draw (287.17,208.63) node [anchor=north west][inner sep=0.75pt]  [font=\scriptsize]  {$k$};
% Text Node
\draw (326,208.4) node [anchor=north west][inner sep=0.75pt]  [font=\scriptsize]  {$t$};
% Text Node
\draw (341.17,208.63) node [anchor=north west][inner sep=0.75pt]  [font=\scriptsize]  {$k+1$};


\end{tikzpicture}

\end{center}
La pendiente del segmento de recta es 
$$\frac{\gamma(k+1)-\gamma(k)}{k+1-k} = \gamma(k+1) - \gamma(k)$$

Usando la ecuación de la recta punto-pendiente
\begin{align*}
    &\gamma(t) - \gamma(k) = \big[\gamma(k+1)-\gamma(k)\big](t-k)\\
    \Rightarrow\quad &\gamma(t) -\gamma(\llcorner t\lrcorner) = \Big[\gamma(\llcorner t\lrcorner+1) -\gamma(\llcorner t\lrcorner)\Big](t-\llcorner t\lrcorner)\\
    \Rightarrow\quad &\gamma(t) = \gamma(\llcorner t\lrcorner) + \Big[\gamma(\llcorner t\lrcorner+1) -\gamma(\llcorner t\lrcorner)\Big](t-\llcorner t\lrcorner)\\
    \Rightarrow\quad &\gamma(t) = OB_{\llcorner t\lrcorner} + \Big[OB_{\llcorner t\lrcorner+1} - OB_{\llcorner t\lrcorner}\Big](t-\llcorner t\lrcorner)...\circledast\\
    &\text{donde }\gamma(\cdot) \text{ cumple las 3 condiciones que se pidieron.}
\end{align*}
La ecuación $\circledast$ motiva la siguiente definición.

\begin{definition}
Se define el valor práctico limpio como
$$\Tilde{V_t}^{[C]} := OB_{\llcorner t\lrcorner} + \Big[OB_{\llcorner t\lrcorner+1}-OB_{\llcorner t\lrcorner}\Big](t-\llcorner t\lrcorner)$$
\end{definition}

\textit{Observación:}
\begin{align*}
    \Tilde{V_t} - \Tilde{V_t}^{[C]} &= OB_{\llcorner t\lrcorner}\big[1+j(t-\llcorner t\lrcorner)\big] - \big[OB_{\llcorner t\lrcorner} + [OB_{\llcorner t\lrcorner+1} - OB_{\llcorner t\lrcorner}](t-\llcorner t\lrcorner)\big] \\
    &= OB_{\llcorner t\lrcorner}\cdot j(t-\llcorner t\lrcorner) - \big[OB_{\llcorner t\lrcorner+1} - OB_{\llcorner t\lrcorner}\big](t-\llcorner t\lrcorner)\\
    &= (t-\llcorner t\lrcorner)\big[OB_{\llcorner t\lrcorner}\cdot j - OB_{\llcorner t\lrcorner+1}+ OB_{\llcorner t\lrcorner}\big]\\
    &= (t-\llcorner t\lrcorner)\big[OB_{\llcorner t\lrcorner}(1+j) - OB_{\llcorner t\lrcorner+1}\big]\\
    &=(t-\llcorner t\lrcorner)\cdot F_r\\
    &\quad\therefore\Tilde{V_t} - \Tilde{V_t}^{[C]} = F_r(t-\llcorner t\lrcorner)\\
    &\text{donde }(t-\llcorner t\lrcorner)\text{ lapso a tiempo entre el último cupón y la fecha de valuación}
\end{align*}
Entonces
$$\Tilde{V_t} - \Tilde{V_t}^{[C]} = \underbrace{F_r(t-\llcorner t\lrcorner)}_{\text{parte proporcional del cupón}}$$
Falta definir el valor técnico limpio
\begin{definition}
Se define el valor teórico limpio como
$$V_t^{[C]} := \frac{\delta}{j}F_r\Bar{a}_{\angl{n-t}j} + C(1+j)^{-(n-t)} \text{ con }\delta\sim j$$
\end{definition}

¿Qué motiva esta definición?\\
Nótese que
$$\frac{\delta}{j}F_r\cdot\Bar{a}_{\angl{n-t}j} + \frac{\delta}{j}F_r\frac{1-v^{n-t}}{\delta} = F_r\frac{1-v^{n-t}}{j}$$
En la última expresión, se tiene la tentación de escribir $F_r\ax{\angl{n-t}}$ pero no tiene sentido pues $n-t\notin \mathbb{N}_+$y el símbolo $\ax{\angl{k}}$ sólo está definido para $k\in\mathbb{N}_+$.\\
Sin embargo, el símbolo $\Bar{a}_{\angl{t}}$ sí está bien definido para $t\in\mathbb{R}$ pues
$$\Bar{a}_{\angl{t}} = \int_0^t v^udu = \frac{1-v^t}{\delta}$$

\begin{proposition}
$$V_t^{[C]} = V_t F_r\Big[\frac{(1+j)^{t-\llcorner t\lrcorner}-1}{j}\Big]$$
\end{proposition}
\underline{\textit{Demostración}}
\begin{align*}
    V_t^{[C]} &= \frac{\delta}{j}F_r\cdot\Bar{a}_{\angl{n-t}} + C(1+j)^{-(n-t)}\\
    &= \frac{\delta}{j}F_r\Bar{a}_{\angl{n-t}} + Cv^{n-t}\\
    &= v^{\llcorner t\lrcorner-t}\Big[v^{t-\llcorner t\lrcorner}\frac{\delta}{j}F_r\Bar{a}_{\angl{n-t}} + v^{t-\llcorner t\lrcorner} Cv^{n-t}\Big]\\
    &= v^{\llcorner t\lrcorner-t}\Big[v^{t-\llcorner t\lrcorner}\frac{\delta}{j}F_r\Bar{a}_{\angl{n-t}} + Cv^{n-\llcorner t\lrcorner}\Big]\\
    &= v^{\llcorner t\lrcorner-t}\Big[v^{t-\llcorner t\lrcorner}\frac{\delta}{j}F_r\Big(\frac{1-v^{n-t}}{\delta} + Cv^{n-\llcorner t\lrcorner}\Big]\\
    &= v^{\llcorner t\lrcorner-t}\Big[F_r\big(\frac{v^{t-\llcorner t\lrcorner}-(v^{n-\llcorner t\lrcorner}}{j}\big) + Cv^{n-\llcorner t\lrcorner}\Big]\\
    &= v^{\llcorner t\lrcorner-t}\Big[F_r\big(\frac{v^{t-\llcorner t\lrcorner}-1+1-(v^{n-\llcorner t\lrcorner}}{j}\big) + Cv^{n-\llcorner t\lrcorner}\Big]\\
    &= v^{\llcorner t\lrcorner-t}\Big[F_r\big(\frac{v^{t-\llcorner t\lrcorner}-1}{j}\big) + F_r\big(\frac{1-v^{n-\llcorner t\lrcorner}}{j}\big) + Cv^{n-\llcorner t\lrcorner}\Big]\\
    &= v^{\llcorner t\lrcorner-t}\Big[F_r\big(\frac{v^{t-\llcorner t\lrcorner}-1}{j}\big) + F_r\ax{\angl{n-\llcorner t\lrcorner}} + Cv^{n-\llcorner t\lrcorner}\Big]\\
    &= v^{\llcorner t\lrcorner-t}\Big[F_r\big(\frac{v^{t-\llcorner t\lrcorner}-1}{j}\big) + OB_{\llcorner t\lrcorner}\Big]\\
    &= v^{\llcorner t\lrcorner-t}\cdot F_r\big(\frac{v^{t-\llcorner t\lrcorner}-1}{j}\big) + v^{\llcorner t\lrcorner-t}OB_{\llcorner t\lrcorner}\Big]\\
    &= F_r\bigg[\frac{1-v^{\llcorner t\lrcorner-t}}{j}\bigg] + v^{\llcorner t\lrcorner-t}OB_{\llcorner t\lrcorner}\\
    &= -F_r\bigg[\frac{(1+j)^{t-\llcorner t\lrcorner}-1}{j}\bigg] + v^{\llcorner t\lrcorner-t}OB_{\llcorner t\lrcorner}\\
    &= -F_r\bigg[\frac{(1+j)^{t-\llcorner t\lrcorner}-1}{j}\bigg] + (1+j)^{t-\llcorner t\lrcorner}OB_{\llcorner t\lrcorner}\\
    &= -F_r\bigg[\frac{(1+j)^{t-\llcorner t\lrcorner}-1}{j}\bigg] + V_t
\end{align*}
En resumen,
\begin{align*}
    \Tilde{V_t} - \Tilde{V_t}^{[C]} &= F_r(t- \llcorner t\lrcorner)>0\\
    V_t - V_t^{[C]} &= F_r\Big[\frac{(1+j)^{t- \llcorner t\lrcorner}-1}{j}\Big]>0
\end{align*}

\begin{center}
\begin{table}[h]
\centering
\begin{tabular}{cc|c|c}
 &
  \begin{tabular}[c]{@{}c@{}}Flat Value\\ (Sucio)\end{tabular} &
  \begin{tabular}[c]{@{}c@{}}Market Value\\ (Limpio)\end{tabular} &
  \begin{tabular}[c]{@{}c@{}}Accured\\ coupon\end{tabular} \\ \hline
Teórico      & $OB_{\llcorner t\lrcorner}(1+j)^{t-\llcorner t\lrcorner} = V_t$        & $V_t^{[C]}$         & $F_r\Big[\frac{(1+j)^{t-\llcorner t\lrcorner}-1}{j}\Big]$ \\ \hline
Práctico     & $OB_{\llcorner t\lrcorner}[1+j(t-\llcorner t\lrcorner)] = \Tilde{V_t}$ & $\Tilde{V_t}^{[C]}$ & $F_r[t-\llcorner t\lrcorner]$                             \\ \hline
Semi-teórico & $OB_{\llcorner t\lrcorner}(1+j)^{t-\llcorner t\lrcorner}=V_t$          & $\Tilde{V_t}^{[C]}$ & $F_r[t-\llcorner t\lrcorner]$                            
\end{tabular}
\end{table}
\end{center}

\end{document}
