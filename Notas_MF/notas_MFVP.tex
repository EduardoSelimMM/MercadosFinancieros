
\documentclass[
letterpaper,
11pt, % Cambiar a 10 si es que no cabe
oneside,
onecolumn, %twocolumn para dos columnas
article
]{memoir}

\usepackage[spanish,es-nodecimaldot]{babel}
\usepackage[utf8]{inputenc}
\usepackage[T1]{fontenc}
\usepackage{tgtermes} % La fuente a usar, si no compila quitar esta línea
\usepackage[svgnames]{xcolor} % Required for colour specification
\usepackage{blindtext} % Controls the indentation and the space between paragraphs
\usepackage{tikzsymbols} % Emojis
\usepackage{tikz} %Grphics 
\usepackage{empheq} % Hace la hoja tamaño carta
\usetikzlibrary{snakes,positioning, decorations.pathreplacing,decorations.markings,babel} % Diagrams
\usepackage{rotating} % Diagrams
\usepackage{pifont} % Figuras para referenciar
\usepackage{cancel} % To draw diagonal lines through expressions
\usepackage{tabularx} % Tables
\usepackage{multicol} % Multiple columns
\usepackage{enumitem} % Enumerate with diferent bullets
\usepackage{ulem} % Underline fixing code errors of normal \underline{•}
\usepackage{color,soul} % Underline with colors
\medievalpage

% Paquetes para matemáticas
\usepackage{amscd}
\usepackage{amsfonts}
\usepackage{amssymb}
\usepackage{amsmath}
\usepackage{amsthm}
\usepackage{latexsym}
\usepackage{mathrsfs}
\usepackage{bm}
\usepackage{bbm}
\usepackage{mathtools}
\usepackage{listings}
\usepackage[spanish,onelanguage,ruled,linesnumbered]{algorithm2e}
\usepackage{stackengine}
\usepackage[mathscr]{euscript}
\usepackage[scr]{rsfso}
\usepackage{empheq}
\usepackage[final]{microtype}
\usepackage{graphicx} % Para incluir figuras
\usepackage{lipsum}
\usepackage{actuarialsymbol} %Actuarial notation
\usepackage{hyperref}

% Command "alignedbox{}{}" for a box within an align environment
% Source: http://www.latex-community.org/forum/viewtopic.php?f=46&t=8144
\newlength\dlf  % Define a new measure, dlf
\newcommand\alignedbox[2]{
% Argument #1 = before & if there were no box (lhs)
% Argument #2 = after & if there were no box (rhs)
&  % Alignment sign of the line
{
\settowidth\dlf{$\displaystyle #1$}  
    % The width of \dlf is the width of the lhs, with a displaystyle font
\addtolength\dlf{\fboxsep+\fboxrule}  
    % Add to it the distance to the box, and the width of the line of the box
\hspace{-\dlf}  
    % Move everything dlf units to the left, so that & #1 #2 is aligned under #1 & #2
\boxed{#1 #2}
    % Put a box around lhs and rhs
}
}

\setlrmarginsandblock{0.15\paperwidth}{*}{1} % Para onecolumn
\setulmarginsandblock{0.5in}{1.5in}{1}  % Márgenes superior e inferior
\checkandfixthelayout

\parindent=0pt % disables indentation
\parskip=12pt % adds vertical space between paragraphs

\addto{\captionsspanish}{%
  \renewcommand{\bibname}{\Large Referencias}
}

\counterwithout{section}{chapter}
\counterwithout{figure}{chapter}

\makepagestyle{plain}
\makeevenfoot{plain}{\thepage}{}{}
\makeoddfoot{plain}{}{}{\thepage}
\makeevenhead{plain}{}{}{}
\makeoddhead{plain}{}{}{}

\makeatletter %
\makechapterstyle{standard}{
  \setlength{\beforechapskip}{2\baselineskip}
  \setlength{\midchapskip}{0\baselineskip}
  \setlength{\afterchapskip}{2\baselineskip}
  \renewcommand{\chapterheadstart}{\vspace*{\beforechapskip}}
  \renewcommand{\chapnamefont}{\normalfont\Large}
  \renewcommand{\printchaptername}{}
  \renewcommand{\chapternamenum}{\space}
  \renewcommand{\chapnumfont}{\normalfont\Large}
  %\renewcommand{\printchapternum}{\chapnumfont \thechapter.}
  %\renewcommand{\afterchapternum}{\par\nobreak\vskip \midchapskip}
  \renewcommand{\afterchapternum}{ }
  \renewcommand{\printchapternonum}{\vspace*{\midchapskip}\vspace*{5mm}}
  \renewcommand{\chaptitlefont}{\bfseries\LARGE}
  \renewcommand{\printchaptertitle}[1]{\chaptitlefont ##1}
  \renewcommand{\afterchaptertitle}{\par\nobreak\vskip \afterchapskip}
}
\makeatother

\chapterstyle{standard}

\makeatletter %
\makechapterstyle{appendix}{
  \setlength{\beforechapskip}{2\baselineskip}
  \setlength{\midchapskip}{0\baselineskip}
  \setlength{\afterchapskip}{2\baselineskip}
  \renewcommand{\chapterheadstart}{\vspace*{\beforechapskip}}
  \renewcommand{\chapnamefont}{\normalfont\Large}
  \renewcommand{\printchaptername}{\chapnamefont \@chapapp}
  \renewcommand{\chapternamenum}{\space}
  \renewcommand{\chapnumfont}{\normalfont\Large}
  \renewcommand{\printchapternum}{\chapnumfont \thechapter.}
  %\renewcommand{\afterchapternum}{\par\nobreak\vskip \midchapskip}
  \renewcommand{\afterchapternum}{ }
  \renewcommand{\printchapternonum}{\vspace*{\midchapskip}\vspace*{5mm}}
  \renewcommand{\chaptitlefont}{\bfseries\LARGE}
  \renewcommand{\printchaptertitle}[1]{\chaptitlefont ##1}
  \renewcommand{\afterchaptertitle}{\par\nobreak\vskip \afterchapskip}
}
\makeatother

\setlength{\columnseprule}{1pt} %Line between paragraphs

\tikzset{
  % style to apply some styles to each segment of a path
  on each segment/.style={
    decorate,
    decoration={
      show path construction,
      moveto code={},
      lineto code={
        \path [#1]
        (\tikzinputsegmentfirst) -- (\tikzinputsegmentlast);
      },
      curveto code={
        \path [#1] (\tikzinputsegmentfirst)
        .. controls
        (\tikzinputsegmentsupporta) and (\tikzinputsegmentsupportb)
        ..
        (\tikzinputsegmentlast);
      },
      closepath code={
        \path [#1]
        (\tikzinputsegmentfirst) -- (\tikzinputsegmentlast);
      },
    },
  },
  % style to add an arrow in the middle of a path
  end arrow/.style={postaction={decorate,decoration={
        markings,
        mark=at position 0.999 with {\arrow[#1]{stealth}}
      }}},
} % Curved lines

% Declaración de comandos y operadores
\newcommand\RR{\mathbb R}
\newcommand\NN{\mathbb N}
\newcommand\PP{\mathbb P}
\newcommand\dpartial[1]{\frac{\partial}{\partial #1}}
\newcommand\deriv[1]{\frac{d}{d #1}}
\newcommand\integral[4]{\int_{#1}^{#2} #3 \, d#4}
\newcommand*\circled[1]{\tikz[baseline=(char.base)]{
            \node[shape=circle,draw,inner sep=2pt] (char) {#1};}}
\DeclareMathOperator\Ber{Bernoulli}

% Se definen los comandos para escribir teoremas, definiciones y demás.
\theoremstyle{plain}
\newtheorem*{theorem}{Teorema}
\newtheorem*{corollary}{Corolario}
\newtheorem*{lemma}{Lema}
\newtheorem*{proposition}{Proposici\'on}
\theoremstyle{definition}
\newtheorem*{definition}{Definici\'on}
\theoremstyle{remark}
\newtheorem*{remark}{Observaci\'on}

\begin{document}

%%%%%%%%%%%%%%%%%%%%%%%%%
% Aquí va la portada
%%%%%%%%%%%%%%%%%%%%%%%%%

\begin{titlingpage} % Portada

    \raggedleft % Alineada a la derecha
    %\raggedright % Alineada a la izquierda
	
	\vspace*{\baselineskip} % Whitespace at the top of the page
	
	\vspace*{0.25\textheight} % Whitespace before the title
	
	%------------------------------------------------
	%	Cosas del título
	%------------------------------------------------
    
    \vspace*{0.1\textheight}

    {\Huge{\textbf{Mercados Financieros y Valuación\\ de Proyectos}}}\\[\baselineskip] % Aquí va el título
    \vspace*{0.1\textheight}

    %------------------------------------------------
	%	Aquí van los nombres
	%------------------------------------------------
    
    {\Large Eduardo Selim Matínez Mayorga}\\[\baselineskip]
	
	\vfill

\end{titlingpage}

\thispagestyle{empty}

Algunas aplicaciones...
\section*{Bonos}
Un bono es un instrumento que a cambio de un precio da derecho al comprador un flujko de efectivo futuro.
\begin{itemize}
    \item A la persona que desembolsa $P$ y recibe el flujo se le llama comprador 
    \item A la persona que recibe $P$ y da el flujo se le llama vendedor/emisor
\end{itemize}
\subsection*{Tipos}
\textbf{Bono cupón cero.}\\
Una de las partes desembolsa $P$ (el precio del bono) a cambio de una cantidad futura, conocida como valor de redención $(C)$, en fecha futura (plazo del bono).

\begin{center}
\tikzset{every picture/.style={line width=0.75pt}} %set default line width to 0.75pt        
\begin{tikzpicture}[x=0.75pt,y=0.75pt,yscale=-1,xscale=1]
%uncomment if require: \path (0,300); %set diagram left start at 0, and has height of 300
%Straight Lines [id:da6097228026432182] 
\draw    (171.5,100) -- (400.5,100) ;
%Straight Lines [id:da9710225863971634] 
\draw    (171,92) -- (171,108) ;
%Straight Lines [id:da20929424073016667] 
\draw    (400,92) -- (400,108) ;
% Text Node
\draw (166,73) node [anchor=north west][inner sep=0.75pt]   [align=left] {\textit{P}};
% Text Node
\draw (394,74) node [anchor=north west][inner sep=0.75pt]   [align=left] {\textit{C}};
% Text Node
\draw (393.4,107.2) node [anchor=north west][inner sep=0.75pt]   [align=left] {\textit{n}};
\end{tikzpicture}
\end{center}

\textcolor{blue}{\textit{¿Cuánto cuesta un bono cupón cero?}}\\
$P=V^nC = (1+j)^{-n}C$ donde a "$j$" se le concoce como tasa de rendimiento del bono y "$n$" es el plazo del bono.

\begin{remark}
\begin{enumerate}
    \item En general se asocia a los bonos cupón cero con instrumentos muy seguros emitidos por los gobiernos.
    \item En México, el bono \textit{cupón-cero} por excelencia son los CETES (con plazos del bono variados).
    \item Muchos modelos matemáticos suponen la existencia de un bono \textit{cupón-cero}.
\end{enumerate}
\end{remark}


\textbf{Bono cuponados}
A cambio de un precio el día de hoy $(P)$, se recibe periodicamente \underline{cupones} y en la fecha de vencimiento del bono se recibe además la cantidad adicional. $C$, conocida como \textit{valor de redención}.

\begin{center}


\tikzset{every picture/.style={line width=0.75pt}} %set default line width to 0.75pt        

\begin{tikzpicture}[x=0.75pt,y=0.75pt,yscale=-1,xscale=1]
%uncomment if require: \path (0,300); %set diagram left start at 0, and has height of 300

%Straight Lines [id:da4478757222579295] 
\draw    (182.43,151.9) -- (421.14,151.9) ;
%Straight Lines [id:da5591492419882169] 
\draw    (186.33,146.73) -- (186.33,156.9) ;
%Straight Lines [id:da24334683125757495] 
\draw    (416.33,147.07) -- (416.33,157.23) ;
%Straight Lines [id:da8829990491144546] 
\draw    (226.33,146.73) -- (226.33,156.9) ;
%Straight Lines [id:da9395682183740436] 
\draw    (266.33,146.73) -- (266.33,156.9) ;
%Straight Lines [id:da0461267558830839] 
\draw    (376.33,146.73) -- (376.33,156.9) ;
%Curve Lines [id:da4997661226682054] 
\draw    (277.4,128.8) .. controls (295.34,130.36) and (296.55,118.99) .. (296.6,93.95) ;
\draw [shift={(296.6,92)}, rotate = 450] [color={rgb, 255:red, 0; green, 0; blue, 0 }  ][line width=0.75]    (10.93,-3.29) .. controls (6.95,-1.4) and (3.31,-0.3) .. (0,0) .. controls (3.31,0.3) and (6.95,1.4) .. (10.93,3.29)   ;
%Curve Lines [id:da8582552555803019] 
\draw    (440,139.8) .. controls (456.02,136.91) and (460.3,136.8) .. (465.62,152.62) ;
\draw [shift={(466.2,154.4)}, rotate = 252.35] [color={rgb, 255:red, 0; green, 0; blue, 0 }  ][line width=0.75]    (10.93,-3.29) .. controls (6.95,-1.4) and (3.31,-0.3) .. (0,0) .. controls (3.31,0.3) and (6.95,1.4) .. (10.93,3.29)   ;

% Text Node
\draw (277.2,70.2) node [anchor=north west][inner sep=0.75pt]  [font=\fontsize{0.57em}{0.68em}\selectfont] [align=left] {\begin{minipage}[lt]{29.92pt}\setlength\topsep{0pt}
\begin{center}
Cantidad \\de dinero
\end{center}

\end{minipage}};
% Text Node
\draw (299.6,166) node [anchor=north west][inner sep=0.75pt]   [align=left] {. . .};
% Text Node
\draw (414.6,133) node [anchor=north west][inner sep=0.75pt]  [font=\footnotesize]  {$+C$};
% Text Node
\draw (176.4,165.2) node [anchor=north west][inner sep=0.75pt]  [font=\small]  {$P_{0}$};
% Text Node
\draw (220.4,166.2) node [anchor=north west][inner sep=0.75pt]  [font=\small]  {$1$};
% Text Node
\draw (260.4,166.2) node [anchor=north west][inner sep=0.75pt]  [font=\small]  {$2$};
% Text Node
\draw (358.8,165.8) node [anchor=north west][inner sep=0.75pt]  [font=\small]  {$n-1$};
% Text Node
\draw (409.8,165.8) node [anchor=north west][inner sep=0.75pt]  [font=\small]  {$n$};
% Text Node
\draw (217.84,145.01) node [anchor=north west][inner sep=0.75pt]  [font=\small,rotate=-269.9] [align=left] {\textit{{\small cupón}}};
% Text Node
\draw (257.04,145.21) node [anchor=north west][inner sep=0.75pt]  [font=\small,rotate=-269.9] [align=left] {\textit{{\small cupón}}};
% Text Node
\draw (367.84,145.01) node [anchor=north west][inner sep=0.75pt]  [font=\small,rotate=-269.9] [align=left] {\textit{{\small cupón}}};
% Text Node
\draw (402.84,145.01) node [anchor=north west][inner sep=0.75pt]  [font=\small,rotate=-269.9] [align=left] {\textit{{\small cupón}}};
% Text Node
\draw (434,156.6) node [anchor=north west][inner sep=0.75pt]  [font=\fontsize{0.57em}{0.68em}\selectfont] [align=left] {\begin{minipage}[lt]{44.88pt}\setlength\topsep{0pt}
\begin{center}
Puede ser un\\valor Payment
\end{center}

\end{minipage}};


\end{tikzpicture}

\end{center}

\textcolor{blue}{\textit{¿Cuánto cuesta un bono cuponado?}}\\
$P= \textit{cupón} \cdot \ax{\angln j} + C(1+j)^{-n}$ donde a "$n$" es el plazo del bono y "$j$" es la \textit{tasa de rendimiento} del bono.

\textcolor{blue}{\textit{¿Cómo establecemos la cantidad del cupón y el valor de redención?}} Artificialmente nosotros los decidimos. \\

\textit{Over the counter: Negociación del bono personalmente}
Saber cuándo ocurre cada pago y traer a valor presente.
Emiten bonos las compañías (salir a pedir prestado) para no dañar el patrimonio propio.

\textcolor{blue}{Notación:}
\begin{itemize}
    \item $P$: precio del bono
    \item $C$: valor de redención
    \item $n$: plazo del bono
    \item $j$: tasa del rendimiento del bono
    \item $F$: valor nominal / facial-face value
    \item $r$: tasa cupón del bono
\end{itemize}

\begin{empheq}[box=\fbox]{flalign*}
&\text{Así se tiene que:}&\\
& P = Fr\cdot\ax{\angln} + C(1+j)^{-n} \text{, donde } j \text{ es efectiva}&
\end{empheq}

\begin{proposition}
El precio del bono también se puede calcular como:
\begin{enumerate}
    \item Fórmula \textit{premio-descuento} $P=C+(Fr-C(j))\ax{\angln}$
    \item Fórmula \textit{cantidad base} $P=G + (C-G)(1+j)^{-n}$, donde $G$ es tal que $Fr=G_j$
    \item Fórmula de \textbf{Makenam} $P=k+\frac{g}{j}(C-k)$, donde $Fr=Cg$ y $k=C(1+j)^{-n}$
\end{enumerate}
\end{proposition}
\textit{\underline{Dem.}}
\begin{enumerate}
    \item $\ax{\angln j} = \frac{1-(1+j)^{-n}}{j} \Rightarrow 1-j\cdot\ax{\angln} = (1+j)^{-n}$
    \begin{align*}
        P &= Fr\cdot\ax{\angln} + C(1-j\ax{\angln})\\
         &= Fr\cdot\ax{\angln} + C - Cj\ax{\angln}\\
         &= C + \ax{\angln}(Fr-Cj)\\
         \therefore\quad P &= C+ (Fr-(Cj)\ax{\angln}\qed
    \end{align*}
    \item \begin{align*}
        P &= Fr\cdot\ax{\angln} + C(1+j)^{-n}\\
        &= Gj\cdot \ax{\angln} + C(1+j)^{-n}\\
        &= Gj\Big(\frac{1-v^n}{j}\Big) + C(1+j)^{-n}\\
        &= G(1-(1+j)^{-n}) + C(1+j)^{-n}\\
        &= G - G(1+j)^{-n} + C(1+j)^{-n}\\
        \therefore\quad P &= G + (1+j)^{-n}[C-G]\qed
    \end{align*}
    \item Sabemos que $Fr = Cg$ y que $k=C(1+j)^{-n}$\begin{align*}
        P &= Fr\cdot\ax{\angln} + C(1+j)^{-n}\\
        &= Cg\cdot\ax{\angln} + k \\
        &= Cg\Big(\frac{1-(1+j)^{-n}}{j}\Big) + k\\
        &= \frac{Cg}{j} - \frac{gC(1+j)^{-n}}{j}+k\\
        &= \frac{Cg}{j} - \frac{gk}{j} + k\\
        \therefore\quad P &= k + \frac{g}{j}(C-k)\qed
    \end{align*}
\end{enumerate}

A $G$ se le conoce como \underline{cantidad base} $G=\frac{Fr}{j}$ y a $g$ se le conoce como \underline{tasa cupón modificada} $g=\frac{Fr}{C}$

\textcolor{blue}{\textit{¿Cómo se interpretan $G$ y $g$?}} Cuando compra un bono, desembolsa $P$ e implícitamente está \textit{prestando} el valor de redención "$C$", al tiempo $n$. Es decir que usted le prestó "$C$" pesos al emisor del bono y este se los pagará en $n$.

\begin{empheq}[box=\fbox]{flalign*}
&\textit{¿Puedo considerar los cupones como pagos de interés sobre C?}\quad\text{No.}&\\
& \textit{¿Cuánto es el interés sobre la deuda C, por un periodo?}\quad\text{Al final de un periodo se acumulan }&\\
&\quad C(1+j) \text{ pero el puro interés es de } C_j.&
\end{empheq}

Sólo se puede considerar los cupones como pago de interés cuando $Fr=C_j$ y en este caso, según la fórmula \textit{premio/descuento}:
$$P=C+(Fr-C_j)\ax{\angln}=C+0=C$$

Es decir, con los cupones vas pagando el interés que te estás "metiendo" periodo a periodo.

\textcolor{blue}{\textit{Los cupones alcanzan a pagar el interés, van pagando los intereses por periodo y así no permites que la deuda crezca.}}

Si $Fr>C_j$ según $\boxed{\text{P/D = Premio/Descuento}}$ 
\begin{align*}
    P &= C + (Fr - C_j) \ax{\angln}\\
    & i.e.\quad P>C
\end{align*}

Si $Fr<C_j$ según P/D 
\begin{align*}
    P &= C + (Fr - C_j) \ax{\angln}\\
    & i.e.\quad P<C
\end{align*}

\textit{Y de hecho...}
\begin{itemize}
    \item Si $P>C$ entonces $Fr-C_j>0$ i.e. $Fr>C_j$
    \item Si $P<C$ entonces $Fr-Cj<0$ pues $\ax{\angln}>0$ i.e. $Fr<C_j$
\end{itemize}
\textit{Conclusión}
\begin{align*}
    Fr>C_j &\Leftrightarrow P>C\\
    Fr<C_j &\Leftrightarrow P<C
\end{align*}

 


\begin{empheq}[box=\fbox]{flalign*}
&\textcolor{blue}{\text{Notación:}}\\
&\text{Si } P>C \text{ lo que estás pagando es mayor a lo que ganas, se dice que el bono se \textit{compra/vende} }&\\
&\quad\text{como \textbf{premio}.}&\\
& \text{Se dice que el bono se \textit{compra/vende} y a } C-P \text{ se le conoce como \textbf{descuento}.}&
\end{empheq}

\begin{remark}
En inglés se ocupa la palabra \textit{trade} y no hay una tradición simple de este, por eso se ocupa \textit{compra/vende}
\end{remark}

\begin{multicols}{2}
\textit{¿Cómo interpreto $G$?}
$$Fr=Gj$$
$G$ es la cantidad que tendría que valer \textit{el valor de redención}  $C$ para que los cupones $Fr$ pagaran completamente el interés por periodo periodicamente.

\columnbreak

\textit{¿Cómo interpreto $g$?}
$$Fr=Cg$$
$g$ es la tasa de interés necesaria para $j$ para que los cupones $Fr$ pagaran completamente el interés $C_j$ periódicamente.
\end{multicols}

\section*{Amortización y Sinking funds (fondos de acumulación)}

Considérese una deuda que se toma hoy, de cantidad $L$, que se liquidará/pagará mediante n-pagos de $R_1,R_2,...,R_n$ (no necesariamente iguales) que se hacen en los tiempos $1,2,...,n$ respectivamente.

\begin{center}
    

\tikzset{every picture/.style={line width=0.75pt}} %set default line width to 0.75pt        

\begin{tikzpicture}[x=0.75pt,y=0.75pt,yscale=-1,xscale=1]
%uncomment if require: \path (0,300); %set diagram left start at 0, and has height of 300

%Straight Lines [id:da6126788792215244] 
\draw    (101,151.9) -- (465,151.9) ;
%Straight Lines [id:da34187503823616106] 
\draw    (150.83,147.23) -- (150.83,157.4) ;
%Straight Lines [id:da7863161656982951] 
\draw    (416.33,147.07) -- (416.33,157.23) ;
%Straight Lines [id:da4565159098709549] 
\draw    (109.83,147.23) -- (109.83,157.4) ;
%Straight Lines [id:da6752537344318184] 
\draw    (190.83,147.23) -- (190.83,157.4) ;
%Straight Lines [id:da055266829754114566] 
\draw    (269.83,147.23) -- (269.83,157.4) ;
%Straight Lines [id:da9839947143070265] 
\draw    (309.83,147.73) -- (309.83,157.9) ;
%Straight Lines [id:da6651881669338392] 
\draw    (350.33,147.23) -- (350.33,157.4) ;
%Straight Lines [id:da1938018116507373] 
\draw    (460.83,147.07) -- (460.83,157.23) ;
%Curve Lines [id:da39046551371801963] 
\draw    (140.5,128) .. controls (145,101.5) and (464,105.5) .. (470,131.5) ;

% Text Node
\draw (215.6,157.5) node [anchor=north west][inner sep=0.75pt]   [align=left] {. . .};
% Text Node
\draw (104.9,161.7) node [anchor=north west][inner sep=0.75pt]  [font=\small]  {$L$};
% Text Node
\draw (145.4,161.7) node [anchor=north west][inner sep=0.75pt]  [font=\small]  {$1$};
% Text Node
\draw (186.4,161.7) node [anchor=north west][inner sep=0.75pt]  [font=\small]  {$2$};
% Text Node
\draw (253.8,162.3) node [anchor=north west][inner sep=0.75pt]  [font=\small]  {$k-1$};
% Text Node
\draw (399.8,161.8) node [anchor=north west][inner sep=0.75pt]  [font=\small]  {$n-1$};
% Text Node
\draw (305.8,162.8) node [anchor=north west][inner sep=0.75pt]  [font=\small]  {$k$};
% Text Node
\draw (334.3,162.8) node [anchor=north west][inner sep=0.75pt]  [font=\small]  {$k+1$};
% Text Node
\draw (454.8,160.3) node [anchor=north west][inner sep=0.75pt]  [font=\small]  {$n$};
% Text Node
\draw (371.1,159) node [anchor=north west][inner sep=0.75pt]   [align=left] {. . .};
% Text Node
\draw (142.9,129.2) node [anchor=north west][inner sep=0.75pt]  [font=\small]  {$R_{1}$};
% Text Node
\draw (183.4,129.2) node [anchor=north west][inner sep=0.75pt]  [font=\small]  {$R_{2}$};
% Text Node
\draw (261.9,129.2) node [anchor=north west][inner sep=0.75pt]  [font=\small]  {$R_{k-1}$};
% Text Node
\draw (302.9,129.7) node [anchor=north west][inner sep=0.75pt]  [font=\small]  {$R_{k}$};
% Text Node
\draw (341.9,129.2) node [anchor=north west][inner sep=0.75pt]  [font=\small]  {$R_{k+1}$};
% Text Node
\draw (406.4,129.7) node [anchor=north west][inner sep=0.75pt]  [font=\small]  {$R_{n-1}$};
% Text Node
\draw (452.9,129.7) node [anchor=north west][inner sep=0.75pt]  [font=\small]  {$R_{n}$};
% Text Node
\draw (267.9,93.2) node [anchor=north west][inner sep=0.75pt]  [font=\scriptsize]  {$n-pagos$};


\end{tikzpicture}

\end{center}

Para una tasa efectiva de interés $j$ por periodo, esta transacción debe satisfacer:
\begin{align*}
    L &= R_1(1+j)^{-1} + R_2(1+j)^{-2} +...+ R_n(1+j)^{-n}\\
    L &= \sum_{k=1}^n R_k(1+j)^{-k} \quad\dotsm\quad\circledast
\end{align*}
Si $R_k=R$ constante, entonces $L=R\ax{\angln j}$

Al final del $1^{er}$ periodo, ¿cuánto vale la deuda?
$$L(1+j)$$
Sin embargo, yo pago $R$, en ese momento, entonces en realidad debo $L(1+j)-R_1$ y sustituyendo en $\circledast$ se tiene que se debe \Big($\sum_{k=1}^nR_k(1+j)^{-k}\Big)(1+j)-R_1$
\begin{align*}
    &= \cancel{R_1}\cancel{(1+j)^{-1}(1+j)} + \sum_{k=2}^nR_k(1+j)^{-k}(1+j)-\cancel{R_1}\\
    &= \Big(\sum_{k=2}^nR_k(1+j)^{-k}\Big)(1+j) := OB_1 \textit{ (esto se debe luego del primer pago)}
\end{align*}

Al final del segundo periodo se debe:
$$OB_1(1+j)$$
Sin embargo, pago $R_2$ en ese momento y la deuda real será
\begin{align*}
    OB_1(1+j)-R_2 &= \Bigg[\Big(\sum_{k=2}^nR_k(1+j)^{-k}\Big)(1+j)\Bigg](1+j) - R_2\\
    &= \cancel{R_2}\cancel{(1+j)^{-2}(1+j)^2} + \bigg[\sum_{k=3}^nR_k(1+j)^{-k}\bigg](1+j)^2 - \cancel{R_2}\\
    &= \bigg[\sum_{k=3}^nR_k(1+j)^{-k}\bigg](1+j)^2 := OB_2
\end{align*}

Inductivamente, al final del \textit{k-ésimo} periodo, ¿cuánto vale la deuda? $OB_{k-1}(1+j)$, sin embargo, yo pago $R_k$ en ese momento y en realidad debo:
$$OB_{k-1}(1+j)-R_k= \bigg[\sum_{l=k+1}^nR_l(1+j)^{-l}\bigg] (1+j)^k = OB_k$$

\textit{¿Qué relación hay entre las $OB_k^{15}$}
$$OB_{k-1} (1+j) -R_k =OB_k$$
\textit{\underline{Observación}}
\begin{align*}
    OB_k &= \big( \sum_{l=k+1}^n R_l(1+j)^{-l} \big)(1+j)^k\\
    &= \sum_{l=k+1}^n R_l(1+j)^{k-l}
\end{align*}

\begin{center}
    \tikzset{every picture/.style={line width=0.75pt}} %set default line width to 0.75pt        
\begin{tikzpicture}[x=0.75pt,y=0.75pt,yscale=-1,xscale=1]
%uncomment if require: \path (0,300); %set diagram left start at 0, and has height of 300

%Straight Lines [id:da6914744590738869] 
\draw    (121,171.9) -- (485,171.9) ;
%Straight Lines [id:da6670825984666403] 
\draw    (170.83,167.23) -- (170.83,177.4) ;
%Straight Lines [id:da13988048709948842] 
\draw    (436.33,167.07) -- (436.33,177.23) ;
%Straight Lines [id:da8289635545012434] 
\draw    (129.83,167.23) -- (129.83,177.4) ;
%Straight Lines [id:da2601125601778915] 
\draw    (210.83,167.23) -- (210.83,177.4) ;
%Straight Lines [id:da6107464454575617] 
\draw    (289.83,167.23) -- (289.83,177.4) ;
%Straight Lines [id:da03272646136315682] 
\draw    (329.83,167.73) -- (329.83,177.9) ;
%Straight Lines [id:da9238373514074949] 
\draw    (370.33,167.23) -- (370.33,177.4) ;
%Straight Lines [id:da3030272522392645] 
\draw    (480.83,167.07) -- (480.83,177.23) ;
%Curve Lines [id:da32532122292629984] 
\draw    (481,142) .. controls (485.95,120.22) and (358.59,93.54) .. (336.63,139.58) ;
\draw [shift={(336,141)}, rotate = 292.62] [color={rgb, 255:red, 0; green, 0; blue, 0 }  ][line width=0.75]    (10.93,-3.29) .. controls (6.95,-1.4) and (3.31,-0.3) .. (0,0) .. controls (3.31,0.3) and (6.95,1.4) .. (10.93,3.29)   ;
%Curve Lines [id:da7330682434681304] 
\draw    (440,146) .. controls (440,123.34) and (364.32,116.21) .. (337.2,139.9) ;
\draw [shift={(336,141)}, rotate = 316.12] [color={rgb, 255:red, 0; green, 0; blue, 0 }  ][line width=0.75]    (10.93,-3.29) .. controls (6.95,-1.4) and (3.31,-0.3) .. (0,0) .. controls (3.31,0.3) and (6.95,1.4) .. (10.93,3.29)   ;
%Curve Lines [id:da4242864339392747] 
\draw    (381,143) .. controls (381,120.34) and (362.57,116.12) .. (337.17,139.89) ;
\draw [shift={(336,141)}, rotate = 316.12] [color={rgb, 255:red, 0; green, 0; blue, 0 }  ][line width=0.75]    (10.93,-3.29) .. controls (6.95,-1.4) and (3.31,-0.3) .. (0,0) .. controls (3.31,0.3) and (6.95,1.4) .. (10.93,3.29)   ;

% Text Node
\draw (235.6,177.5) node [anchor=north west][inner sep=0.75pt]   [align=left] {. . .};
% Text Node
\draw (124.9,181.7) node [anchor=north west][inner sep=0.75pt]  [font=\small]  {$L_{0}$};
% Text Node
\draw (165.4,181.7) node [anchor=north west][inner sep=0.75pt]  [font=\small]  {$1$};
% Text Node
\draw (206.4,181.7) node [anchor=north west][inner sep=0.75pt]  [font=\small]  {$2$};
% Text Node
\draw (273.8,182.3) node [anchor=north west][inner sep=0.75pt]  [font=\small]  {$k-1$};
% Text Node
\draw (419.8,181.8) node [anchor=north west][inner sep=0.75pt]  [font=\small]  {$n-1$};
% Text Node
\draw (325.8,182.8) node [anchor=north west][inner sep=0.75pt]  [font=\small]  {$k$};
% Text Node
\draw (354.3,182.8) node [anchor=north west][inner sep=0.75pt]  [font=\small]  {$k+1$};
% Text Node
\draw (474.8,180.3) node [anchor=north west][inner sep=0.75pt]  [font=\small]  {$n$};
% Text Node
\draw (391.1,179) node [anchor=north west][inner sep=0.75pt]   [align=left] {. . .};
% Text Node
\draw (361.9,149.2) node [anchor=north west][inner sep=0.75pt]  [font=\small]  {$R_{k+1}$};
% Text Node
\draw (426.4,149.7) node [anchor=north west][inner sep=0.75pt]  [font=\small]  {$R_{n-1}$};
% Text Node
\draw (472.9,149.7) node [anchor=north west][inner sep=0.75pt]  [font=\small]  {$R_{n}$};

\end{tikzpicture}
\end{center}

\begin{align*}
    R_{k+1}(1+j)&^{k-k-1} + R_{k+2}(1+j)^{k-k-2} + R_{k+3}(1+j)^{k-k-3} + ... + R_{k+n}(1+j)^{k-n}\\
    &= R_{k+1}(1+j)^{-1} + R_{k+2}(1+j)^{-2} +R_{k+3}(1+j)^{-3} + ... + R_{k+n}(1+j)^{-(n-k)}
\end{align*}

La ecuación dice que traemos al tiempo $k$ los pagos faltantes para liquidar la deuda.

\textbf{Notación.}\\
\begin{itemize}
    \item $OB_k$: Saldo restante al tiempo $k$ \textit{(outstanding balance $k$)}
    \item $k$: El valor presente de lo que me falta por pagar
\end{itemize}

Si todos los pagos son iguales \textit{i.e.} $R_k = R$, $k\in\{1,...,n\}$

\begin{align*}
    OB_k &= R(1+j)^{-1} + R(1+j)^{-2} + ... + R(1+j)^{-(n-k)}\\
    &= R(v) + Rv^2 + ... + Rv^{n-k} \\
    &= R(v+v^2+...+v^{n-k} = R\ax{\angl{n-k}j}
\end{align*}

\textit{\underline{Observación}}
$$L(1+j)^k - \sum_{l=1}^k R_{l}(1+j)^{k-l+1}$$
$Conjetura=OB_k$
\begin{align*}
    OB_k^{[R]} &:= L(1+j)^k - \sum_{l=1}^k R_{l}(1+j)^{k-l+1} \textit{ retrospectivo}\\
    OB_k^{[P]} &:= \sum_{l=1}^n R_{l}(1+j)^{-} \textit{ prospectivo}
\end{align*}

Amortización = de latin mortis $\rightarrow$ darle \underline{muerte} a una deuda\\
$Conjetura\quad OB_k^{[P]} = OB_k^{[R]}$

\textit{\underline{Repaso}}\\
$OB_{k}$: Outstanding balance al tiempo $k$, saldo insoluto al tiempo $k$. Saldo restante al tiempo $t$. $OB_k$ es la cantidad que se debe justo después del $k$-ésimo pago.\\
$$OB_1 = L(1+j) - R_1$$
Licencia: $OB_0=L$ y en general...
$$OB_k= OB_{k-1}(1+j) - R_k,\quad k\in\{1,2,...,n\}$$

Deducimos que:
$$OB_k = \big[ \sum_{m=k+1}^n R_m(1+j)^{-m}\big](1+j)^k$$
Demostramos que:
$$OB_k = R_{k+1}(1+j)^{-1} + R_{k+2}(1+j)^{-2}+...+ R_n(1+j)^{-(n-k)}$$

Esto motivó la definición:
$$ OB_k^{[P]} = R_{k+1}(1+j)^{-1} + R_{k+2}(1+j)^{-2} + ... + R_{k+n}(1+j)^{-(n-k)}$$

$OB_k^{[P]}$: Prospectivo outstanding balance al tiempo $k$, saldo insoluto prospectivo al tiempo $k$\\

''Artificialmente'' definimos:
\begin{align*}
    OB_{k}^{[R]} &= L(1+j)^k - [R_1(1+j)^{k-1} + ... + R_{k+1}(1+j)^1] + R_k(1+j)^0\\
    &= L(1+j)^k - \big( R_1(1+j)^{k-1} + ... + R_{k-1}(1+j) + R_k \big)
\end{align*}

$OB_k^{[R]}$: Retrospective outstanding balance al tiempo $k$ saldo insoluto retrospectivo al tiempo $k$\\

De manera sucinta...
$$OB_k^{[R]} = L(1+j)^k - \sum_{m=1}^kR_m(1+j)^{k-m}$$

¿Cómo interpretamos/construimos $OB_k^{R}$?
$$\underbrace{OB_k^{R}}_{\text{La deuda al tiempo }k} = \underbrace{L(1+j)^k}_{\text{Si no hubiere hecho ningún pago, debería esto al tiempo }k} - \underbrace{\sum_{m=1}^k R_m(1+j)^{k-m}}_{\text{Sin embargo, sí se hacen pagos}}$$

Conjetura... $OB_k^{[R]} = OB_k^{[P]}$\\
\underline{\textit{Dem.}} \\
\begin{align*}
    OB_k^{[R]} &= L(1+j)^k - \sum_{m=1}^k R_m(1+j)^{k-m}\\
    &= \big( \sum_{m=1}^n R_m(1+j)^{-m} \big) (1+j)^k - \sum_{m=1}^k R_m(1+j)^{k-m}\\
    &= \sum_{m=1}^n R_m(1+j)^{k-m} - \sum_{m=1}^k R_m(1+j)^{k-m}\\
    &= \sum_{m=k+1}^n R_m(1+j)^{k-m} = R_{k+1}(1+j)^{-1} + R_{k+2} + ... + r_n(1+j)^{-(n-k)} = OB_k^{[P]}\\
    &\quad\quad\quad \therefore OB_k^{[P]} = OB_k^{[R]}
\end{align*}

\textit{\underline{Observación}}
\begin{align*}
    OB_k &= OB_{k-1}(1+j) - R_k\\
    &= OB_{k-1} + \underbrace{OB_{k-1}\cdot j}_{I_k} - R_k\\
    &= OB_{k-1} + I_k - R_k \textit{, donde }I_k=OB_{k-1}\cdot j\\
    & \quad I_k := \text{ El interés de la deuda al tiempo }k\\
    &= OB_{k-1} - \underbrace{(R_k-I_k)}_{P_k}\\
    &= OB_{k-1} - P_k \quad \textit{donde }P_k := R_k - I_k
\end{align*}

$P_k$: Pago de principal (de la deuda) al tiempo $k$, y respresenta la parte de $R_k$ que efectivamente disminuyó la deuda y no sólo interés. \textit{Lo que realmente te ayudó a disminuir tu deuda.}\\

\textbf{Notación.} \\
$$R_k = P_k + I_k \textit{ i.e. cada pago se divide en pago de principal y pago de interés}$$

\textbf{Casos}\\
1. Que todos los pagos sean iguales $\forall i, 1 \leq i \leq n, R$,  con $R$ constante:
\begin{itemize}
    \item \begin{align*} OB_k &= R(1+j)^{-1} + R(1+j)^{-2} + ... + R(1+j)^{-(n-k)}\\
    &= R\big[(1+j)^{-1} + (1+j)^{-2} +...+ (1+j)^{-(n-k)}\big]\\
    &= R\big(v+v^2+...+v^{n-k}\big]\\
    &= R\ax{\angl{n-k}}
    \end{align*}
    \item \begin{align*} I_k &= OB_{k-1}\cdot j\\
        &= R\ax{\angl{n-(k-1)}}\cdot j\\
        &= R\ax{\angl{n-k+1}}\cdot j\\
        &= R\big(\frac{1-r^{n-k+1}}{j} = R(1-v^{n-k+1}) = R - Rv^{n-k+1}
    \end{align*}
    \item \begin{align*} P_k &= R_k -I_k = R- (R-Rv^{n-k+1}\\
    &= Rv^{n-k+1} = R(1+j)^{k-n-1}
    \end{align*}
\end{itemize}

\underline{\textit{Observación}}\\
(1) $k\to R_k$ es creciente, esto significa que conforme va pasando el tiempo los pagos efectivos a la deuda (\textit{i.e.} sin interés) son mayor.\\

(2) $k\to I_k$ es decreciente, esto significa que conforme va pasando el tiempo los pagos de interés son menores. Por lo que, tendremos la \underline{tabla de amortización}.\\



\begin{table}[h]
\begin{tabular}{c|c|c|c|c}
\multicolumn{5}{c}{Tabla de amortización}                                                                 \\
Periodo & Pago      & Interés                   & Pago Principal            & Saldo Insoluto              \\
$0$     & -         & -                         & -                         & $OB_0=L$                    \\
$1$     & $R_1$     & $I_1=OB_0 \cdot j$         & $P_1=R_1-I_1$             & $OB_1=OB_0-P_1$             \\
$2$     & $R_2$     & $I_2=OB_1 \cdot j$        & $P_2=R_2-I_2$             & $OB_2=OB_1-P_2$             \\
$\vdots$     & $\vdots$       & $\vdots$        & $\vdots$                  & $\vdots$                         \\
$k-1$   & $R_{k-1}$ & $I_{k-1}=OB_{k-2}\cdot j$ & $P_{k-1}=R_{k-1}-I_{k-1}$ & $OB_{k-1}=OB_{k-2}-P_{k-1}$ \\
$k$     & $R_k$     & $I_k=OB_{k-1}\cdot j$     & $P_k=R_k-I_k$             & $OB_k=OB_{k-1}-P_k$         \\
$k+1$   & $R_{k+1}$ & $I_{k+1}=OB_k\cdot j$     & $P_{k+1}=R_{k+1}-I_{k+1}$ & $OB_{k+1}=OB_k-P_{k+1}$     \\
$\vdots$     & $\vdots$      & $\vdots$       & $\vdots$                   & $\vdots$                      \\
$n$     & $R_n$     & $I_n=OB_{n-1}\cdot j$     & $P_n=R_n-I_n$             & $OB_n=OB_{n-1}-P_n=0$      
\end{tabular}
\end{table}

\textit{Tarea.} Rehacer la tabla cuando los pagos son constantes sabieron que:\\
$$I_k=R[1-(1+j)^{-(n-k+1)}], P_k=R(1+j)^{-(n-k+1)}, OB_k=R\ax{\angl{n-k}j}$$

\textit{Herramientas}: Google sheets, Googl drive R
\\

1. $OB_k^{[P]} = VP$ pagos futuros no hechos\\
$OB_k^{[R]}=$ Valor acumulado de la deuda | Valor acumulado de los pagos ya hechos\\

2. Una deuda se pagará mediante pagos trimestrales durante 3 años con una tasa de interés del 8\% anual convertible trimestral. Encontrar el saldo restante de la deuda justo después de 2 años, la deuda es de $10,000$.\\
Pagos trimestrales 3 años. $i^{(4)}= 8\% \to j=\frac{8\%}{4}=2\%$ efect. trim.\\
$n=3(4) = 12$ pagos trimestrales\\
$L=10,000$\\
$R=945.59$\\
Sabemos que $L=R\ax{\angln i} \Rightarrow 10,000 = R\ax{\angl{12}2\%}$\\
\begin{align*}
    OB_k &= R\cdot\ax{\angl{n-k}} =945.59\cdot\ax{\angl{12-8}2\%} = 945.59\ax{\angl{4}2\%}\\
    &= 945.59\big(\frac{1-(1.02^{-4}}{0.02}\big) = 3,600.55
\end{align*}

6. Deuda $L$. Pagos semestr. nivelados por 2 años $\Rightarrow n=2(2) = 4$ semestres\\
$R_4 = 3,151.49$\\
$I_3 = 122.38\quad\quad OB_1 = 9,088.51$\\
$P_2= 2,969.71 \quad$ Complementar la tabla de amortización 

\begin{table}[h]
\begin{tabular}{c|c|c|c|c}
\multicolumn{5}{c}{Tabla de amortización}                                        \\
Periodo & Pago       & Interés    & Pago Principal & Saldo Insoluto              \\
$0$     & -          & -          & -              & $L=R\ax{\angln j} = 12,000$ \\
$1$     & $3,151.49$ & $240$      & $2,911.49$     & $9,088.51$                  \\
$2$     & $3,151.49$ & $181.7702$ & $2,969.71$     & $6,118.8$                   \\
$3$     & $3,151.49$ & $122.38$   & $3,029.11$     & $3,089.69$                  \\
$4$     & $3,151.49$ & $61.7938$  & $3,089.69$     & $0$
\end{tabular}
\end{table}

\begin{align*}
    P_2 &= Rv^{n-2+1} = Rv^{4-2+1} = Rv^3 \Rightarrow 2,969.71 = 3,151.41(1+j)^{-3}\\
    & \Rightarrow \bigg(\frac{2,969.71}{3,151.41}\bigg)^{-\frac{1}{3}} = (1+j) \Rightarrow j = \bigg(\frac{2,969.71}{3,151.41}\bigg)^{-\frac{1}{3}} - 1 = 2\% \text{ efect. semestral}
\end{align*}

\circled{3} $L= R\ax{\angln j} = 3,151.49\ax{\angl{4}2\%} = 3,151.49\bigg(\frac{1-v^4}{0.02}\bigg) = 12,000.01$\\

\circled{4} $I_1 = 12,000(0.02) = 240$\\

\circled{5} $P_1 = 3,151.49 - 240 = 2,911.49$\\

\circled{6} $9,088.51(0.02) = 181.7702$\\

\circled{7} $OB_1 - P_2 = 9,088.51 - 2,969.71 = 6,118.8$\\

Si no se dice quién es $C \Rightarrow F=C$

\section*{Método de Sinking-fund (Fondo de acumulación)}
En el esquema de amortización, tomamos una deuda hoy y lo vamos pagando poco a poco.

En el método de \textit{Sinking-fond}, tomando una deuda hoy y la pagaremos completa al final de $n$ periodos, sin embargo, se harán depósitos en un fondo (llamado \textit{Sinking-fund} para lograr acumular la cantidad adecuada; además iremos pagando el interés sobre la deuda periódicamente.\\
\textit{Para el sinking fund}
\begin{center}
    

\tikzset{every picture/.style={line width=0.75pt}} %set default line width to 0.75pt        

\begin{tikzpicture}[x=0.75pt,y=0.75pt,yscale=-1,xscale=1]
%uncomment if require: \path (0,300); %set diagram left start at 0, and has height of 300

%Straight Lines [id:da7559120436362685] 
\draw    (121,171.9) -- (485,171.9) ;
%Straight Lines [id:da8772966460107505] 
\draw    (170.83,167.23) -- (170.83,177.4) ;
%Straight Lines [id:da30462270670980895] 
\draw    (436.33,167.07) -- (436.33,177.23) ;
%Straight Lines [id:da5077518918169632] 
\draw    (129.83,167.23) -- (129.83,177.4) ;
%Straight Lines [id:da6736197418737911] 
\draw    (210.83,167.23) -- (210.83,177.4) ;
%Straight Lines [id:da4264489594115969] 
\draw    (289.83,167.23) -- (289.83,177.4) ;
%Straight Lines [id:da6434130448694175] 
\draw    (329.83,167.73) -- (329.83,177.9) ;
%Straight Lines [id:da6286140326762423] 
\draw    (370.33,167.23) -- (370.33,177.4) ;
%Straight Lines [id:da17368503094360932] 
\draw    (480.83,167.07) -- (480.83,177.23) ;
%Curve Lines [id:da8368648508770308] 
\draw    (480.83,171.23) .. controls (496.85,168.34) and (499.24,166.9) .. (504.42,182.62) ;
\draw [shift={(505,184.4)}, rotate = 252.35] [color={rgb, 255:red, 0; green, 0; blue, 0 }  ][line width=0.75]    (10.93,-3.29) .. controls (6.95,-1.4) and (3.31,-0.3) .. (0,0) .. controls (3.31,0.3) and (6.95,1.4) .. (10.93,3.29)   ;

% Text Node
\draw (235.6,177.5) node [anchor=north west][inner sep=0.75pt]   [align=left] {. . .};
% Text Node
\draw (124.9,181.7) node [anchor=north west][inner sep=0.75pt]  [font=\small]  {$0$};
% Text Node
\draw (165.4,181.7) node [anchor=north west][inner sep=0.75pt]  [font=\small]  {$1$};
% Text Node
\draw (206.4,181.7) node [anchor=north west][inner sep=0.75pt]  [font=\small]  {$2$};
% Text Node
\draw (273.8,182.3) node [anchor=north west][inner sep=0.75pt]  [font=\small]  {$k-1$};
% Text Node
\draw (419.8,181.8) node [anchor=north west][inner sep=0.75pt]  [font=\small]  {$n-1$};
% Text Node
\draw (325.8,182.8) node [anchor=north west][inner sep=0.75pt]  [font=\small]  {$k$};
% Text Node
\draw (354.3,182.8) node [anchor=north west][inner sep=0.75pt]  [font=\small]  {$k+1$};
% Text Node
\draw (474.8,180.3) node [anchor=north west][inner sep=0.75pt]  [font=\small]  {$n$};
% Text Node
\draw (391.1,179) node [anchor=north west][inner sep=0.75pt]   [align=left] {. . .};
% Text Node
\draw (162.9,149.2) node [anchor=north west][inner sep=0.75pt]  [font=\small]  {$D_{1}$};
% Text Node
\draw (203.4,149.2) node [anchor=north west][inner sep=0.75pt]  [font=\small]  {$D_{2}$};
% Text Node
\draw (273.9,149.2) node [anchor=north west][inner sep=0.75pt]  [font=\small]  {$D_{k-1}$};
% Text Node
\draw (322.9,149.7) node [anchor=north west][inner sep=0.75pt]  [font=\small]  {$D_{k}$};
% Text Node
\draw (359.9,150.2) node [anchor=north west][inner sep=0.75pt]  [font=\small]  {$D_{k+1}$};
% Text Node
\draw (423.4,149.7) node [anchor=north west][inner sep=0.75pt]  [font=\small]  {$D_{n-1}$};
% Text Node
\draw (472.9,149.7) node [anchor=north west][inner sep=0.75pt]  [font=\small]  {$D_{n}$};
% Text Node
\draw (498.9,185.7) node [anchor=north west][inner sep=0.75pt]  [font=\small]  {$L$};


\end{tikzpicture}

\end{center}

Se debe cumplir que:
\begin{align*}
    L &= D_n + D_{n-1}(1+\hat{j}) + D_{n-2}(1+\hat{j})^{-2} + ... + D_1(1+\hat{j})^{n-1}\\
    &= \sum_{m=1}^n D_m(1+\hat{j})^{n-m} 
\end{align*}

donde $\hat{j}$ es la tasa de interés efectiva por periodo del \textit{ sinking-fund}

¿Cuánto es el interés sobre la deuda en cada periodo? $L\cdot j$ donde $j$ es la tasa de interés efectiva de la deuda. \\
En el \textit{sinking-fund} no disminuye la deuda solo se pagan los intereses.\\

A la cantidad $L\cdot j$ se le conoce como \underline{servicio de la deuda}. \textit{Al que le debes solo le pagas los intereses y tú mismo buscas la forma de pagar }$L$.\\

¿De cuánto es el desembolso total el deudor al tiempo $k$?\\
$$\underbrace{D_k}_{\text{depósito al SF}} + \underbrace{L_j}_{\text{servicio de la deuda}} \leq R_k$$

$R_k$: es la renta si estuviera pagando por amortización\\
Sería ideal que se cumpla la desigualdad para que fuera atractivo para pagar en la amortización.\\

\textit{Tabla del Sinking-fund}
\begin{table}[H]
\begin{tabular}{c|c|c|c|c}
Periodo  & Servicio de la deuda & Depósito al S.F. & Saldo en S.F.  & Interés ganado por el S.F. \\ $0$      & -       & -   & -   & -   \\ $1$      & $L_j$     & $D_1$            & $D_1=M_1$        & -    \\
$2$      & $L_j$     & $D_2$  & $D_1(1+\hat{j}) + D_2 = M_2$        & $M_1\hat{j}$     \\
$\vdots$ & $\vdots$    & $\vdots$ & $\vdots$   & $\vdots$ \\ $k$  & $L_j$ & $D_k$    & $M_{k-1}(1+\hat{j})+ D_k = M_k$     & $M_{k-1}\hat{j}$   \\ $\vdots$ & $\vdots$   & $\vdots$ & $\vdots$    & $\vdots$  \\ $n$ & $L_j$ & $D_n$  & $M_{n-1}(1+\hat{j}) + D_n = M_n =L$ & $M_{n-1}\hat{j}$  
\end{tabular}
\end{table}

\textit{Outstanding Balance}.
\begin{table}[H]
\begin{tabular}{cc}
$0$ & $L = OB_0$ \\
$1$     & $L-D_1=L-M_1 = OB_1$ \\
$2$     & $L-M_2=OB_2$ \\
$\vdots$ & \\
$k$     & $L-M_k=OB_k$ \\
$\vdots$ & \\
$n$     & $L-M_n=OB_n$ 
\end{tabular}
\end{table}

\textbf{Notación.}\\
$M_k$: Saldo en el SF, justo después del $k$-ésimo depósito\\
$D_k$: El depósito al tiempo $k$\\
$OB_k$: Outstanding balance al tiempo $k$\\
$IS_k$: Interés ganado en el SF en el periodo $k$
$$IS_k=M_{k-1}\hat{j}$$

\underline{\textit{Observación}}\\
\begin{align*}
    M_k &= M_{k-1}(1+\hat{j}) + D_k\\
    &= M_{k-2}(1+j\hat{j})^2 + D_{k-1}(1+\hat{j}) + D_k\\
    &= [M_{k-3}(1+\hat{j}) + D_{k-2}](1+\hat{j})^2 + D_{k-1}(1+\hat{j}) + D_k \\
    &= M_{k-3}(1+\hat{j})^3 + D_{k-2}(1+\hat{j})^2 + D_{k-1}(1+\hat{j}) + D_k\\
    M_k &= D_1(1+\hat{j})^{k-1} + D_2 (1+\hat{j})^{k-2} + ... + D_{k-1} (1+\hat{j}) + D_k
\end{align*}

\underline{En un caso particular:}\\
Si los pagos $D_k=D$ son todos iguales.\\

\circled{1} Se observa que 
$$M_k = D(1+\hat{j})^{k-1} + D(1+\hat{j})^{k-2} + ... + D(1+\hat{j}) + D = D\cdot S\angl{k}\hat{j} $$

\circled{2} E igualmente se ve que $L=DS\angl{n}\hat{j}$\\

\circled{3} De la misma forma
$$IS_k = M_{k-1}\cdot \hat{j} = DS\angl{k-1}\hat{j}\cdot\hat{j} = D\bigg( \frac{(1+\hat{j})^{k-1} -1}{\hat{j}}\bigg)\hat{j} = D\Big[(1+\hat{j})^{k-1} -1\Big]$$

\circled{4} Recordando la amortización ''nivelada'' vemos:\\
\begin{align*}
    \begin{cases}
    L=R\ax{\angln j} & L=DS\angln\hat{j}\\
    R=\frac{L}{\ax{\angln j}} & D= \frac{L}{S\angl{n}\hat{j}}
    \end{cases}
\end{align*}
Desembolso en $k$ del deudor por SF $D+L_j$

\textit{Recordatorio}
$$\frac{1}{\ax{\angln i}} = \frac{1}{S\angln i} + i$$

\subsection*{Caso nivelado}
\begin{itemize}
    \item Amortización. Desembolso periódico
    $$R=\frac{L}{\ax{\angln j}}$$
    \item Sinking Fund
    $$D + L_j = \frac{L}{S\angln\hat{j}} + L_j$$
\end{itemize}

¿Qué pasa si $j=\hat{j}$?\\
\textit{i.e.} la tasa de interés de la deuda es la misma que la del SF?

Intuitivamente el interés me de ganas el SF, es el mismo que estoy pagando sobre la deuda:\\
De hecho se ve el desembolso en:
$$SF = L \Big(\frac{1}{S\angln \hat{j}} + j\Big) = L\Big(\frac{1}{S\angln j} + j\Big) = L\Big(\frac{1}{\ax{\angln j}}\Big) = \frac{L}{\ax{\angln j}} = \textit{ Desembolso de amortización}$$

\textbf{Notación.} \\
$$\frac{1}{\ax{\angln j\&\hat{j}}} = \frac{1}{S\angln\hat{j}} + j \text{ con esta notación el desembolso total bajo el método de SF es:}$$
$$L\Big(\frac{1}{S\angln\hat{j}}+j\Big) = L\Big(\frac{1}{\ax{\angln j\&\hat{j}}}\Big) = \frac{L}{\ax{\angln j\&\hat{j}}}$$

Abusando de la notación anterior, tenemos que
\begin{align*}
    \frac{1}{\ax{\angln j\&\hat{j}}} &= \frac{1}{S\angln\hat{j}} + j = \Big(\frac{1}{\ax{\angln\hat{j}}} - \hat{j}\Big) +j \\
    &= \frac{1}{\ax{\angln\hat{j}}} + (j- \hat{j}) = \frac{1}{S\angln\hat{j}} + \hat{j} = \frac{1}{\ax{\angln\hat{j}}} \\
    &\Rightarrow \frac{1}{S\angln\hat{j}} = \frac{1}{\ax{\angln\hat{j}}} - \hat{j}
\end{align*}
Así,
\begin{align*}
    \frac{1}{\ax{\angln j\&\hat{j}}} &= \frac{1}{\ax{\angln\hat{j}}} - \hat{j} + j\\
    & \Rightarrow \ax{\angln j\&\hat{j}} = \frac{1}{\frac{1}{\ax{\angln\hat{j}}} - \hat{j} + j} = \frac{1}{\frac{1+(j-\hat{j}\ax{\angln \hat{j}}}{\ax{\angln\hat{j}}}} = \frac{\ax{\angln\hat{j}}}{(j-\hat{j})\ax{\angln\hat{j}} + 1}\\
    &\therefore \ax{\angln j\&\hat{j}} = \frac{\ax{\angln\hat{j}}}{(j-\hat{j})\ax{\angln\hat{j}} + 1}
\end{align*}

\section*{Análisis de Proyectos de inversión}
\begin{definition}
Un proyecto de inversión es un vector en $\mathbb{R}^{n+1}$, $\underline{C} = (C_0,C_1, ..., C_n$ \underline{asociado} a los tiempos $0, t_1, ..., t_n)\in\mathbb{R}^{n+1}$ donde: \textit{(suponiendo que los valores son conocidos)}
\begin{itemize}
    \item $C_k$: $A_k - L_k$, $k=0,1,2,...,n$
    \item $A_k$: Activos al tiempo $t_k$ $A_k\geq 0$
    \item $L_k$: Pasivos al tiempo $t_k$ $L_k\geq 0$
\end{itemize}
\end{definition}

\begin{center}
    

\tikzset{every picture/.style={line width=0.75pt}} %set default line width to 0.75pt        

\begin{tikzpicture}[x=0.75pt,y=0.75pt,yscale=-1,xscale=1]
%uncomment if require: \path (0,300); %set diagram left start at 0, and has height of 300

%Straight Lines [id:da35750079158205095] 
\draw    (8,150.9) -- (274,150.9) ;
%Straight Lines [id:da4504851993482528] 
\draw    (57.83,146.23) -- (57.83,156.4) ;
%Straight Lines [id:da08842937994220501] 
\draw    (217.33,146.07) -- (217.33,156.23) ;
%Straight Lines [id:da8934425975273562] 
\draw    (16.83,146.23) -- (16.83,156.4) ;
%Straight Lines [id:da20941080688586933] 
\draw    (97.83,146.23) -- (97.83,156.4) ;
%Straight Lines [id:da039091682936601035] 
\draw    (157.83,146.23) -- (157.83,156.4) ;
%Straight Lines [id:da6491299121844407] 
\draw    (265.83,146.07) -- (265.83,156.23) ;
%Curve Lines [id:da6048859255468285] 
\draw    (289,148) .. controls (328.6,118.3) and (313.31,175.24) .. (351.82,147.27) ;
\draw [shift={(353,146.4)}, rotate = 143.13] [color={rgb, 255:red, 0; green, 0; blue, 0 }  ][line width=0.75]    (10.93,-3.29) .. controls (6.95,-1.4) and (3.31,-0.3) .. (0,0) .. controls (3.31,0.3) and (6.95,1.4) .. (10.93,3.29)   ;
%Straight Lines [id:da7017111993629034] 
\draw    (366.1,150.9) -- (526,150.9) ;
%Straight Lines [id:da13047854427245875] 
\draw    (415.93,146.23) -- (415.93,156.4) ;
%Straight Lines [id:da731467047061797] 
\draw    (374.93,146.23) -- (374.93,156.4) ;
%Straight Lines [id:da8887539714752437] 
\draw    (455.93,146.23) -- (455.93,156.4) ;
%Straight Lines [id:da09220224193043425] 
\draw    (515.93,146.23) -- (515.93,156.4) ;

% Text Node
\draw (117.6,156.5) node [anchor=north west][inner sep=0.75pt]   [align=left] {. . .};
% Text Node
\draw (11.9,160.7) node [anchor=north west][inner sep=0.75pt]  [font=\small]  {$0$};
% Text Node
\draw (52.4,160.7) node [anchor=north west][inner sep=0.75pt]  [font=\small]  {$t_{1}$};
% Text Node
\draw (93.4,160.7) node [anchor=north west][inner sep=0.75pt]  [font=\small]  {$t_{2}$};
% Text Node
\draw (150.8,160.3) node [anchor=north west][inner sep=0.75pt]  [font=\small]  {$t_{k}$};
% Text Node
\draw (210.8,159.8) node [anchor=north west][inner sep=0.75pt]  [font=\small]  {$t_{n-1}$};
% Text Node
\draw (257.8,159.3) node [anchor=north west][inner sep=0.75pt]  [font=\small]  {$t_{n}$};
% Text Node
\draw (176.1,158) node [anchor=north west][inner sep=0.75pt]   [align=left] {. . .};
% Text Node
\draw (49.9,128.2) node [anchor=north west][inner sep=0.75pt]  [font=\small]  {$A_{1}$};
% Text Node
\draw (90.4,128.2) node [anchor=north west][inner sep=0.75pt]  [font=\small]  {$A_{2}$};
% Text Node
\draw (150.9,127.7) node [anchor=north west][inner sep=0.75pt]  [font=\small]  {$A_{k}$};
% Text Node
\draw (207.4,128.7) node [anchor=north west][inner sep=0.75pt]  [font=\small]  {$A_{n-1}$};
% Text Node
\draw (257.9,128.7) node [anchor=north west][inner sep=0.75pt]  [font=\small]  {$A_{n}$};
% Text Node
\draw (7.9,129.2) node [anchor=north west][inner sep=0.75pt]  [font=\small]  {$A_{0}$};
% Text Node
\draw (475.7,156.5) node [anchor=north west][inner sep=0.75pt]   [align=left] {. . .};
% Text Node
\draw (370,160.7) node [anchor=north west][inner sep=0.75pt]  [font=\small]  {$0$};
% Text Node
\draw (410.5,160.7) node [anchor=north west][inner sep=0.75pt]  [font=\small]  {$t_{1}$};
% Text Node
\draw (451.5,160.7) node [anchor=north west][inner sep=0.75pt]  [font=\small]  {$t_{2}$};
% Text Node
\draw (508.9,159.3) node [anchor=north west][inner sep=0.75pt]  [font=\small]  {$t_{n}$};
% Text Node
\draw (408,128.2) node [anchor=north west][inner sep=0.75pt]  [font=\small]  {$C_{1}$};
% Text Node
\draw (448.5,128.2) node [anchor=north west][inner sep=0.75pt]  [font=\small]  {$C_{2}$};
% Text Node
\draw (508,128.7) node [anchor=north west][inner sep=0.75pt]  [font=\small]  {$C_{n}$};
% Text Node
\draw (366,129.2) node [anchor=north west][inner sep=0.75pt]  [font=\small]  {$C_{0}$};
% Text Node
\draw (48,181.2) node [anchor=north west][inner sep=0.75pt]  [font=\small]  {$L_{1}$};
% Text Node
\draw (88.5,181.2) node [anchor=north west][inner sep=0.75pt]  [font=\small]  {$L_{2}$};
% Text Node
\draw (149,180.7) node [anchor=north west][inner sep=0.75pt]  [font=\small]  {$L_{k}$};
% Text Node
\draw (205.5,181.7) node [anchor=north west][inner sep=0.75pt]  [font=\small]  {$L_{n-1}$};
% Text Node
\draw (256,181.7) node [anchor=north west][inner sep=0.75pt]  [font=\small]  {$L_{n}$};
% Text Node
\draw (6,182.2) node [anchor=north west][inner sep=0.75pt]  [font=\small]  {$L_{0}$};
% Text Node
\draw (274,91) node [anchor=north west][inner sep=0.75pt]   [align=left] {\begin{minipage}[lt]{46.24pt}\setlength\topsep{0pt}
\begin{center}
{\scriptsize Son pagos}\\{\scriptsize deterministas}
\end{center}

\end{minipage}};


\end{tikzpicture}

\end{center}

\textit{\underline{Observación}}. Los $t_k's$ no son necesariamente equistantes. ¿Qué nos interesa de un proyecto de inversión?

\circled{1} ¿Hay una ganancia efectiva? \textit{i.e.} ¿Gano o pierdo?

\circled{2} Dados dos proyectos, ¿cuál es el \underline{''MEJOR''}?\\

Mejor:\\
$\quad$2.1: Mejor:más ganancia\\
$\quad$2,2: Mejor: recupera mi inversión más rápido\\
$\quad$2.3: Mejor: Que la inversión inicial sea menor\\
$\quad$2.4; Mejor: ¿Gano más que depositando en un banco?

La hipótesis de no arbitral = oferta y demanda.

\begin{definition}
Para un proyecto \underline{$C$} se define la función valor presente como $\Psi_{\underline{C}}: [-1,\infty)\to\mathbb{R}$ dado por:
\begin{align*}
    \Psi_{\underline{C}}(i) &= \sum_{k=0}^n C_k(1+i)^{-t_k}\\
    \Psi_{A}(i) &= \sum_{k=0}^n A_k(1+i)^{-t_k}\\
    \Psi_{L}(i) &= \sum_{k=0}^n L_k(1+i)^{-t_k}
\end{align*}
\end{definition}

\underline{\textit{Observación}}. 
\begin{itemize}
    \item[i)] $i\to\Psi_A$ es no-creciente pues
    $$\frac{\delta}{\delta_i}\Psi_A(i) = \frac{\delta}{\delta_i} \sum_{k=0}^n A_k(1+i)^{-t_k} = \sum_{k=0}^n \frac{\delta}{\delta_i} A_k(1+i)^{-t_k} = \sum_{k=1}^n -t_k A_k(1+i)^{-t_k} < 0$$
    Así $\frac{\delta}{\delta_i}\Psi_A(i) < 0 \Rightarrow \Psi_A$ es estrictamente decreciente
    \item[ii)] $i\to\Psi_L(i)$ es no decreciente por la misma razón
    \item[iii)] $i\to\Psi_{\underline{C}}(i)$ no sabemos los criterios
\end{itemize}

\begin{definition}
Se dice que el proyecto $\underline{C}$ es rentable si $\exists i^*\in(-1,\infty)$ tal que $\Psi_{\underline{C}} (i^*)>0$ (\textit{i.e.} su valor presente es positivo).
Equivalentemente $\exists i^*\in(-1,\infty)$ tal que $\Psi_A(i^*) > \Psi_L(i^*)$
\end{definition}
\begin{center}
    

\tikzset{every picture/.style={line width=0.75pt}} %set default line width to 0.75pt        

\begin{tikzpicture}[x=0.75pt,y=0.75pt,yscale=-1,xscale=1]
%uncomment if require: \path (0,300); %set diagram left start at 0, and has height of 300

%Shape: Axis 2D [id:dp847504424215119] 
\draw  (150,203.72) -- (307,203.72)(164.21,96.4) -- (164.21,215.4) (300,198.72) -- (307,203.72) -- (300,208.72) (159.21,103.4) -- (164.21,96.4) -- (169.21,103.4)  ;
%Curve Lines [id:da6455330390669454] 
\draw    (164,161.4) .. controls (201,281.4) and (197,211.4) .. (226,136.4) ;
%Curve Lines [id:da8514432273859649] 
\draw [color={rgb, 255:red, 0; green, 0; blue, 0 }  ,draw opacity=0.92 ]   (226,136.4) .. controls (236,118.4) and (246,150.4) .. (249,161.4) ;
%Curve Lines [id:da42408835899900654] 
\draw [color={rgb, 255:red, 0; green, 0; blue, 0 }  ,draw opacity=0.92 ]   (141,136.4) .. controls (151,118.4) and (158,149.3) .. (164,161.4) ;
%Curve Lines [id:da6579230623370902] 
\draw    (249,161.4) .. controls (278,275.4) and (281,220.4) .. (304,144.4) ;

% Text Node
\draw (152,204.4) node [anchor=north west][inner sep=0.75pt]  [font=\small]  {$0$};
% Text Node
\draw (301,208.4) node [anchor=north west][inner sep=0.75pt]  [font=\small]  {$i$};
% Text Node
\draw (153,80.4) node [anchor=north west][inner sep=0.75pt]  [font=\small]  {$\Psi_{\underline{C}}$};


\end{tikzpicture}

\end{center}

\begin{definition}
Sean $\underline{C}$ y $\underline{\hat{C}}$ dos proyectos de inversión. Se dice que $\underline{C}$ es mejor que $\underline{\hat{C}}$ en valor presente a la tasa $i^*$ si $\Psi_{\underline{C}}(i^*) \geq \Psi_{\underline{\hat{C}}} (i^*)$ denotado por
$$\underline{C} \succeq \underline{\hat{C}}$$
\end{definition}
\begin{center}
\tikzset{every picture/.style={line width=0.75pt}} %set default line width to 0.75pt        

\begin{tikzpicture}[x=0.75pt,y=0.75pt,yscale=-1,xscale=1]
%uncomment if require: \path (0,300); %set diagram left start at 0, and has height of 300

%Shape: Axis 2D [id:dp9522468765174452] 
\draw  (146,208.51) -- (367,208.51)(169,70.5) -- (169,222.4) (360,203.51) -- (367,208.51) -- (360,213.51) (164,77.5) -- (169,70.5) -- (174,77.5)  ;
%Curve Lines [id:da06317747888029013] 
\draw [color={rgb, 255:red, 0; green, 0; blue, 0 }  ,draw opacity=0.92 ]   (159,129.4) .. controls (182,108.9) and (196,132.4) .. (202,138.4) ;
%Curve Lines [id:da3356942508994254] 
\draw    (202,138.4) .. controls (249,241.4) and (253,197.4) .. (295,134.4) ;
%Curve Lines [id:da36837495479157134] 
\draw    (188,137.4) .. controls (231,397.4) and (232,86.4) .. (262,114.4) ;
%Curve Lines [id:da5637884937561276] 
\draw    (262,114.4) .. controls (300,240.4) and (300,323.4) .. (329,117.4) ;
%Curve Lines [id:da43192884110125307] 
\draw [color={rgb, 255:red, 0; green, 0; blue, 0 }  ,draw opacity=0.92 ]   (295,134.4) .. controls (318,113.9) and (333,134.8) .. (339,140.8) ;
%Curve Lines [id:da33277351068432426] 
\draw [color={rgb, 255:red, 0; green, 0; blue, 0 }  ,draw opacity=0.92 ]   (157,120.4) .. controls (173,95.4) and (181,116.4) .. (188,137.4) ;

% Text Node
\draw (327,97.4) node [anchor=north west][inner sep=0.75pt]  [font=\small]  {$\Psi_{\underline{C}}$};
% Text Node
\draw (344,137.4) node [anchor=north west][inner sep=0.75pt]  [font=\small]  {$\Psi_{\widehat{\underline{C}}}$};
\end{tikzpicture}
\end{center}

\begin{definition}
Sean $\underline{C}$ un proyecto de inversión. Se define una tasa interna de rendimiento como $TIR\in\mathbb{R}$ tal que $\Psi_{\underline{C}} (TIR)=0$, equivalentemente 
$$\Psi_A(TIR)=\Psi_L(TIR)$$
\textit{A veces $TIR$ puede no existir y si existe, no es único.}
\end{definition}

\begin{center}
    

\tikzset{every picture/.style={line width=0.75pt}} %set default line width to 0.75pt        

\begin{tikzpicture}[x=0.75pt,y=0.75pt,yscale=-1,xscale=1]
%uncomment if require: \path (0,300); %set diagram left start at 0, and has height of 300

%Shape: Axis 2D [id:dp9522468765174452] 
\draw  (146,187.61) -- (367,187.61)(169,70.5) -- (169,199.4) (360,182.61) -- (367,187.61) -- (360,192.61) (164,77.5) -- (169,70.5) -- (174,77.5)  ;
%Curve Lines [id:da06317747888029013] 
\draw [color={rgb, 255:red, 0; green, 0; blue, 0 }  ,draw opacity=0.92 ]   (159,129.4) .. controls (182,108.9) and (196,132.4) .. (202,138.4) ;
%Curve Lines [id:da3356942508994254] 
\draw    (202,138.4) .. controls (249,241.4) and (253,197.4) .. (295,134.4) ;
%Curve Lines [id:da43192884110125307] 
\draw [color={rgb, 255:red, 0; green, 0; blue, 0 }  ,draw opacity=0.92 ]   (295,134.4) .. controls (318,113.9) and (333,134.8) .. (339,140.8) ;
%Shape: Ellipse [id:dp7893274056052362] 
\draw  [color={rgb, 255:red, 208; green, 2; blue, 27 }  ,draw opacity=1 ] (217.99,186.91) .. controls (218.08,184.07) and (222.38,181.91) .. (227.58,182.08) .. controls (232.78,182.26) and (236.92,184.7) .. (236.83,187.55) .. controls (236.73,190.39) and (232.44,192.55) .. (227.23,192.38) .. controls (222.03,192.2) and (217.89,189.76) .. (217.99,186.91) -- cycle ;
%Shape: Ellipse [id:dp02607015406839175] 
\draw  [color={rgb, 255:red, 208; green, 2; blue, 27 }  ,draw opacity=1 ] (252.99,186.91) .. controls (253.08,184.07) and (257.38,181.91) .. (262.58,182.08) .. controls (267.78,182.26) and (271.92,184.7) .. (271.83,187.55) .. controls (271.73,190.39) and (267.44,192.55) .. (262.23,192.38) .. controls (257.03,192.2) and (252.89,189.76) .. (252.99,186.91) -- cycle ;
%Curve Lines [id:da3376885015560015] 
\draw    (217.99,186.91) .. controls (193.56,193.01) and (197.4,196.03) .. (198.77,205.5) ;
\draw [shift={(199,207.4)}, rotate = 264.81] [color={rgb, 255:red, 0; green, 0; blue, 0 }  ][line width=0.75]    (10.93,-3.29) .. controls (6.95,-1.4) and (3.31,-0.3) .. (0,0) .. controls (3.31,0.3) and (6.95,1.4) .. (10.93,3.29)   ;
%Curve Lines [id:da8706765760871811] 
\draw    (271.83,187.55) .. controls (292.26,192.23) and (292.97,192.39) .. (288.51,209.47) ;
\draw [shift={(288,211.4)}, rotate = 284.74] [color={rgb, 255:red, 0; green, 0; blue, 0 }  ][line width=0.75]    (10.93,-3.29) .. controls (6.95,-1.4) and (3.31,-0.3) .. (0,0) .. controls (3.31,0.3) and (6.95,1.4) .. (10.93,3.29)   ;

% Text Node
\draw (341,144.2) node [anchor=north west][inner sep=0.75pt]  [font=\small]  {$\Psi_{\underline{C}}( i)$};
% Text Node
\draw (190,209.4) node [anchor=north west][inner sep=0.75pt]  [font=\scriptsize]  {$TIR_{1}$};
% Text Node
\draw (274,213.4) node [anchor=north west][inner sep=0.75pt]  [font=\scriptsize]  {$TIR_{2}$};


\end{tikzpicture}

\end{center}

\end{document}