
\documentclass[
letterpaper,
11pt, % Cambiar a 10 si es que no cabe
oneside,
onecolumn, %twocolumn para dos columnas
article
]{memoir}

\usepackage[spanish,es-nodecimaldot]{babel}
\usepackage[utf8]{inputenc}
\usepackage[T1]{fontenc}
\usepackage{tgtermes} % La fuente a usar, si no compila quitar esta línea
\usepackage[svgnames]{xcolor} % Required for colour specification
\usepackage{blindtext} % Controls the indentation and the space between paragraphs
\usepackage{tikzsymbols} % Emojis
\usepackage{tikz} %Grphics 
\usepackage{empheq} % Hace la hoja tamaño carta
\usetikzlibrary{snakes,positioning, decorations.pathreplacing,decorations.markings,babel} % Diagrams
\usepackage{rotating} % Diagrams
\usepackage{pifont} % Figuras para referenciar
\usepackage{cancel} % To draw diagonal lines through expressions
\usepackage{tabularx} % Tables
\usepackage{multicol} % Multiple columns
\usepackage{enumitem} % Enumerate with diferent bullets
\usepackage{ulem} % Underline fixing code errors of normal \underline{•}
\usepackage{color,soul} % Underline with colors
\medievalpage

% Paquetes para matemáticas
\usepackage{amscd}
\usepackage{amsfonts}
\usepackage{amssymb}
\usepackage{amsmath}
\usepackage{amsthm}
\usepackage{latexsym}
\usepackage{mathrsfs}
\usepackage{bm}
\usepackage{bbm}
\usepackage{mathtools}
\usepackage{listings}
\usepackage[spanish,onelanguage,ruled,linesnumbered]{algorithm2e}
\usepackage{stackengine}
\usepackage[mathscr]{euscript}
\usepackage[scr]{rsfso}
\usepackage{empheq}
\usepackage[final]{microtype}
\usepackage{graphicx} % Para incluir figuras
\usepackage{lipsum}
\usepackage{actuarialsymbol} %Actuarial notation
\usepackage{hyperref}

% Command "alignedbox{}{}" for a box within an align environment
% Source: http://www.latex-community.org/forum/viewtopic.php?f=46&t=8144
\newlength\dlf  % Define a new measure, dlf
\newcommand\alignedbox[2]{
% Argument #1 = before & if there were no box (lhs)
% Argument #2 = after & if there were no box (rhs)
&  % Alignment sign of the line
{
\settowidth\dlf{$\displaystyle #1$}  
    % The width of \dlf is the width of the lhs, with a displaystyle font
\addtolength\dlf{\fboxsep+\fboxrule}  
    % Add to it the distance to the box, and the width of the line of the box
\hspace{-\dlf}  
    % Move everything dlf units to the left, so that & #1 #2 is aligned under #1 & #2
\boxed{#1 #2}
    % Put a box around lhs and rhs
}
}

\setlrmarginsandblock{0.15\paperwidth}{*}{1} % Para onecolumn
\setulmarginsandblock{0.5in}{1.5in}{1}  % Márgenes superior e inferior
\checkandfixthelayout

\parindent=0pt % disables indentation
\parskip=12pt % adds vertical space between paragraphs

\addto{\captionsspanish}{%
  \renewcommand{\bibname}{\Large Referencias}
}

\counterwithout{section}{chapter}
\counterwithout{figure}{chapter}

\makepagestyle{plain}
\makeevenfoot{plain}{\thepage}{}{}
\makeoddfoot{plain}{}{}{\thepage}
\makeevenhead{plain}{}{}{}
\makeoddhead{plain}{}{}{}

\makeatletter %
\makechapterstyle{standard}{
  \setlength{\beforechapskip}{2\baselineskip}
  \setlength{\midchapskip}{0\baselineskip}
  \setlength{\afterchapskip}{2\baselineskip}
  \renewcommand{\chapterheadstart}{\vspace*{\beforechapskip}}
  \renewcommand{\chapnamefont}{\normalfont\Large}
  \renewcommand{\printchaptername}{}
  \renewcommand{\chapternamenum}{\space}
  \renewcommand{\chapnumfont}{\normalfont\Large}
  %\renewcommand{\printchapternum}{\chapnumfont \thechapter.}
  %\renewcommand{\afterchapternum}{\par\nobreak\vskip \midchapskip}
  \renewcommand{\afterchapternum}{ }
  \renewcommand{\printchapternonum}{\vspace*{\midchapskip}\vspace*{5mm}}
  \renewcommand{\chaptitlefont}{\bfseries\LARGE}
  \renewcommand{\printchaptertitle}[1]{\chaptitlefont ##1}
  \renewcommand{\afterchaptertitle}{\par\nobreak\vskip \afterchapskip}
}
\makeatother

\chapterstyle{standard}

\makeatletter %
\makechapterstyle{appendix}{
  \setlength{\beforechapskip}{2\baselineskip}
  \setlength{\midchapskip}{0\baselineskip}
  \setlength{\afterchapskip}{2\baselineskip}
  \renewcommand{\chapterheadstart}{\vspace*{\beforechapskip}}
  \renewcommand{\chapnamefont}{\normalfont\Large}
  \renewcommand{\printchaptername}{\chapnamefont \@chapapp}
  \renewcommand{\chapternamenum}{\space}
  \renewcommand{\chapnumfont}{\normalfont\Large}
  \renewcommand{\printchapternum}{\chapnumfont \thechapter.}
  %\renewcommand{\afterchapternum}{\par\nobreak\vskip \midchapskip}
  \renewcommand{\afterchapternum}{ }
  \renewcommand{\printchapternonum}{\vspace*{\midchapskip}\vspace*{5mm}}
  \renewcommand{\chaptitlefont}{\bfseries\LARGE}
  \renewcommand{\printchaptertitle}[1]{\chaptitlefont ##1}
  \renewcommand{\afterchaptertitle}{\par\nobreak\vskip \afterchapskip}
}
\makeatother

\setlength{\columnseprule}{1pt} %Line between paragraphs

\tikzset{
  % style to apply some styles to each segment of a path
  on each segment/.style={
    decorate,
    decoration={
      show path construction,
      moveto code={},
      lineto code={
        \path [#1]
        (\tikzinputsegmentfirst) -- (\tikzinputsegmentlast);
      },
      curveto code={
        \path [#1] (\tikzinputsegmentfirst)
        .. controls
        (\tikzinputsegmentsupporta) and (\tikzinputsegmentsupportb)
        ..
        (\tikzinputsegmentlast);
      },
      closepath code={
        \path [#1]
        (\tikzinputsegmentfirst) -- (\tikzinputsegmentlast);
      },
    },
  },
  % style to add an arrow in the middle of a path
  end arrow/.style={postaction={decorate,decoration={
        markings,
        mark=at position 0.999 with {\arrow[#1]{stealth}}
      }}},
} % Curved lines

% Declaración de comandos y operadores
\newcommand\RR{\mathbb R}
\newcommand\NN{\mathbb N}
\newcommand\PP{\mathbb P}
\newcommand\dpartial[1]{\frac{\partial}{\partial #1}}
\newcommand\deriv[1]{\frac{d}{d #1}}
\newcommand\integral[4]{\int_{#1}^{#2} #3 \, d#4}
\newcommand*\circled[1]{\tikz[baseline=(char.base)]{
            \node[shape=circle,draw,inner sep=2pt] (char) {#1};}}
\DeclareMathOperator\Ber{Bernoulli}

% Se definen los comandos para escribir teoremas, definiciones y demás.
\theoremstyle{plain}
\newtheorem*{theorem}{Teorema}
\newtheorem*{corollary}{Corolario}
\newtheorem*{lemma}{Lema}
\newtheorem*{proposition}{Proposici\'on}
\theoremstyle{definition}
\newtheorem*{definition}{Definici\'on}
\theoremstyle{remark}
\newtheorem*{remark}{Observaci\'on}

\begin{document}

%%%%%%%%%%%%%%%%%%%%%%%%%
% Aquí va la portada
%%%%%%%%%%%%%%%%%%%%%%%%%

\begin{titlingpage} % Portada

    \raggedleft % Alineada a la derecha
    %\raggedright % Alineada a la izquierda
	
	\vspace*{\baselineskip} % Whitespace at the top of the page
	
	\vspace*{0.25\textheight} % Whitespace before the title
	
	%------------------------------------------------
	%	Cosas del título
	%------------------------------------------------
    
    \vspace*{0.1\textheight}

    {\Huge{\textbf{Mercados Financieros y Valuación\\ de Proyectos}}}\\[\baselineskip] % Aquí va el título
    \vspace*{0.1\textheight}

    %------------------------------------------------
	%	Aquí van los nombres
	%------------------------------------------------
    
    {\Large Eduardo Selim Matínez Mayorga}\\[\baselineskip]
	
	\vfill

\end{titlingpage}

\thispagestyle{empty}

\chapter*{Análisis de Proyectos de inversión}


\begin{definition} Un proyecto de inversión es un vector $\underline{c} = (c_0, c_1, ... , c_n)\in \mathbb{R}^{n+1}$ asociado a otro vector $\underline{t} = (t_0, t_1, ... , t_n)\in \mathbb{R}^{n+1}$, $n\in\mathbb{N}\cup\{\infty\}$ en el que $c_k :=A_k-B_k$.\\
donde:
\begin{itemize}
    \item $A_k$ representa una entrada de efectivo al tiempo $t_k$
    \item $B_k$ representa una salida de efectivo al tiempo $t_k$
\end{itemize}
\textit{i.e.} $c_k$ es el "ingreso neto" al tiempo $t_k$.
\end{definition}

\textit{Obs.} $c_k$ puede ser positivo, negativo o cero
\begin{itemize}
    \item $\underline{c}$ representa los pagos
    \item $\underline{t}$ representa los tiempos
\end{itemize}

A un proyecto de inversión se le denota como $(\underline{c},\underline{t})$, es decir, no se puede hablar de un proyecto de inversión sin hablar del flujo de efectivo y los tiempos en los que ocurre cada pago.

Ejemplos:
\begin{itemize}
    \item Bono cuponado:\\
    $\underline{c} = (-P,-F_r,F_r,...,F_{r+c})$\\
    $\underline{t} = (0,1,2,...,n)$
    \item Depósito bancario\\
    $\underline{c} = (-k,M)$\\
    $\underline{t} = (0,t^*)$
    \item Anualidad nivelada\\
    $\underline{c} = (-Ra_{{m}i},-F_r,F_r,...,F_{r+c})$\\
    $\underline{t} = (0,1,2,...,n)$
\end{itemize}

Dados 2 proytectos $(\underline{c}_I,\underline{t}_I)$ y $(\underline{c}_II,\underline{t}_II)$\\
t\textit{¿Cuál es mejor?} No se puede hablar de un "mejor" proyecto en general. Siempre el mejor proyecto es dado a algún criterio elección como:

\begin{itemize}
    \item ¿Con cuál recupero más rápido mi inversión?
    \item ¿Si necesito mucho flujo?
    \item ¿Con cuál gano más?
    \item ...
\end{itemize}

\textbf{Ejemplo.} Considerese los siguientes proyectos
%Falta tabla con los proyectos

Comparación dependiendo:\\
1. En cuanto tiempo recupero mi inversión \\
I: En 4 periodos ya recuperé la inversión inicial\\
II: En 5 periodos ya recuperé mi inversión inicial
$$\therefore I\geq_{PR} II$$
\\

1) Netamente cuánto dinero te da cada proyecto\\
\begin{center}
    

\tikzset{every picture/.style={line width=0.75pt}} %set default line width to 0.75pt        

\begin{tikzpicture}[x=0.75pt,y=0.75pt,yscale=-1,xscale=1]
%uncomment if require: \path (0,235); %set diagram left start at 0, and has height of 235

%Straight Lines [id:da8186857307094206] 
\draw    (92,141) -- (172.2,140.75) ;
%Straight Lines [id:da2988911929087691] 
\draw    (202,91) -- (282.2,90.75) ;

\draw (126,22) node [anchor=north west][inner sep=0.75pt]   [align=left] {I};
% Text Node
\draw (236,22) node [anchor=north west][inner sep=0.75pt]   [align=left] {II};
% Text Node
\draw (110,52) node [anchor=north west][inner sep=0.75pt]   [align=left] {250(2)};
% Text Node
\draw (221,50) node [anchor=north west][inner sep=0.75pt]   [align=left] {180(8)};
% Text Node
\draw (106,77) node [anchor=north west][inner sep=0.75pt]   [align=left] {+280(2)};
% Text Node
\draw (220,75) node [anchor=north west][inner sep=0.75pt]   [align=left] {\mbox{-} 900};
% Text Node
\draw (108,100) node [anchor=north west][inner sep=0.75pt]   [align=left] {+100};
% Text Node
\draw (110,124) node [anchor=north west][inner sep=0.75pt]   [align=left] {\mbox{-}1000};
% Text Node
\draw (117,146) node [anchor=north west][inner sep=0.75pt]   [align=left] {160};
% Text Node
\draw (232,94) node [anchor=north west][inner sep=0.75pt]   [align=left] {540};


\end{tikzpicture}

\end{center}

\begin{center}
$$ \therefore \phantom{abc}I \leq_{NM} II $$ 
\end{center}


\begin{definition}
Un proyecto de inversión\\
$\underline{c} = (C_0,C_1,\cdots, C_n) \in \mathbb{R}^{n+1}\phantom{abc} n\in \mathbb{N}\cup\{0\} \\
\underline{t} = (0,t_1,\cdots, t_n) \in \mathbb{R}^{n+1}$\\ \\
No necesariamente\\
$\rightarrow |t_{j+1} - t_j | = 1 \\
\rightarrow |t_{j+1} - t_j | = |t_{j+2} - t_{j+1} |$\\ \\
i.e los tiempos no necesariamente son equidistantes. \\
* Dijimos que hay varios criterios para decir cuándo un proyecto es mejor que otro.\\
$\rightarrow$ Uno de ellos era cuanto tiempo en recuperar la inversión.\\ \\
\end{definition}
En el ejemplo. \\
El tiempo en el que $\sum$activos $> \sum$pasivos \\
\textbf{Ejemplo:}
\begin{center}
    

\tikzset{every picture/.style={line width=0.75pt}} %set default line width to 0.75pt        

\begin{tikzpicture}[x=0.75pt,y=0.75pt,yscale=-1,xscale=1]
%uncomment if require: \path (0,300); %set diagram left start at 0, and has height of 300


% Text Node
\draw (107.6,44) node [anchor=north west][inner sep=0.75pt]   [align=left] {0};
% Text Node
\draw (158.6,44) node [anchor=north west][inner sep=0.75pt]   [align=left] {1};
% Text Node
\draw (209.6,44) node [anchor=north west][inner sep=0.75pt]   [align=left] {2};
% Text Node
\draw (259.6,45) node [anchor=north west][inner sep=0.75pt]   [align=left] {3};
% Text Node
\draw (305.6,45) node [anchor=north west][inner sep=0.75pt]   [align=left] {4};
% Text Node
\draw (356.6,45) node [anchor=north west][inner sep=0.75pt]   [align=left] {5};
% Text Node
\draw (407.6,45) node [anchor=north west][inner sep=0.75pt]   [align=left] {6};
% Text Node
\draw (456.6,46) node [anchor=north west][inner sep=0.75pt]   [align=left] {7};
% Text Node
\draw (506.6,45) node [anchor=north west][inner sep=0.75pt]   [align=left] {8};
% Text Node
\draw (89.6,84) node [anchor=north west][inner sep=0.75pt]   [align=left] {\mbox{-} 1000};
% Text Node
\draw (150.6,84) node [anchor=north west][inner sep=0.75pt]   [align=left] {250};
% Text Node
\draw (247.6,84) node [anchor=north west][inner sep=0.75pt]   [align=left] {280};
% Text Node
\draw (200.6,83) node [anchor=north west][inner sep=0.75pt]   [align=left] {250};
% Text Node
\draw (298.6,84) node [anchor=north west][inner sep=0.75pt]   [align=left] {280};
% Text Node
\draw (347.6,83) node [anchor=north west][inner sep=0.75pt]   [align=left] {100};
% Text Node
\draw (407.6,84) node [anchor=north west][inner sep=0.75pt]   [align=left] {0};
% Text Node
\draw (460.6,84) node [anchor=north west][inner sep=0.75pt]   [align=left] {0};
% Text Node
\draw (506.6,83) node [anchor=north west][inner sep=0.75pt]   [align=left] {0};
% Text Node
\draw (89.6,136) node [anchor=north west][inner sep=0.75pt]   [align=left] {\mbox{-} 900};
% Text Node
\draw (149.6,135) node [anchor=north west][inner sep=0.75pt]   [align=left] {180};
% Text Node
\draw (200.6,134) node [anchor=north west][inner sep=0.75pt]   [align=left] {180};
% Text Node
\draw (250.6,136) node [anchor=north west][inner sep=0.75pt]   [align=left] {180};
% Text Node
\draw (300.6,135) node [anchor=north west][inner sep=0.75pt]   [align=left] {180};
% Text Node
\draw (350.6,134) node [anchor=north west][inner sep=0.75pt]   [align=left] {180};
% Text Node
\draw (399.6,135) node [anchor=north west][inner sep=0.75pt]   [align=left] {180};
% Text Node
\draw (450.6,134) node [anchor=north west][inner sep=0.75pt]   [align=left] {180};
% Text Node
\draw (496.6,134) node [anchor=north west][inner sep=0.75pt]   [align=left] {180};
% Text Node
\draw (56.6,84) node [anchor=north west][inner sep=0.75pt]   [align=left] {1};
% Text Node
\draw (55.6,134) node [anchor=north west][inner sep=0.75pt]   [align=left] {2};


\end{tikzpicture}

\end{center}
\begin{center}
    Periodo de recuperación 1 = 4 Periodos\\
    Periodo de recuperación 2 = 5 Periodos
\end{center}
Entonces\\
Sea $\underline{c} = (C_0,C_1,...,C_n)$ un proyecto de inversión tal que:
$$C_0,C1,...,C_k <0 ,C_k,...,C_n >0$$
Se define el periodo de recuperación:\\
    $C_k<0$ pasivo/ salidas / egresos \\
    $C_k>0$ activos / entradas / ingresos\\
 $$t_c = min\big\{l:\sum_{j=0}^{k}C_j\leq \sum_{j=k+1}^l C_i\big\}$$ \\
 Se define el periodo de recuperación de  $\underline{c}$ $t_{\underline{c}}$ como ``El más chico de los $l$ tal que pasa eso``.\\
 \textbf{Ejemplo:}

\begin{center}
\tikzset{every picture/.style={line width=0.75pt}} %set default line width to 0.75pt        

\begin{tikzpicture}[x=0.75pt,y=0.75pt,yscale=-1,xscale=1]
%uncomment if require: \path (0,300); %set diagram left start at 0, and has height of 300


% Text Node
\draw (126,154) node [anchor=north west][inner sep=0.75pt]   [align=left] {$\displaystyle \underline{c}$};
% Text Node
\draw (143,151) node [anchor=north west][inner sep=0.75pt]   [align=left] {= (-100,-150,-5,200,120,0,180,100)};
% Text Node
\draw (172,133) node [anchor=north west][inner sep=0.75pt]  [font=\scriptsize] [align=left] {0 \ \ \ \ \ \ \ \ \ 1 \ \ \ \ \ 2 \ \ \ \ 3 \ \ \ \ \ \ \ 4 \ \ \ \ 5 \ \ \ \ 6 \ \ \ \ \ 7};
\end{tikzpicture}
\end{center}
¿$t_{\underline{c}}$?\\
$\sum$ pasivos = -255\\
$t_{\underline{c}} = 4$, pues 200+120 $>$255\\
\textbf{Ejemplo:}
\begin{center}
    

\tikzset{every picture/.style={line width=0.75pt}} %set default line width to 0.75pt        

\begin{tikzpicture}[x=0.75pt,y=0.75pt,yscale=-1,xscale=1]
%uncomment if require: \path (0,300); %set diagram left start at 0, and has height of 300


% Text Node
\draw (126,154) node [anchor=north west][inner sep=0.75pt]   [align=left] {$\displaystyle \underline{c}$};
% Text Node
\draw (143,151) node [anchor=north west][inner sep=0.75pt]   [align=left] {= (-90,-90,-10,5,5,5,20,150,150,40,50)};
% Text Node
\draw (173,175) node [anchor=north west][inner sep=0.75pt]  [font=\scriptsize] [align=left] {0 \ \ \ \ 1 \ \ \ \ \ \ 2 \ \ 3 \ 4 \ 5 \ \ \ 6 \ \ \ \ \ 7 \ \ \ \ \ \ 8 \ \ \ \ \ \ 9 \ \ 10};


\end{tikzpicture}

\end{center}
$t_{\underline{c}} = 8$\\
$\sum$ pasivos = -190\\
$t_{\underline{c}} = 8 $ pues 5 + 5 + 5 + 20 + 150 + 150 $>$ 190 \\ \\
\textbf{Ejemplo:}
\begin{center}
    \tikzset{every picture/.style={line width=0.75pt}} %set default line width to 0.75pt        

\begin{tikzpicture}[x=0.75pt,y=0.75pt,yscale=-1,xscale=1]
%uncomment if require: \path (0,300); %set diagram left start at 0, and has height of 300


% Text Node
\draw (126,154) node [anchor=north west][inner sep=0.75pt]   [align=left] {$\displaystyle \underline{c}$};
% Text Node
\draw (143,151) node [anchor=north west][inner sep=0.75pt]   [align=left] {= (-100,-150,-5,200,120,0,180,100)};
% Text Node
\draw (172,133) node [anchor=north west][inner sep=0.75pt]  [font=\scriptsize] [align=left] {0 \ \ \ \ \ \ \ \ \ 1 \ \ \ \ \ 2 \ \ \ \ 3 \ \ \ \ \ \ \ 4 \ \ \ \ 5 \ \ \ \ 6 \ \ \ \ \ 7};
\end{tikzpicture}
\end{center}
¿$t_{\underline{c}}$?\\
$\sum$ pasivos = -255\\
$t_{\underline{c}} = 4$, pues 200+120 $>$255\\ \\
\textbf{Ejemplo:}
\begin{center}
    

\tikzset{every picture/.style={line width=0.75pt}} %set default line width to 0.75pt        

\begin{tikzpicture}[x=0.75pt,y=0.75pt,yscale=-1,xscale=1]
%uncomment if require: \path (0,300); %set diagram left start at 0, and has height of 300


% Text Node
\draw (126,154) node [anchor=north west][inner sep=0.75pt]   [align=left] {$\displaystyle \underline{c}$};
% Text Node
\draw (143,151) node [anchor=north west][inner sep=0.75pt]   [align=left] {= (-50,40,-30,20,20,-100,150) $\phantom{abc}\bigstar$};

\end{tikzpicture}
\end{center}
¿$t_{\underline{c}}$?\\ \\
No tiene sentido hablar de periodo de recuperación pues las salidas y entradas son itinerantes.\\ \\
Si queremos extender la definición de periodo de recuperación para proyectos en los que no necesariamente las salidas ocurran primero y posteriormente las entradas.¿Qué harían?\\ \\
Queremos definir

$t=\alpha_1 t_1 + \alpha_2 t_2 + ... + \alpha_n t_n $ \phantom{abc} con \phantom{a}$\alpha_1 + ... + \alpha_n =1$\\ \\
i.e $t$ es un promedio ponderado de los tiempos en los que se hace cada $C_k$
\begin{itemize}
    \item Primera propuesta
    $$\alpha_k : = \frac{C_k}{\sum_{j=1}^n C_j}  \phantom{abcde} \alpha_1 + ... + \alpha_n = 1$$
- No considera el valor del dinero en el tiempo \\
- No considera intereses
\item Segunda propuesta
$$\alpha_k = \frac{C_k (1+j) ^{-tk}}{\sum_{j=1}^{n}C_j(1+i)^{-tj}}$$
- Sí considera el valor del dinero en el tiempo \\
- Sí considera intereses
\end{itemize}
\begin{center}
    

\tikzset{every picture/.style={line width=0.75pt}} %set default line width to 0.75pt        

\begin{tikzpicture}[x=0.75pt,y=0.75pt,yscale=-1,xscale=1]
%uncomment if require: \path (0,300); %set diagram left start at 0, and has height of 300


% Text Node
\draw (99,42) node [anchor=north west][inner sep=0.75pt]   [align=left] {1a Propuesta};
% Text Node
\draw (68,74.4) node [anchor=north west][inner sep=0.75pt]    {$ter\ :=\ \frac{\sum\limits _{k\ =\ 1}^{n} C_{k} \ tk}{\sum\limits _{j\ =\ 1}^{n} C_{j}} \ $};
% Text Node
\draw (290,41) node [anchor=north west][inner sep=0.75pt]   [align=left] {2a Propuesta};
% Text Node
\draw (257,74.4) node [anchor=north west][inner sep=0.75pt]    {$\tilde{d} \ =\ t\ =\ \frac{\sum\limits _{k\ =\ 1}^{n} C_{k} \ ( 1+i)^{-tk} \ tk}{\sum\limits _{j\ =\ 1}^{n} C_{j} \ ( 1+i)^{-tj}} \ $};


\end{tikzpicture}

\end{center}

$\rightarrow$ A $t_{ET}$ se le conoce como tiempo ``equited time``.\\
$\rightarrow$ A $\tilde{d}$ se le conoce como ``duración``.\\ \\
Para el ejemplo $\bigstar$: 
\begin{align*}
  t_{ET} =& \frac{40(1)-30(2)+20(3)+20(4)-100(5) + 150(6)}{40-30+20+20-100+150}\\
=& \frac{40-\cancel{60}+\cancel{60}+80-500+900}{100}\\
=& 5.2
\end{align*}
En promedio tardamos 5.2 años en recuperar la inversión.\\ \\ 
Para la duración hay que day una tasa de interés. \\
Supongamos que en el ejemplo $\bigstar$ $i=3\%$
$$\tilde{d} = \frac{40(103)^{-1}\cdot 1-30(103)^{-2}\cdot2+20(103)^{-3}\cdot 3+...+150(103)^{-6}\cdot 6}{40(103)^{-1}-30(103)^{-2}+20(103)^{-3}+...+150(103)^{-6}}$$\\
\begin{definition}
Con las mimas hipótesis de la definición del periodo de recuperación, definimos el periodo de recuperación modificado por la tasa $i$ como
$$\tilde{t}_c := min \{ l : -\sum_{j=0}^k C_i(1+i)^{-tj} \leq \sum_{j=k+1}^n C_i(1+i)^{-tj}\}$$
\end{definition}
\textbf{Ejemplo:}\\
En un bono cupón cero ¿$t_{\underline{c}}$?¿$t_{ET}$?¿$\tilde{d}$?\\
$\underline{c} = (-p,c) \phantom{abc} t_{\underline{c}} = n$\\
Pues $p=Cv^{\circ n}\overset{n}{\leq} C$
\begin{center}
    

\tikzset{every picture/.style={line width=0.75pt}} %set default line width to 0.75pt        

\begin{tikzpicture}[x=0.75pt,y=0.75pt,yscale=-1,xscale=1]
%uncomment if require: \path (0,300); %set diagram left start at 0, and has height of 300


% Text Node
\draw (68,80.4) node [anchor=north west][inner sep=0.75pt]    {$t_{ET} =\frac{C\cdot n}{C} =\ n\ $};
% Text Node
\draw (214,79.4) node [anchor=north west][inner sep=0.75pt]    {$\tilde{d} =\frac{C\cdot V^{n} n}{CV^{n}} =\ n\ $};


\end{tikzpicture}

\end{center}

\begin{definition}
Sea $\underline{C}=(C_0,C_1,...,C_n)$, $\underline{t}= (0,t_1,...,t_n)$ un proyecto de inversión. Se define el valor presente neto a la tasa $i$ del proyecto $\underline{C}$ como:
\begin{align*}
    \psi_{\underline{c}}(i)=& C_0 + C_1(1+i)^{-t_1}+C_2(1+i)^{-t_2}+...+C_n(1+i)^{-t_n}\\
    =& \sum_{j=0}^{n}C_j(1+i)^{-t_j}
\end{align*}
\end{definition}
\textbf{Ejemplo:}\\
$\underline{C}= (-900,200,200,200,200,200)$

    \begin{align*}
      1)\phantom{abc} \psi_{\underline{c}}(5\%) =& -900+200(1.05)^{-1}+200(1.05)^{-2}+200(1.05)^{-3}+...+200(1.05)^{-5}\\
        =& -900+ 200\ax{\angl{5}5\%}\\ %casita
        =& -34.1046 \\ \\
      2)\phantom{ab} \psi_{\underline{c}}(10\%) =& -900 +200 \ax{\angl{5}10\%}\\
        =& -141.84 \\ \\
      3) \phantom{abc} \psi_{\underline{c}}(1\%) =& -900+200 \ax{\angl{5}1\%}\\
      = & 70.68
    \end{align*}
Existen tasas para el que $\psi_{\underline{c}}(i)>0$ y otras para las que $\psi_{\underline{c}}(i)<0$.\\
\begin{definition}
Se dice que el proyecto de inversión $\underline{C}$ es profitable a la tasa $i$, si 
$$\psi_{\underline{c}}>0$$ 
Es decir, si su valor presente neto (VPN) es positivo.
\end{definition}
\textbf{Ejemplo:}\\
En el ejemplo anterior el proyecto no era profitable ni al 5$\%$, ni al 10$\%$. Pero sí profitable al 1$\%$.\\ \\
\textit{Obs.}
\begin{align*}
    \psi_{\underline{c}}(i) > 0 \Leftrightarrow & \sum_{j=0}^{n}C_j(1+i)^{-tj} > 0\\
    \Leftrightarrow & \sum_{j=0}^{n}(A_j-B_j)(1+i)^{-tj}>0\\
    \Leftrightarrow & \sum_{j=1}^{n}A_j(1+i)^{-tj}>\sum_{j=1}^{n}B_j(1+i)^{-tj}\\
    & \text{VP(Entrada de \$)} > \text{VP(Salida de \$)}
\end{align*}
\textbf{Ejemplo:} (continuación)\\
$\underline{C} = (-900,200,200,200,200,200)$\\
¿Existe $i \in \mathbb{R}$ tal que $\psi_{\underline{c}}(i) = 0$?
\begin{align*}
    \psi_{\underline{c}}(i) = 0 \Leftrightarrow & -900+200\ax{\angl{5}i\%} = 0\\
    \Leftrightarrow & 200  \ax{\angl{5}i\%} = 900\\
    \Leftrightarrow &  \ax{\angl{5}i\%} = 4.5\\
    \Leftrightarrow & i = 3.618\%
\end{align*}
Si $i \in (0,0.03618)$ \phantom{abcde}
$\psi_{\underline{c}}>0$\\
Si $i \in (0.03618,\infty)$ \phantom{abcd}
$\psi_{\underline{c}}<0$\\ \\
Si un banco les ofreciera que una vez de entrar al proyecto de inversión, simplemente hiciera depósitos, ¿Qué tasa le tendrían que ofrecer? \\
Tendría que ofrecer $i\in (0,0.3618)$\\
\begin{definition}
Dados 2 proyectos de inversión $\underline{C}^{I}$ y $\underline{C}^{II}$. Se prefiere al proyecto $\underline{C}^{I}$ en vez del proyecto $\underline{C}^{II}$ a la tasa $i$.
$$\psi_{\underline{c}^I}(i) >\psi_{\underline{c}^{II}}(i)$$
Se denota como:
\begin{center}
    

\tikzset{every picture/.style={line width=0.75pt}} %set default line width to 0.75pt        

\begin{tikzpicture}[x=0.75pt,y=0.75pt,yscale=-1,xscale=1]
%uncomment if require: \path (0,300); %set diagram left start at 0, and has height of 300


% Text Node
\draw (185,81.4) node [anchor=north west][inner sep=0.75pt]    {$\underline{C}_{I}$};
% Text Node
\draw (221,80.4) node [anchor=north west][inner sep=0.75pt]    {$\bigr\}$};
% Text Node
\draw (243,81.4) node [anchor=north west][inner sep=0.75pt]    {$\underline{C}_{II}$};
% Text Node
\draw (215,100) node [anchor=north west][inner sep=0.75pt]  [font=\tiny]  {$\overline{VPN} ,i$};


\end{tikzpicture}

\end{center}
\end{definition}
\textbf{Ejemplo:}
$$\underline{C}^{II} = (-900,1000)$$
$\psi_{\underline{c}^{II}}(1\%) = -900 + 1000 (1.01)^{-1}=90.09$\\
$\psi_{\underline{c}^{II}}(5\%)=-900 + 1000 (1.05)^{-1}= 52.38$\\
$\psi_{\underline{c}^{II}}(10\%)=-900 + 1000 (1.10)^{-1}= 9.09$
$$\underline{C}^{I} = (-900,200,...,200)$$
$\psi_{\underline{c}^{I}}(1\%) = 70.68$\\
$\psi_{\underline{c}^{I}}(5\%) = 34.1$\\
$\psi_{\underline{c}^{I}}(10\%) = -141.84$

\tikzset{every picture/.style={line width=0.75pt}} %set default line width to 0.75pt        

\begin{tikzpicture}[x=0.75pt,y=0.75pt,yscale=-1,xscale=1]
%uncomment if require: \path (0,300); %set diagram left start at 0, and has height of 300


% Text Node
\draw (185,81.4) node [anchor=north west][inner sep=0.75pt]    {$\Rightarrow\underline{C}^{I}$};
% Text Node
\draw (230,80.4) node [anchor=north west][inner sep=0.75pt]    {$\bigr\}$};
% Text Node
\draw (255,81.4) node [anchor=north west][inner sep=0.75pt]    {$\underline{C}^{II}$};
% Text Node
\draw (223,100) node [anchor=north west][inner sep=0.75pt]  [font=\tiny]  {$\overline{VPN} ,1\%$};


\end{tikzpicture}
\tikzset{every picture/.style={line width=0.75pt}} %set default line width to 0.75pt        

\begin{tikzpicture}[x=0.75pt,y=0.75pt,yscale=-1,xscale=1]
%uncomment if require: \path (0,300); %set diagram left start at 0, and has height of 300


% Text Node
\draw (185,81.4) node [anchor=north west][inner sep=0.75pt]    {$\Rightarrow\underline{C}^{I}$};
% Text Node
\draw (230,80.4) node [anchor=north west][inner sep=0.75pt]    {$\bigr\}$};
% Text Node
\draw (255,81.4) node [anchor=north west][inner sep=0.75pt]    {$\underline{C}^{II}$};
% Text Node
\draw (223,100) node [anchor=north west][inner sep=0.75pt]  [font=\tiny]  {$\overline{VPN} ,5\%$};


\end{tikzpicture}
\tikzset{every picture/.style={line width=0.75pt}} %set default line width to 0.75pt        

\begin{tikzpicture}[x=0.75pt,y=0.75pt,yscale=-1,xscale=1]
%uncomment if require: \path (0,300); %set diagram left start at 0, and has height of 300


% Text Node
\draw (185,81.4) node [anchor=north west][inner sep=0.75pt]    {$\Rightarrow\underline{C}^{I}$};
% Text Node
\draw (230,80.4) node [anchor=north west][inner sep=0.75pt]    {$\bigr\}$};
% Text Node
\draw (255,81.4) node [anchor=north west][inner sep=0.75pt]    {$\underline{C}^{II}$};
% Text Node
\draw (223,100) node [anchor=north west][inner sep=0.75pt]  [font=\tiny]  {$\overline{VPN}$};


\end{tikzpicture} \\ \\
Existirá una tasa $i$ tal que:

\tikzset{every picture/.style={line width=0.75pt}} %set default line width to 0.75pt        

\begin{tikzpicture}[x=0.75pt,y=0.75pt,yscale=-1,xscale=1]
%uncomment if require: \path (0,300); %set diagram left start at 0, and has height of 300


% Text Node
\draw (185,81.4) node [anchor=north west][inner sep=0.75pt]    {$\underline{C}^{I}$};
% Text Node
\draw (221,80.4) node [anchor=north west][inner sep=0.75pt]    {$\bigr\}$};
% Text Node
\draw (243,81.4) node [anchor=north west][inner sep=0.75pt]    {$\underline{C}^{II}$};
% Text Node
\draw (215,100) node [anchor=north west][inner sep=0.75pt]  [font=\tiny]  {$\overline{VPN} ,i$};


\end{tikzpicture}\\
Es equivalente \phantom{ab}
$\psi_{\underline{c}^{I}}(i) \geq \psi_{\underline{c}^{II}}(i) $
\begin{align*}
     \Leftrightarrow & -900+200\ax{\angl{5}i} > -900+1000(1+i)^{-1}\\ 
     \Leftrightarrow & 200 \ax{\angl{5}i} > 1000(1+i)^{-1}
\end{align*}
\begin{definition}
Sean $\underline{C}^{I}$ y $\underline{C}^{II}$ dos proyectos de inversión. \\ \\
1) Se dice que se prefiere a $\underline{C}^{I}$ en vez de $\underline{C}^{II}$, en periodo de recuperación si $t_{\underline{c}^{I}}<t_{\underline{c}^{II}}$\\
Se denota como: 

\begin{center}
    \tikzset{every picture/.style={line width=0.75pt}} %set default line width to 0.75pt        

\begin{tikzpicture}[x=0.75pt,y=0.75pt,yscale=-1,xscale=1]
%uncomment if require: \path (0,300); %set diagram left start at 0, and has height of 300


% Text Node
\draw (185,81.4) node [anchor=north west][inner sep=0.75pt]    {$\underline{C}^{I}$};
% Text Node
\draw (221,80.4) node [anchor=north west][inner sep=0.75pt]    {$\bigr\}$};
% Text Node
\draw (243,81.4) node [anchor=north west][inner sep=0.75pt]    {$\underline{C}^{II}$};
% Text Node
\draw (197,99) node [anchor=north west][inner sep=0.75pt]  [font=\tiny]  {$PR$};


\end{tikzpicture}
\end{center}
2) Se dice que se prefiere a $\underline{C}^{I}$ en vez de $\underline{C}^{II}$ en equated time si $t_{ET}^{I}<t_{ET}^{II}$\\
Se denota como: 
\begin{center}
    

\tikzset{every picture/.style={line width=0.75pt}} %set default line width to 0.75pt        

\begin{tikzpicture}[x=0.75pt,y=0.75pt,yscale=-1,xscale=1]
%uncomment if require: \path (0,300); %set diagram left start at 0, and has height of 300


% Text Node
\draw (185,81.4) node [anchor=north west][inner sep=0.75pt]    {$\underline{C}^{I}$};
% Text Node
\draw (221,80.4) node [anchor=north west][inner sep=0.75pt]    {$\bigr\}$};
% Text Node
\draw (243,81.4) node [anchor=north west][inner sep=0.75pt]    {$\underline{C}^{II}$};
% Text Node
\draw (217,102) node [anchor=north west][inner sep=0.75pt]  [font=\tiny]  {$\overline{ET}$};


\end{tikzpicture}

\end{center}
3) Se dice que se prefiere a  $\underline{C}^{I}$ en vez de  $\underline{C}^{II}$ en duración, si:
\begin{align*}
    \tilde{d}_{I,i} \leq \tilde{d}_{II,i}\\
    \underline{C}^{I}\phantom{a} \Big \} \phantom{a} \underline{C}^{II}
\end{align*}
\end{definition}
Cosas feas del CRITERIO VPN\\ \\
1) $\psi_{\underline{c}}(i)$ es tediosa de calcular. \\
2) Existen algunas $i$ tal que $\psi_{\underline{c}^{I}}(i)>\psi_{\underline{c}^{II}}(i)$ y otras que $\psi_{\underline{c}^{I}}(i)<\psi_{\underline{c}^{II}}(i)$\\ \\
\textit{Obs.}
\begin{align*}
    \frac{\partial}{\partial i} \psi_{\underline{c}}(i) = & \sum_{j=0}^{n} C_j(1+i)^{-tj}\\
    =&\frac{\partial}{\partial i}(C_0 + C_1 (1+i)^{-t_1}+...+ C_n(1+i)^{-t_n}\\
    =& -C_1(1+i)^{-t_n-1}t_1-...-C_nt_n(1+i)^{-t_n-1}\\
    =& -(1+i)^{-1}\cdot\tilde{d}\cdot\sum_{k=1}^{n}C_k(1+i)^{-tk}\\
    =& -(1+i)^{-1}\cdot\tilde{d}[\psi_{\underline{c}}(i)-C]
\end{align*}
\newpage
Recordatorio \\
$\tilde{d} = \frac{\sum_{j=1}^{n}C_j t_j(1+i)^{-tj}}{\sum_{j=1}^{n}C_j(1+i)^{-tj}}$\\ \\
Dijimos ya también
\begin{align*}
    \frac{\partial}{\partial i}\psi_{\underline{c}}(i) = & -(1+i)^{-1}\sum_{j=1}^{n}C_j t_j (1+i)^{-tj} = -(1+i)^{-1}[\tilde{d}\cdot\sum_{j=1}^{n}C_j(1+i)^{-tj}]\\
    =& -(1+i)^{-1}\tilde{d}[\sum_{j=1}^{n}C_j(1+i)^{-tj}-C_0]\\
    =& -(1+i)^{-1}\tilde{d}\cdot[\psi_{\underline{c}}(i)-C_0]= (1+i)\tilde{d}C_0-(1+i)^{-1}\tilde{d}\psi_{\underline{c}}(i)
\end{align*}
Entonces según la definición de derivada
\begin{align*}
\lim_{n \to \infty}\frac{\psi(i+h)-\psi(i)}{h}=(1+i)^{-1}\tilde{d}C_0-(1+i)^{-1}\tilde{d}\psi(i)
\end{align*}
Esto significa que para $h\approx0$
\begin{align*}
    \psi(i+h) -\psi(i) \approx h(1+i)^{-1}\cdot\tilde{d}\cdot C_0 - h(1+i)^{-1}\cdot\tilde{d}\cdot\psi(i)\\
    \psi(i+h) \approx h(1+i)^{-1}\cdot\tilde{d}\cdot C_0 + \psi(i)[1-h(1+i)^{-1}\cdot \tilde{d}] 
\end{align*}
\textcolor{red}{NOTA:} $\tilde{d}_m$ se le conoce como duración modificada.\\
\phantom{abc}\phantom{abc}\phantom{a}$\tilde{d}_m:= \frac{\tilde{d}}{1+i}$ con esa notación.\\
\phantom{abc}\phantom{abc}\phantom{a} $\Rightarrow \psi (1+h) = h\cdot \tilde{d}_m \cdot C_0 + \psi(i)[1-h\cdot\tilde{d}_m]$\\
\phantom{abc}\phantom{abc}\phantom{a} Esta expresión sirve para aproximar el valor presente si conozco la duración modificada.\\
\phantom{abc}\phantom{abc}\phantom{a} Acuérdense que es ``Doloroso`` calcular $\psi_{\underline{c}}(i)$ pues $\psi_{\underline{c}}(i) = \sum_{j=1}^{n}C_j(1+i)^{-tj}$ \\ \\ 
\textcolor{red}{$\rightarrow$Ayer al final de la clase}\\ \\
TIR $\Leftrightarrow$ Tasa Interna de Rendimiento $\big/$ tasa interna de retorno.\\
$\psi_{\underline{c}}(TIR) = 0$ \phantom{abc} EQUIU $\sum_{j=0}^{n}A_j(1+TIR){-tj} = \sum_{j=0}^{n}B_j(1+TIR)^{-tj}$\\
¿Esta definición tiene sentido siempre? (...)\\ \\
\textcolor{red}{Ejemplo 0.1:}\\
Considere los primeros 3 proyectos de inversión.
\begin{table}[h]
\centering
\begin{tabular}{lllll}
Proyecto & $C_0$ & $C_1$ & $C_2$ & $C_3$ \\
I        & -2000 & 500   & 500   & 500   \\
II       & -2000 & 500   & 1800  & 0     \\
III      & -2000 & 1800  & 500   & 0    
\end{tabular}
\end{table}
\begin{enumerate}
    \item Para que el proyecto I, nótese que $2000<500+500+500$, entonces el periodo de recuperación es 3.
\item Para el proyecto II, nótese que $2000<500+1800$, entonces el periodo de recuperación es 2.
\item Para el proyecto III, nótese que $2000<1800+500$, entonces el periodo de recuperación es 2.
\end{enumerate}
Con tasa del 10$\%$ el valor presente neto de los tres proyectos está dada por. 
\begin{enumerate}
    \item Proyecto I \\
    $-2000+500(1.1)^{-1} + 500(1.1)^{-2} +5000(1.1)^{-3} = 2624.3371$
\item Proyecto II \\
$-2000+500(1.1)^{-1}+1800(1.1)^{-2}=-57.8512$
\item Proyecto III \\
$-2000+1800(1.1)^{-1}+500(1.1)^{-2}$
\end{enumerate}
¿Qué proyecto elegirían?\\
$\rightarrow$ Depende de qué quiero.\\
Si lo que nos interesa es recuperar pronto mi inversión, elijo el proyecto II o III $\phantom{ab}$
$\underline{C}_{II} \sim_{P_n} \underline{C}_{III}$\\ \\
$\underline{C}_{II} \succ_{P_n} \underline{C}_{III}$ y $\underline{C}_{III}\succ_{P_n} \underline{C}_{I}$\\
y con este criterio de valor presente\\ \\
$\underline{C}_I \succ_{VPN, 10\%}\underline{C}_{III}   \succ_{VPN,10\%}\underlind{C}_{II}$



%%% Página 16
\textcolor{blue}{Ejemplo 0.2}\\
Considere el proyecto $(-4,000,2,000,4,000)$ para encontrar una $TIR$.
\begin{align*}
    \varphi_{\underline{C}}(TIR) &= 0 \Leftrightarrow \sum_{j=0}^n C_j(1+TIR)^{-t_j} =0\\
    \Leftrightarrow & -4,000(1+TIR)^0 + 2,000(1+TIR)^{-1} + 4,000(1+TIR)^{-2} = 0\\
    \Leftrightarrow & 4,000=4,000x^2 + 2,000x, x=(1+TIR)\\
    &u=\frac{-1\pm\sqrt{1+4\cdot 2\cdot 2}}{4} \begin{cases}
        &\frac{-1+\sqrt{17}}{4}\\
        &\frac{-1-\sqrt{17}}{4}
    \end{cases}
\end{align*}
Es decir,
$$1+TIR = \begin{cases} 
&\frac{4}{-1+\sqrt{17}}\\
&\frac{4}{-1-\sqrt{17}}
\end{cases}$$
Por lo tanto,
$$TIR =\begin{cases} 
&\frac{4}{-1+\sqrt{17}}-1\\
&\frac{4}{-1-\sqrt{17}}-1
\end{cases} = \begin{cases} 
&\frac{5-\sqrt{17}}{\sqrt{17}-1}\\
-&\frac{5-\sqrt{17}}{1+\sqrt{17}}
\end{cases}= \begin{cases} 
&0.280776\\
-&1.780776
\end{cases}
$$

\begin{definition}
Para un proyecto $\underline{C} =(C_0,...,C_n)$ se defina el valor presente neto de $\underline{C}$ como $\varphi_{\underline{C}}(v) = C_0 + C_1 v^{t_1} + C_2v^{t_2} + ...+ C_nv^{t_n}$.
\end{definition}
 
En este caso para el proyecto de ejemplo:\\
\textcolor{blue}{Ejemplo 0.3}\\
$\underline{C} = (1,000,-3,600,4,320,-1,728)$. En este caso $\varphi_{\underline{C}}(v) = 1,000 -3,600v + 4,320v^2 - 1,728v^3$ con $\varphi_{\underline{C}} (v=\frac{1}{1.1} = -0.751315$ y $\varphi_{\underline{C}}(v=\frac{1}{1.2}) = 0$\\

Obsérvese que $\varphi_{\underline{C}}(v)$ es un polinomio en $v$, preguntarnos si $\varphi_{\underline{C}}(v) =0$ significa preguntarnos acerca de las raices de este polinomio.\\

\textcolor{blue}{¿Cuáles son las raices de este polinomio?} Según el Teorema fundamental del Álgebra, puede tener hasta 3 raices diferentes.\\

Ya sabemos que $v=\frac{1}{1.2}=\frac{5}{6}$ es una raiz\\
\textcolor{blue}{¿Cómo obtenemos los otros 2?} División de polinomios
\begin{center}
    

\tikzset{every picture/.style={line width=0.75pt}} %set default line width to 0.75pt        

\begin{tikzpicture}[x=0.75pt,y=0.75pt,yscale=-1,xscale=1]
%uncomment if require: \path (0,300); %set diagram left start at 0, and has height of 300

%Straight Lines [id:da7712050352858931] 
\draw    (100,51) -- (323.33,51) ;
%Straight Lines [id:da7661410289262178] 
\draw    (100,51) -- (100,75) ;
%Straight Lines [id:da7492631596021389] 
\draw    (103,96) -- (215.33,96) ;
%Straight Lines [id:da6195931716764791] 
\draw    (160,133) -- (272.33,133) ;
%Straight Lines [id:da8039874083911014] 
\draw    (212,167) -- (324.33,167) ;

% Text Node
\draw (62,47.4) node [anchor=north west][inner sep=0.75pt]  [font=\footnotesize]  {$v-\frac{5}{6}$};
% Text Node
\draw (102,57.4) node [anchor=north west][inner sep=0.75pt]  [font=\footnotesize]  {$-1,728v^{3} +4,370v^{2} -3,600v+1,000$};
% Text Node
\draw (102,32.4) node [anchor=north west][inner sep=0.75pt]  [font=\footnotesize]  {$-1,728v^{2} +2,280v -1,200$};
% Text Node
\draw (102,78.4) node [anchor=north west][inner sep=0.75pt]  [font=\footnotesize]  {$\ \ 1,728v^{3} -1,440v^{2}$};
% Text Node
\draw (169,98.4) node [anchor=north west][inner sep=0.75pt]  [font=\footnotesize]  {$2,280v^{2} -3,600v$};
% Text Node
\draw (160,116.4) node [anchor=north west][inner sep=0.75pt]  [font=\footnotesize]  {$-2,280v^{2} +2,400v$};
% Text Node
\draw (221,136.4) node [anchor=north west][inner sep=0.75pt]  [font=\footnotesize]  {$-1,200v+1,000$};
% Text Node
\draw (221,152.4) node [anchor=north west][inner sep=0.75pt]  [font=\footnotesize]  {$-1,200v+1,000$};
% Text Node
\draw (296,169.4) node [anchor=north west][inner sep=0.75pt]  [font=\footnotesize]  {$0$};


\end{tikzpicture}

\end{center}

Con esto, solo tomamos el polinomio $-1,728^2 + 2,280v-1,200=0$. Utilizando fórmula general $v_1=v_2=\frac{5}{6}$ por lo que, $\varphi_{\underline{C}}(v) = -1,728(v-\frac{5}{6})^3$, $v=\frac{5}{6}$ es una raiz de multiplicidad $3$ de $\varphi_{\underline{C}}$\\
$\therefore$ La $TIR$ es única y $TIR = 20\%$

\textcolor{blue}{Ejemplo 0.5}\\
Considerérese dos proyectos:\\
Proyecto $A$: $(-9,000,6,000,5,000,4,000)$\\
Proyecto $B$: $(-9,000,1,800,1,800,1,800,...)$\\
- Notese que para el proyecto $A$,
\begin{align*}
    \varphi_A(v) &= -9,000+6,000v+5,000v^2+4,000v^3\\
    &= 4,000\bigg(v^3+\frac{5}{4}v^2+\frac{8}{2}v +\big(-\frac{9}{4}\big)\bigg)\\
    &= 4,000\big(v-\frac{3}{4}\big)(v^2+2v+3)\\
    &= 4,000\big(v-\frac{3}{4}\big)(v-c)(v-\Bar{C}),\Bar{C}:=-1+\sqrt{2}\sqrt{-1}
\end{align*}
Entonces la única $TIR$ real, satisface
$$\frac{1}{1+TIR} = \frac{3}{4}, i.e.\quad TIR=\frac{1}{3}$$
- Para el proyecto $B_1$
\begin{align*}
    \varphi_B(v) &= - 9,000 + 1,800v + 1,800v^2+...\\
    &= -9,000+1,800\frac{v}{1-v} =0, i.e. v=\frac{5}{6}
\end{align*}
Por tanto, la única $TIR$ satisface $\frac{1}{1+TIR}=\frac{5}{6}$, $i.e.$ $TIR=\frac{1}{5}=20\%$

Bajo el criterio de la $TIR$, se selecciona el proyecto $A$, pues $TIR_A=\frac{1}{3}>\frac{1}{5}=TIR_B$, sin embargo, el proyecto $B$ hace pagos de $\$1,800$ indefinidamente, mientras que el proyecto $A$ sólo lo hace durante tres periodos.\\

\textcolor{blue}{Aprendizajes hasta el momento}\\
\begin{itemize}
    \item La $TIR$ no es única
    \item El Teorema fundamental del Álgebra garantiza la existencia en $\mathbb{C}$
    \item No necesariamente esta en $\mathbb{R}$, y además nos gustaría que $TIR>0$
    \item El $TIR$ del Álgebra solo dice que existe, pero no dice como encontrarla y en general es difícil
\end{itemize}
\begin{definition}
Sean $\underline{C}^I$ y $\underline{C}^{II}$ dos proyectos de inversión se dice que prefiero al proyecto $I$ en vez del $II$ bajo el criterio de la $TIR$ si $TIR^I \geq TIR^{II}$ y se denota como
$$\underline{C}^I \succeq_{TIR} \underline{C}^{II}$$
\end{definition}
\textcolor{blue}{\textit{Moraleja General.}} Se podría aceptar un proyecto vs otro bajo un criterio y rechazarlo bajo otro criterio

\textit{\textbf{Nota:}} La $TIR$ es la mínima tasa de interés que un banco me tendría que ofrecer para que en vez de entrar en el proyecto de inversión, invirtiera el flujo en el banco.

\begin{definition} $TIR$
\begin{itemize}
\item Difícil de calcular
\item Involucra raices de polinomios\begin{itemize}
    \item TDA
    \item Raices existen $\mathbb{C}$ aunque nos gustaría que estuvieran en $\mathbb{R}$
    \item El tema de la no unicidad como financieros nos causa conflictos
\end{itemize}
\end{itemize}
\end{definition}

Dos alternativas para obtener la $TIR$ son las tasas: \begin{itemize}
    \item[(1)] Dollar-Weighter
    \item[(2)] Time-Weighter
\end{itemize}
El problema es que son aproximaciones para proyectos con plazo menor o igual a un año.

\section*{Tasa de Interés Dollar-Weighted}
\begin{itemize}
    \item Su interpretación es más fácil si se piensa al proyecto como un fondo de inversión.
    \item Solo sirve para plazos menores o iguales a un año
    \item Utiliza interés simple
    \item Supone que se conocen las siguientes cantidades\begin{itemize}
        \item[1.] Valor del fondo de inversión al inicio del año $V$
        \item[2.] Valor del fondo de inversión al final del año $V_1$
        \item[3.] Las cantidades y fechas de depósitos y retiros realizados durante el año.
    \end{itemize}
    \item $A_k$: La cantidad retirada al tiempo $t_k$
    \item $B_k$: La cantidad depositada al tiempo $t_k$
\end{itemize}

\begin{center}
    

\tikzset{every picture/.style={line width=0.75pt}} %set default line width to 0.75pt        

\begin{tikzpicture}[x=0.75pt,y=0.75pt,yscale=-1,xscale=1]
%uncomment if require: \path (0,300); %set diagram left start at 0, and has height of 300

%Straight Lines [id:da4507960847289094] 
\draw    (158,114.9) -- (424,114.9) ;
%Straight Lines [id:da6217468544202908] 
\draw    (207.83,110.23) -- (207.83,120.4) ;
%Straight Lines [id:da7673121282711733] 
\draw    (367.33,110.07) -- (367.33,120.23) ;
%Straight Lines [id:da5179628248428941] 
\draw    (166.83,110.23) -- (166.83,120.4) ;
%Straight Lines [id:da150051487012828] 
\draw    (247.83,110.23) -- (247.83,120.4) ;
%Straight Lines [id:da7166658914817258] 
\draw    (307.83,110.23) -- (307.83,120.4) ;
%Straight Lines [id:da4030082983828851] 
\draw    (415.83,110.07) -- (415.83,120.23) ;

% Text Node
\draw (267.6,120.5) node [anchor=north west][inner sep=0.75pt]   [align=left] {. . .};
% Text Node
\draw (161.9,124.7) node [anchor=north west][inner sep=0.75pt]  [font=\small]  {$0$};
% Text Node
\draw (202.4,124.7) node [anchor=north west][inner sep=0.75pt]  [font=\small]  {$t_{1}$};
% Text Node
\draw (243.4,124.7) node [anchor=north west][inner sep=0.75pt]  [font=\small]  {$t_{2}$};
% Text Node
\draw (300.8,124.3) node [anchor=north west][inner sep=0.75pt]  [font=\small]  {$t_{k}$};
% Text Node
\draw (360.8,123.8) node [anchor=north west][inner sep=0.75pt]  [font=\small]  {$t_{m}$};
% Text Node
\draw (410.8,124.3) node [anchor=north west][inner sep=0.75pt]  [font=\small]  {$1$};
% Text Node
\draw (326.1,122) node [anchor=north west][inner sep=0.75pt]   [align=left] {. . .};
% Text Node
\draw (199.9,92.2) node [anchor=north west][inner sep=0.75pt]  [font=\small]  {$A_{1}$};
% Text Node
\draw (240.4,92.2) node [anchor=north west][inner sep=0.75pt]  [font=\small]  {$A_{2}$};
% Text Node
\draw (300.9,91.7) node [anchor=north west][inner sep=0.75pt]  [font=\small]  {$A_{k}$};
% Text Node
\draw (357.4,92.7) node [anchor=north west][inner sep=0.75pt]  [font=\small]  {$A_{n-1}$};
% Text Node
\draw (198,145.2) node [anchor=north west][inner sep=0.75pt]  [font=\small]  {$B_{1}$};
% Text Node
\draw (238.5,145.2) node [anchor=north west][inner sep=0.75pt]  [font=\small]  {$B_{2}$};
% Text Node
\draw (299,144.7) node [anchor=north west][inner sep=0.75pt]  [font=\small]  {$B_{k}$};
% Text Node
\draw (355.5,145.7) node [anchor=north west][inner sep=0.75pt]  [font=\small]  {$B_{m}$};
% Text Node
\draw (156,146.2) node [anchor=north west][inner sep=0.75pt]  [font=\small]  {$V_{0}$};
% Text Node
\draw (402.4,92.7) node [anchor=north west][inner sep=0.75pt]  [font=\small]  {$V_{1}$};


\end{tikzpicture}

\end{center}

¿Cuál es la relación entre $V_1$ y $V_0$?
$$V_1 = V_0 \textit{ acumulado } + \textit{ Entradas acumuladas } - \textit{ Salidas acumulados}$$

Bajo interés simple, ¿cómo escribimos esta relación?
\begin{align*}
    V_1 &= V_0(1+i\cdot 1) + B_1[1+i(1-t_1)] + B_2[1+ i(1-t_2)]\\
    &  + ... + B_m [1+i(1-t_m)] - A_1[1+i(1-t_1)] - A_2[1+i(1-t_2)]  \\
    & - ... - A_m[1+i(1-t_m)]
\end{align*}

Intentemos despejar $i$.
\begin{align*}
    V_1 -& V_0 - B_1 - B_2 - ... - B_m + A_1 + A_2 + ... + A_m\\
    &= i[V_0 + B_1(1-t_1) + B_2(1-t_2) + ... + B_m(1-t_m) \\
    &- A_1(1-t_1) - A_2(1-t_2) - ... - A_m(1-t_m)]\\
    \Leftrightarrow  V_1 -& V_0 - \sum_{k=1}^m B_k + \sum_{k=1}^m A_k = i[V_0 + \sum_{k=1}^m B_k(1-t_k) - \sum_{k=1}^m A_k(1-t_k)]\\
    \Rightarrow i &= \frac{V_1-V_0-\sum_{k=1}^m(B_k-A_k)}{V_0 + \sum_{k=1}^m(B_k-A_k)(1-t_k)} = \frac{V_1-V_0 + \sum_{k=1}^m C_k}{V_0 - \sum_{k=1}^m C_k(1+t_k)}
\end{align*}
donde $C_k := \underbrace{A_k}_{Salida} - \underbrace{B_k}_{Entrada}$, $C_k$ es la salida neta.

\end{document}
